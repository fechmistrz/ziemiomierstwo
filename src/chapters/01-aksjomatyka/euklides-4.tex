%

\subsection{Księga IV}
\subsubsection{Definicje}
\begin{enumerate}
	\item [4.1] Dwie figury prostokreślne są wpisane jedna w drugą, jeśli wierzchołki pierwszej leżą na bokach drugiej, a boki drugiej przechodzą przez wierzchołki pierwszej.
	\item [4.2] Podobnie definiuje się figurę, która jest opisana na innej figurze.
	\item [4.3] Figura prostokreślna jest wpisana w okrąg, kiedy wszystkie jej wierzchołki leżą na tym okręgu.
	\item [4.4] Figura prostokreślna jest opisana na okręgu, kiedy każdy jej bok jest styczny do tego okręgu.
	\item [4.5] Podobnie definiuje się, że okrąg jest wpisany w figurę prostokreślną.
	\item [4.6] Koło opisuje się na figurze prostokreślnej, kiedy okrąg (jego brzeg) przechodzi przez wszystkie wierzchołki figury.
	\item [4.7] Linia prosta (właściwie: odcinek) kreśli się w kole, gdy jej końca leżą na brzegu tego koła.
\end{enumerate}

\subsubsection{Twierdzenia}
\begin{enumerate}
	\item [4.1] Wpisać odcinek krótszy od średnicy w dany okrąg.
	\item [4.2] Wpisać trójkąt podobny do danego w okrąg. % equiangular
	\item [4.3] Opisać trójkąt podobny do danego na okręgu.
	\item [4.4] Wpisać okrąg w dany trójkąt.
	\item [4.5] Opisać okrąg na danym kwadracie. \index{okrąg!opisany} % TODO OKRG NA OKREGU???
	\item [4.6] Wpisać kwadrat w dany okrąg.
	\item [4.7] Opisać kwadrat na danym okręgu.
	\item [4.8] Wpisać okrąg w dany kwadrat.
	\item [4.9] Opisać okrąg na danym kwadracie.
	\item [4.10] Wykreślić trójkąt równoramienny, którego kąt przy podstawie jest podwojeniem kąta przy wierzchołku (o kątach $\pi/5$, $2\pi/5$, $2\pi/5$; taki trójkąt nazywa się czasami złotym). \index{trójkąt!złoty}
	\item [4.11] Wpisać pięciokąt foremny\footnote{Euklides mówi tu o pięciokącie równobocznym i jednocześnie równokątnym.} w dany okrąg.
	\item [4.12] Opisać pięciokąt foremny na danym okręgu. \index{pięciokąt|see{wielokąt}}
	\index{wielokąt!pięciokąt}
	\item [4.13] Wpisać okrąg w dany pięciokąt równoboczny i równokątny.
	\item [4.14] Opisać okrąg na danym pięciokącie równobocznym i równokątnym
	\item [4.15] Wpisać sześciokąt równoboczny i równokątny w dany okrąg.
	\index{sześciokąt|see{wielokąt}}
	\index{wielokąt!sześciokąt}
	\item [4.16] Wpisać piętnastokąt równoboczny i równokątny w dany okrąg.
	\index{piętnastokąt|see{wielokąt}}
	\index{wielokąt!piętnastokąt}
\end{enumerate}

%