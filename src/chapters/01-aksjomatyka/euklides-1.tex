\subsection{Księga I}	
Treść pierwszej księgi przytoczy Eves \cite[s. 375-379]{eves1_1972}.

\subsubsection{Definicje}	
\begin{enumerate}	
    \item [1.1] Punkt.
    \item [1.2] Odcinek.
    \item [1.3] Końce odcinka.
    \item [1.4] Prosta.
    \item [1.5] Powierzchnia.
    \item [1.6] Krawędzie powierzchni.
    \item [1.7] Płaszczyzna.
    \item [1.8] Kąt płaski.
    \item [1.9] Kąt prostoliniowy (?)
    \item [1.10] Kąt prosty, proste prostopadłe.
    \item [1.11] Kąt rozwarty.
    \item [1.12] Kąt ostry.
    \item [1.13] Kres albo granica (?).
    \item [1.14] Figura.
    \item [1.15] Koło i okrąg.
    \item [1.16] Środek koła.
    \item [1.17] Średnica koła.
    \item [1.18] Półokrąg, środek półokręgu.
    \item [1.19] Figura prostokreślna, trójkąt, czworobok (czworokąt), wielobok.
    \item [1.20] Trójkąt równoboczny, równoramienny i różnoboczny.
    \item [1.21] Trójkąt prostokątny, rozwartokątny, ostrokątny.
    \item [1.22] Kwadrat, prostokąt (nie będący kwadratem), romb (nie będący kwadratem), równoległobok (nie będący kwadratem, prostokątem ani rombem).
    \item [1.23] Proste równoległe.
\end{enumerate}	
	
\subsubsection{Postulaty}	
\begin{enumerate}	
    \item [1.1] Przez każde dwa punkty przechodzi prosta.
    \item [1.2] Każdy odcinek można przedłużyć do prostej.
    \item [1.3] Z każdego środka można wykreślić okrąg o dowolnym promieniu.
    \item [1.4] Wszystkie kąty proste są sobie równe.
    \item [1.5] Jeżeli prosta przecinająca dwie proste tworzy z nimi kąty jednostronnie wewnętrzne o sumie mniejszej niż $\pi$, to te dwie proste przecinają się po tej stronie.
\end{enumerate}	
	
Jak łatwo zauważyć, sformułowanie ostatniego postulatu używa więcej słów niż pozostałe razem wzięte; wbrew przekonaniu, że postulaty miały wyrażać treści oczywiste i proste.	
Piąty postulat wydawał się bardziej skomplikowany, więc nasuwał podejrzenie, że wynika z poprzednich czterech.	
Zauważył to już Proklos zwany Diadochem (410-485):
\index[persons]{Proklos zwany Diadochem}%
\emph{,,Nie jest możliwe, aby uczony tej miary co Euklides godził się na obecność tak długiego postulatu w aksjomatyce -- obecność postulatu wzięła się z pospiesznego kończenia przez niego Elementów, tak aby zdążyć przed nadejściem słusznie oczekiwanej rychłej śmierci; my zatem -- czcząc jego pamięć -- powinniśmy ten postulat usunąć lub co najmniej znacznie uprościć.''}	
	
Wiele osób próbowało stawić czoło wyzwaniu postawionemu przez Proklosa.	
Bezskutecznie, ponieważ piąty postulat jest niezależny od pozostałych!

\subsubsection{Pojęcia pierwotne}	
\begin{enumerate}	
    \item [1.1] Wyrażenia, które są równe się temu samemu wyrażeniowi, są sobie równe.
    \item [1.2] Równania można dodawać stronami.
    \item [1.3] Równania można odejmować stronami.
    \item [1.4] Wyrażenia, które się pokrywają, są sobie równe.
    \item [1.5] Całość jest większa od części.
\end{enumerate}	
	
\subsubsection{Twierdzenia}	
\begin{enumerate}	
    \item [1.1] Skonstruować trójkąt równoboczny o zadanym boku.
    \item [1.2] Skonstruować odcinek równy danemu, z końcem w zadanym punkcie.
    \item [1.3] Skonstruować różnicę dwóch odcinków.
    \item [1.4] Dane są punkty $A$, $B$, $C$, $D$, $E$, $F$ takie, że $AB = DE$, $BC = EF$, $\angle ABC = \angle DEF$.
    Wtedy trójkąty $\triangle ABC$ i $\triangle DEF$ są przystające.
    \item [1.5] Kąty przy podstawie trójkąta równoramiennego są równe.
    \item [1.6] Boki trójkąta leżące naprzeciw równych kątów są równe.
    \item [1.7] Nie można zbudować innego trójkąta o tych samych dwóch bokach przy tej samej podstawie po tej samej stronie prostej, na której leży podstawa.
    \item [1.8] Dane są punkty $A$, $B$, $C$, $D$, $E$, $F$ takie, że $AB = DE$, $BC = EF$ i $AC = DF$.
    Wtedy kąty $\angle ABC = \angle DEF$ są równe.
    \item [1.9] Podzielić dany kąt na dwie równe części.
    \item [1.10] Podzielić dany odcinek na dwie równe części.
    \item [1.11] Skonstruować prostopadłą do prostej przechodzącą przez dany punkt na tej prostej.
    \item [1.12] Skonstruować prostopadłą do prostej przechodzącą przez dany punkt poza tą prostą.
    \item [1.13] Suma miar kątów przyległych wynosi $\pi$.
    \item [1.14] Jeśli suma miar dwóch kątów o wspólnym ramieniu (i drugich ramionach leżących po różnych stronach) wynosi $\pi$, to są kątami przyległymi.
    \item [1.15] Kąty wierzchołkowe mają równe miary.
    \item [1.16] Kąt przyległy do dowolnego kąta wewnętrznego trójkąta jest większy od pozostałych dwóch kątów wewnętrznych.
    \item [1.17] W każdym trójkącie suma dwóch kątów jest mniejsza od $\pi$.
    \item [1.18] W każdym trójkącie większy bok leży naprzeciw większego kąta.
    \item [1.19] W każdym trójkącie większy kąt leży naprzeciw większego boku.
    \item [1.20] W każdym trójkącie suma dwóch boków jest większa od boku trzeciego.
    \item [1.21] Punkt $D$ leży wewnętrz trójkąta $ABC$. Wtedy $BD + CD < AB + AC$.
    \item [1.22] Z trzech odcinków można skonstruować trójkąt, kiedy suma dwóch krótszych jest dłuższa od trzeciego.
    \item [1.23] Odłożyć kąt na danej prostej, z wierzchołkiem w danym punkcie.
    \item [1.24] Dane są punkty $A$, $B$, $C$, $D$, $E$, $F$ takie, że $AB = DE$, $AC = DF$, $\angle BAC > \angle EDF$. Wtedy $BC > EF$.
    % TODO: https://en.wikipedia.org/wiki/Hinge_theorem
    \item [1.25] Dane są punkty $A$, $B$, $C$, $D$, $E$, $F$ takie, że $AB = DE$, $AC = DF$, $BC > EF$. Wtedy $\angle BAC > \angle EDF$. 
    \item [1.26] Dane są punkty $A$, $B$, $C$, $D$, $E$, $F$ takie, że $\angle ABC = \angle DEF$, $\angle BCA = \angle EFD$, $BC = EF$. Wtedy $AB = DE$, $AC = DF$ i $\angle BAC = \angle EDF$.
    \item [1.27] Jeśli prosta $EF$ przecina dwie inne $AB$ i $CD$ pod tym samym kątem: $\angle AEF = \angle EFD$, to są równoległe do siebie: $AB \parallel CD$.
    \item [1.27] Jeśli prosta $EF$ przecina dwie inne $AB$ i $CD$ w punktach $G$ i $H$ pod tym samym kątem: $\angle EGB = \angle GHD$, to są równoległe do siebie: $AB \parallel CD$.
    \item [1.29] Kąty naprzemianległe są równe.
    \item [1.30] Dwie proste równoległe do trzeciej prostej są też równoległe do siebie.
    \item [1.31] Skonstruować prostą równoległą do danej, przechodzącą przez punkt poza nią.
    \item [1.32] W trójkącie $ABC$ kąt zewnętrzny przy wierzchołku $C$ jest równy sumie kątów wewnętrznych przy wierzchołkach $A$ i $B$.
    \item [1.33] Dane są równe i równoległe odcinki $AB$ i $CD$, wtedy odcinki $AC$ i $BD$ też są równe i równoległe.
    \item [1.34] Przekątna równoległoboku dzieli go na dwa przystające trójkąty.
    \item [1.35] Dwa równoległoboki $ABCD$ i $EBCF$, których drugie podstawy leżą na tej samej prostej mają równe pola.
    \item [1.36] Dwa równoległoboki, które mają równe podstawy i równe wysokości mają równe pola. (uogólnienie poprzedniego)
    \item [1.37] Dwa trójkąty które mają tę samą podstawę i równe wysokości mają równe pola.
    \item [1.38] Dwa trójkąty które mają równe podstawy i równe wysokości mają równe pola.
    \item [1.39] Dane są dwa trójkąty $ABC$ i $ABD$ o równych polach takie, że punkty $C$ i $D$ leżą po tej samej stronie prostej $AB$. Wtedy $AB \parallel CD$.
    \item [1.40] (dziwne) % http://aleph0.clarku.edu/~djoyce/elements/bookI/propI40.html
    \item [1.41] Dany jest równoległobok i trójkąt o tej samej podstawie i równych wysokościach, wtedy pole równoległoboku  jest dwa razy większe od pola trójkąta.
    \item [1.42] Dany jest trójkąt oraz kąt, skonstruować równoległobok o polu takim samym jak trójkąt i danym kącie wewnętrznym.
    \item [1.43] W równoległoboku $ABCD$ wybrano punkt $P$ na przekątnej $AC$ i poprowadzono przez niego proste równoległe do boków, przecinające boki $AB$, $BC$, $CD$, $AD$ w punktach $H$, $F$, $G$, $I$. Wtedy równoległoboki $HBFP$ i $IPGD$ mają równe pola.\footnote{znane po angielsku jako \emph{theorem of the gnomon}, ponieważ Euklides wprowadzi termin gnomon w drugiej definicji drugiej księgi.}
    \item [1.44] Skonstruować równoległobok, którego jeden z boków jest równy danemu odcinkowi, jeden z kątów jest równy danemu kątowi, zaś pole jest równe polu danego trójkąta.
    \item [1.45]  Skonstruować równoległobok, którego jeden z boków jest równy danemu odcinkowi, jeden z kątów jest równy danemu kątowi, zaś pole jest równe polu danego czworokąta.
    \item [1.46] Skonstruować kwadrat.
    \item [1.47] W trójkącie prostokątnym, suma pól kwadratów zbudowanych na przyprostokątnych jest równa polu kwadratu zbudowanego na przeciwprostokątnej.
    \index{twierdzenie!Pitagorasa}
    \item [1.48] Jeżeli suma pól kwadratów zbudowanych na dwóch bokach trójkąta jest równa polu kwadratu zbudowanego na trzecim boku, to trójkąt jest prostokątny.
\end{enumerate}

%