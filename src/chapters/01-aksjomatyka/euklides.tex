%

\index{aksjomaty!Euklidesa|(}%

\section{Aksjomaty Euklidesa}

\todofoot{https://kpbc.umk.pl/dlibra/publication/37/edition/66/content}

\todofoot{https://www.claymath.org/library/historical/euclid/}

Opowiemy krótko o rozwoju geometrii w okresie klasycznym starożytnej Grecji, pomijając przy tym liczne szczegóły, ponieważ światu przekaże je lepiej Proklos zwany Diadochem, operając się na wcześniejszej pracy  Eudemosa z Rodos, obejmującej okres do 355 roku przed Chrystusem.
\index[persons]{Proklos zwany Diadochem}%
\index[persons]{Eudemos z Rodos}%
Później przytoczymy definicje i twierdzenia zgromadzone przez Euklidesa z Aleksandrii.
\index[persons]{Euklides z Aleksandrii}%
(Imię Euklidesa znamy z tylko jednego źródła: komentarza Proklosa!
Pozostali nazywali go autorem Στοιχεῖα, znaczy \emph{Stoicheia}, znaczy \emph{Elementów}).
Interesować nas będzie nie wszystkie trzynaście ksiąg, a:
\begin{itemize}
    \item I~(o~trójkątach),
    \item II (o~prostokątach),
    \item III (o~okręgach),
    \item IV (o~wielokątach),
    \item V (o~proporcjach),
    \item VI (o~podobieństwie), ale także, choć w~mniejszym stopniu,
    \item XI i~XIII (o~wielościanach).
\end{itemize}

Według Eudemosa/Proklosa, początków geometrii greckiej należy szukać w dokonaniach Talesa z Miletu, a więc mniej więcej w VI wieku p.n.e.
Tales zacznie stosować metody dedukcyjne.
\index[persons]{Tales z Miletu}%
Potem będzie na pewno pobierający (a być może u Talesa) nauki Pitagoras i jego szkoła, której uczniowie zaczną tworzyć łańcuchy różnych geometrycznych faktów oraz wniosków z nich. 
\index[persons]{Pitagoras}%
Hipokrates z Chios postara się uporządkować twierdzenia i częściowo uda mu się to.
\index[persons]{Hipokrates z Chios}%
Podobne próby podejmą Teudios z Magnezji oraz Leon (hehe); ale nie wiadomo za dużo ani o nich, ani o ich dziełach poza tym, że miały tytuł \emph{Elementy}.
\index[persons]{Teudios z Magnezji}%
\index[persons]{Leon, uczeń Neokleidesa}%

Proces przekształcania empirycznej geometrii Egipcjan i Babilończyków w systematyczną naukę, której fakty ustala się nie przez obserwacje, ale na drodze rozumowania dedukcyjnego, będzie mieć swoje ukoronowanie około 300 roku p.n.e.
Wtedy wielki sukces odniesie Euklides ze swoim epokowym dziełem -- także \emph{Elementami}, spójnym ciągiem 465 twierdzeń dotyczących geometrii płaskiej, przestrzennej, teorii liczb oraz częściowo algebry.
\index[persons]{Euklides z Aleksandrii}%
Grek formułuje tam pewien minimalny zbiór aksjomatów (czyli zdań przyjętych bez dowodu), z którego sukcesywnie wyprowadza co raz bardziej skomplikowane twierdzenia.
Pogląd, że tak powinno uprawiać się naukę, utrzyma się przez tysiące lat, jeśli nie wieczność; jego książka zaś ukształtuje wykład geometrii aż do początków XIX wieku.
Imponujące!

Oprócz \emph{Elementów}, Euklides napisze jeszcze dzieła takie jak:
\begin{itemize}
    \item \emph{Δεδομένα} (\emph{Dedomena} czyli po polsku \emph{Dane}) o problemach geometrycznych, gdzie znamy część informacji o figurze i musimy odtworzyć brakujące, na przykład: znamy bok trójkąta, kąt naprzeciw niego i promień okręgu opisanego, szukamy pozostałych wierzchołków,
    \item \emph{Περὶ Διαιρέσεων} (\emph{Peri diaipeseon} czyli \emph{O podziałach}), które znamy tylko częściowo z arabskiego tłumaczenia; zajmuje się podziałem danych figur na części w ustalonych proporcjach,
    \item \emph{Ὀπτικά} (\emph{Optika} czyli \emph{Optyka} \texttt{:D}) jest najstarszym greckim traktatem o perspektywie,
    \item \emph{Φαινόμενα} (\emph{Feinomena} czyli \emph{Zjawiska}) o astronomii sferycznej.
\end{itemize}

O czterech pracach podejrzewa się, że spisał je Euklides, ale ponieważ przepadną (i pojawią się później tylko w komentarzach innych uczonych), trudno jest coś więcej napisać:
\begin{itemize}
    \item \emph{Ψευδάρια} (\emph{Pseudaria} czyli \emph{Błędy rozumowania}) napisane dla początkujących o tym, jak unikać powszechnych błędów w rozumowaniach geometrycznych,
    \item \emph{Πορίσματα} (\emph{Porismata} czyli \emph{Poryzmy}) pomimo tytułu nie dotyczy wniosków z twierdzeń, ale raczej czegoś pomiędzy twierdzeniem i problemem, celem czego jest odkrycie własności już istniejącego obiektu, np. środku danego okręgu.
    Chasles podejrzewa, że dzieło to obejmuje teorię transwersali i geometrii rzutowej,
\index[persons]{Chasles, Michel}%
    \item \emph{Κωνικά} (\emph{Konika} czyli \emph{Stożkowe}) o stożkowych, traktat w czterech książkach, później uzupełniony i rozszerzony przez Apoloniusza,
    \item \emph{Τόποι πρὸς ἐπιφανείᾳ} (\emph{Topi pros epifania} czyli \emph{Obszary przy powierzchniach}) o którym prawie nic nie wiemy.
\end{itemize}

Powszechnie przyjmuje się, że szczytowy punkt rozwoju greckiej geometrii jest reprezentowany (oprócz Euklidesa, ma się rozumieć) przez Archimedesa oraz Apoloniusza.
Spotkamy tych panów wielokrotnie na następnych stronach książki: tego pierwszego jako autora dobrego przybliżenia liczby $\pi$;
drugiego jako osobę, która szukała czwartego okręgu stycznego do trzech danych i rozwinęła teorię krzywych stożkowych.

Wiele twierdzeń zawartych w Elementach Euklidesa to jawne przepisy, jak korzystając z cyrkla i~linijki otrzymać żądane figury.
Dlatego nigdzie w~tekście nie pojawia się na przykład siedmiokąt foremny, chociaż w trzecim tysiącleciu każdy akceptuje, że ten istnieje.
Ale nie można go wykreślić wspomnianymi przed chwilą przyrządami, więc Euklides nie zaczyna nawet badania jego własności.

\subsection{Księga I}	
Treść pierwszej księgi przytoczy Eves \cite[s. 375-379]{eves1_1972}.

\subsubsection{Definicje}	
\begin{enumerate}	
    \item [1.1] Punkt.
    \item [1.2] Odcinek.
    \item [1.3] Końce odcinka.
    \item [1.4] Prosta.
    \item [1.5] Powierzchnia.
    \item [1.6] Krawędzie powierzchni.
    \item [1.7] Płaszczyzna.
    \item [1.8] Kąt płaski.
    \item [1.9] Kąt prostoliniowy (?)
    \item [1.10] Kąt prosty, proste prostopadłe.
    \item [1.11] Kąt rozwarty.
    \item [1.12] Kąt ostry.
    \item [1.13] Kres albo granica (?).
    \item [1.14] Figura.
    \item [1.15] Koło i okrąg.
    \item [1.16] Środek koła.
    \item [1.17] Średnica koła.
    \item [1.18] Półokrąg, środek półokręgu.
    \item [1.19] Figura prostokreślna, trójkąt, czworobok (czworokąt), wielobok.
    \item [1.20] Trójkąt równoboczny, równoramienny i różnoboczny.
    \item [1.21] Trójkąt prostokątny, rozwartokątny, ostrokątny.
    \item [1.22] Kwadrat, prostokąt (nie będący kwadratem), romb (nie będący kwadratem), równoległobok (nie będący kwadratem, prostokątem ani rombem).
    \item [1.23] Proste równoległe.
\end{enumerate}	
	
\subsubsection{Postulaty}	
\begin{enumerate}	
    \item [1.1] Przez każde dwa punkty przechodzi prosta.
    \item [1.2] Każdy odcinek można przedłużyć do prostej.
    \item [1.3] Z każdego środka można wykreślić okrąg o dowolnym promieniu.
    \item [1.4] Wszystkie kąty proste są sobie równe.
    \item [1.5] Jeżeli prosta przecinająca dwie proste tworzy z nimi kąty jednostronnie wewnętrzne o sumie mniejszej niż $\pi$, to te dwie proste przecinają się po tej stronie.
\end{enumerate}	
	
Jak łatwo zauważyć, sformułowanie ostatniego postulatu używa więcej słów niż pozostałe razem wzięte; wbrew przekonaniu, że postulaty miały wyrażać treści oczywiste i proste.	
Piąty postulat wydawał się bardziej skomplikowany, więc nasuwał podejrzenie, że wynika z poprzednich czterech.	
Zauważył to już Proklos zwany Diadochem (410-485):
\index[persons]{Proklos zwany Diadochem}%
\emph{,,Nie jest możliwe, aby uczony tej miary co Euklides godził się na obecność tak długiego postulatu w aksjomatyce -- obecność postulatu wzięła się z pospiesznego kończenia przez niego Elementów, tak aby zdążyć przed nadejściem słusznie oczekiwanej rychłej śmierci; my zatem -- czcząc jego pamięć -- powinniśmy ten postulat usunąć lub co najmniej znacznie uprościć.''}	
	
Wiele osób próbowało stawić czoło wyzwaniu postawionemu przez Proklosa.	
Bezskutecznie, ponieważ piąty postulat jest niezależny od pozostałych!

\subsubsection{Pojęcia pierwotne}	
\begin{enumerate}	
    \item [1.1] Wyrażenia, które są równe się temu samemu wyrażeniowi, są sobie równe.
    \item [1.2] Równania można dodawać stronami.
    \item [1.3] Równania można odejmować stronami.
    \item [1.4] Wyrażenia, które się pokrywają, są sobie równe.
    \item [1.5] Całość jest większa od części.
\end{enumerate}	
	
\subsubsection{Twierdzenia}	
\begin{enumerate}	
    \item [1.1] Skonstruować trójkąt równoboczny o zadanym boku.
    \item [1.2] Skonstruować odcinek równy danemu, z końcem w zadanym punkcie.
    \item [1.3] Skonstruować różnicę dwóch odcinków.
    \item [1.4] Dane są punkty $A$, $B$, $C$, $D$, $E$, $F$ takie, że $AB = DE$, $BC = EF$, $\angle ABC = \angle DEF$.
    Wtedy trójkąty $\triangle ABC$ i $\triangle DEF$ są przystające.
    \item [1.5] Kąty przy podstawie trójkąta równoramiennego są równe.
    \item [1.6] Boki trójkąta leżące naprzeciw równych kątów są równe.
    \item [1.7] Nie można zbudować innego trójkąta o tych samych dwóch bokach przy tej samej podstawie po tej samej stronie prostej, na której leży podstawa.
    \item [1.8] Dane są punkty $A$, $B$, $C$, $D$, $E$, $F$ takie, że $AB = DE$, $BC = EF$ i $AC = DF$.
    Wtedy kąty $\angle ABC = \angle DEF$ są równe.
    \item [1.9] Podzielić dany kąt na dwie równe części.
    \item [1.10] Podzielić dany odcinek na dwie równe części.
    \item [1.11] Skonstruować prostopadłą do prostej przechodzącą przez dany punkt na tej prostej.
    \item [1.12] Skonstruować prostopadłą do prostej przechodzącą przez dany punkt poza tą prostą.
    \item [1.13] Suma miar kątów przyległych wynosi $\pi$.
    \item [1.14] Jeśli suma miar dwóch kątów o wspólnym ramieniu (i drugich ramionach leżących po różnych stronach) wynosi $\pi$, to są kątami przyległymi.
    \item [1.15] Kąty wierzchołkowe mają równe miary.
    \item [1.16] Kąt przyległy do dowolnego kąta wewnętrznego trójkąta jest większy od pozostałych dwóch kątów wewnętrznych.
    \item [1.17] W każdym trójkącie suma dwóch kątów jest mniejsza od $\pi$.
    \item [1.18] W każdym trójkącie większy bok leży naprzeciw większego kąta.
    \item [1.19] W każdym trójkącie większy kąt leży naprzeciw większego boku.
    \item [1.20] W każdym trójkącie suma dwóch boków jest większa od boku trzeciego.
    \item [1.21] Punkt $D$ leży wewnętrz trójkąta $ABC$. Wtedy $BD + CD < AB + AC$.
    \item [1.22] Z trzech odcinków można skonstruować trójkąt, kiedy suma dwóch krótszych jest dłuższa od trzeciego.
    \item [1.23] Odłożyć kąt na danej prostej, z wierzchołkiem w danym punkcie.
    \item [1.24] Dane są punkty $A$, $B$, $C$, $D$, $E$, $F$ takie, że $AB = DE$, $AC = DF$, $\angle BAC > \angle EDF$. Wtedy $BC > EF$.
    % TODO: https://en.wikipedia.org/wiki/Hinge_theorem
    \item [1.25] Dane są punkty $A$, $B$, $C$, $D$, $E$, $F$ takie, że $AB = DE$, $AC = DF$, $BC > EF$. Wtedy $\angle BAC > \angle EDF$. 
    \item [1.26] Dane są punkty $A$, $B$, $C$, $D$, $E$, $F$ takie, że $\angle ABC = \angle DEF$, $\angle BCA = \angle EFD$, $BC = EF$. Wtedy $AB = DE$, $AC = DF$ i $\angle BAC = \angle EDF$.
    \item [1.27] Jeśli prosta $EF$ przecina dwie inne $AB$ i $CD$ pod tym samym kątem: $\angle AEF = \angle EFD$, to są równoległe do siebie: $AB \parallel CD$.
    \item [1.27] Jeśli prosta $EF$ przecina dwie inne $AB$ i $CD$ w punktach $G$ i $H$ pod tym samym kątem: $\angle EGB = \angle GHD$, to są równoległe do siebie: $AB \parallel CD$.
    \item [1.29] Kąty naprzemianległe są równe.
    \item [1.30] Dwie proste równoległe do trzeciej prostej są też równoległe do siebie.
    \item [1.31] Skonstruować prostą równoległą do danej, przechodzącą przez punkt poza nią.
    \item [1.32] W trójkącie $ABC$ kąt zewnętrzny przy wierzchołku $C$ jest równy sumie kątów wewnętrznych przy wierzchołkach $A$ i $B$.
    \item [1.33] Dane są równe i równoległe odcinki $AB$ i $CD$, wtedy odcinki $AC$ i $BD$ też są równe i równoległe.
    \item [1.34] Przekątna równoległoboku dzieli go na dwa przystające trójkąty.
    \item [1.35] Dwa równoległoboki $ABCD$ i $EBCF$, których drugie podstawy leżą na tej samej prostej mają równe pola.
    \item [1.36] Dwa równoległoboki, które mają równe podstawy i równe wysokości mają równe pola. (uogólnienie poprzedniego)
    \item [1.37] Dwa trójkąty które mają tę samą podstawę i równe wysokości mają równe pola.
    \item [1.38] Dwa trójkąty które mają równe podstawy i równe wysokości mają równe pola.
    \item [1.39] Dane są dwa trójkąty $ABC$ i $ABD$ o równych polach takie, że punkty $C$ i $D$ leżą po tej samej stronie prostej $AB$. Wtedy $AB \parallel CD$.
    \item [1.40] (dziwne) % http://aleph0.clarku.edu/~djoyce/elements/bookI/propI40.html
    \item [1.41] Dany jest równoległobok i trójkąt o tej samej podstawie i równych wysokościach, wtedy pole równoległoboku  jest dwa razy większe od pola trójkąta.
    \item [1.42] Dany jest trójkąt oraz kąt, skonstruować równoległobok o polu takim samym jak trójkąt i danym kącie wewnętrznym.
    \item [1.43] W równoległoboku $ABCD$ wybrano punkt $P$ na przekątnej $AC$ i poprowadzono przez niego proste równoległe do boków, przecinające boki $AB$, $BC$, $CD$, $AD$ w punktach $H$, $F$, $G$, $I$. Wtedy równoległoboki $HBFP$ i $IPGD$ mają równe pola.\footnote{znane po angielsku jako \emph{theorem of the gnomon}, ponieważ Euklides wprowadzi termin gnomon w drugiej definicji drugiej księgi.}
    \item [1.44] Skonstruować równoległobok, którego jeden z boków jest równy danemu odcinkowi, jeden z kątów jest równy danemu kątowi, zaś pole jest równe polu danego trójkąta.
    \item [1.45]  Skonstruować równoległobok, którego jeden z boków jest równy danemu odcinkowi, jeden z kątów jest równy danemu kątowi, zaś pole jest równe polu danego czworokąta.
    \item [1.46] Skonstruować kwadrat.
    \item [1.47] W trójkącie prostokątnym, suma pól kwadratów zbudowanych na przyprostokątnych jest równa polu kwadratu zbudowanego na przeciwprostokątnej.
    \index{twierdzenie!Pitagorasa}
    \item [1.48] Jeżeli suma pól kwadratów zbudowanych na dwóch bokach trójkąta jest równa polu kwadratu zbudowanego na trzecim boku, to trójkąt jest prostokątny.
\end{enumerate}

%
\subsection{Księga II}	
\subsubsection{Definicje}	
\begin{enumerate}
    \item [2.1] Definicja ...
    % Definicja 1 % Każdy równoległobok prostokątny wyraża i wykreśla się dwiema liniami prostymi które zawierają właściwy kąt.
    \item [2.2] Definicja ...
    % Definicja 2 % W równoległoboku jeżeli poprowadzimy przekątną i przez punkt gdziekolwiek obrany na tej przekątnej poprowadzimy dwie linie równoległe do boków równoległoboku, równoległobok podzieli się na cztery części, każda z dwóch części której przekątna jest częścią przekątnej całego równoległoboku, wzięta z dwiema jej przyległymi zwać będziemy węgielnicą.
\end{enumerate}	
	
\subsubsection{Twierdzenia}	
\begin{enumerate}	
    \item [2.1] Twierdzenie ...
    % Twierdzenie 1 % Jeżeli z dwóch linii prostych podzielimy jedną którąkolwiek na ilekolwiek części (które będziemy nazywać odcinkami), prostokąt zawarty dwiema liniami prostymi, równy będzie prostokątom wykreślonym z linii prostej nieprzecietej i z odcinków drugiej linii prostej.
    \item [2.2] Twierdzenie ...
    % Twierdzenie 2 % Jeżeli linie prostą podzielimy jakkolwiek, prostokąty zawarte całą linią i jej oddzielnymi odcinkami będą równe kwadratowi z całej linii.
    \item [2.3] Twierdzenie ...
    % Twierdzenie 3 % Jeżeli linie prostą podzielimy na dwa jakiekolwiek odcinki; prostokąt całą linią i jednym odcinkiem, zawarty, będzie równy prostokątowi odcinkami linii prostej zawartymi wraz z kwadratem wyrażonym na odcinku wziętym z boku drugiego prostokąta pierwszego.
    \item [2.4] Twierdzenie ...
    % Twierdzenie 4 % Jeżeli linię prostą podzielimy na dwa jakiekolwiek odcinki, kwadrat z całej linii będzie równy kwadratom z obydwu odcinków linii dwa razy wziętemu prostokątowi zawartemu odcinkami linii.
    \item [2.5] Twierdzenie ...
    % Twierdzenie 5 % Jeżeli linię prostą podzielimy na dwa równe odcinki i na dwa odcinki nierówne; to prostokąt odcinkami nierównymi zawartymi wraz z kwadratem wystawionym na odcinkach między podziałami zawartymi będzie równy kwadratowi wystawionemu na połowie linii.
    \item [2.6] Twierdzenie ...
    % Twierdzenie 6 % Jeżeli linię prostą na dwa różne odcinki podzieloną przedłużymy podług upodobani; prostokąt zawarty linią prostą wraz z przedłużeniem wziętą i samym przedłużeniem, wraz z kwadratem wystawionym na połowie linii, będzie równy kwadratowi wystawionemu na połowie linii wraz z przedłużeniem wziętym.
    \item [2.7] Twierdzenie ...
    % Twierdzenie 7 % Jeżeli linię prostą podzielimy na dwa różne odcinki nierówne; kwadraty: pierwszy z całej linii, drugi z jej odcinka, będą równe dwa razy wziętemu prostokątowi całą linią i tym samym odcinkiem zawartym wraz z kwadratem z odcinka drugiego.
    \item [2.8] Twierdzenie ...
    % Twierdzenie 8 % Jeżeli linię podzielimy na dwa odcinki nierówne; cztery razy wzięty prostokąt całą linią i jej jednym odcinkiem zawarty wraz z kwadratem z odcinka drugiego, będzie równy kwadratowi wystawionemu na linii złożonej z całej linii i z odcinka pierwszego.
    \item [2.9] Twierdzenie ...
    % Twierdzenie 9 % Jeżeli linię prostą podzielimy na dwa odcinki równe, i na dwa odcinki nierówne; kwadraty z odcinków nierównych będą dwa razy większe od kwadratów, z których jeden byłby wystawiony na połowie linii, drugi na linii miedzy podziałami zawartej.
    \item [2.10] Twierdzenie ...
    % Twierdzenie 10 % Jeżeli linię prostą na dwa odcinki równe podzieloną przedłużymy według upodobania; kwadraty: pierwszy z całej linii wraz z przedłużeniem, drugi z samego przedłużenia, będą dwa razy większe od kwadratów, z których pierwszy byłby wystawiony na połowie linii, a drugi na połowie linii wraz z przedłużeniem wziętym.
    \item [2.11] Twierdzenie ...
    % Twierdzenie 11 % Daną linię prostą podzielić na dwa odcinki tak, aby prostokąt całą linią i jednym jej odcinkiem zawarty, był równy kwadratowi z odcinka drugiego.
    \item [2.12] Twierdzenie ...
    % Twierdzenie 12 % W trójkątach rozwartokątnych, kwadrat z boku kątowi przeciwnemu rozwartemu, większy jest od kwadratów z ramion kąta rozwartego o dwa razy wzięty prostokąt, zawarty ramionami kąta rozwartego i przedłużeniem tego ramienia zamkniętym między wierzchołkiem kąta rozwartego i punktem w którym prostopadła z końca drugiego ramienia kąta rozwartego spuszczona na pierwsze ramie, spotyka przedłużenie odcinka.
    \item [2.13] Twierdzenie ...
    % Twierdzenie 13 % W każdym trójkącie, kwadrat z boku przeciwnego kątowi ostremu, mniejszy jest od kwadratów z ramion ten kąt obejmujących, o dwa razy wzięty prostokąt zawarty ramieniem tego kąta ostrego i odcinka, lub przedłużeniem tego ramienia zamkniętym między wierzchołkami kata ostrego i punktem, w którym linia prostopadła z końca drugiego ramienia kąta ostrego spuszczona na pierwsze ramię spotyka to ramię lub przedłużenie danego ramienia.
    \item [2.14] Twierdzenie ...
    % Twierdzenie 14 % Na danej figurze prostokreślnej równy kwadrat wykreślić.
\end{enumerate}	
%

% http://aleph0.clarku.edu/~djoyce/java/elements/bookIII/bookIII.html

\subsection{Księga III}
\subsubsection{Definicje}
\begin{enumerate}
    \item [3.1] Dwa okręgi są \emph{równe}, kiedy mają równe średnice (lub równoważnie, promienie).
    \item [3.2] Prosta jest \emph{styczna} do okręgu, jeśli ma z nim punkt wspólny, ale po przedłużeniu nie ma drugiego takiego punktu.
    \item [3.3] Dwa okręgi są \emph{styczne}, kiedy mają punkt wspólny, ale nie przecinają się.
    \item [3.4] Dwie cięciwy (oryginalnie: proste) w okręgu są \emph{równoodległe} od jego środka, kiedy odcinki prostopadłe do nich opuszczone ze środka okręgu są równe.
    \item [3.5] Spośród dwóch cięciw ta jest \emph{bardziej oddalona}, do której odcinek prostopadły ze środka okręgu jest dłuższy.
    \item [3.6] \emph{Odcinek kołowy} to figura ograniczona przez cięciwę okręgu (a właściwie koła) oraz jego brzeg.
    \item [3.7] \emph{Kątem odcinka kołowego} jest kąt między cięciwą i brzegiem koła\footnote{To sformułowanie jest dziwne, ale jest potrzebne tylko raz, w twierdzeniu 3.16}.
    \item [3.8] Kątem w odcinku kołowym, co nie jest (!) tym samym, co wyżej, nazywamy kąt, którego wierzchołek leży na brzegu koła, zaś ramiona przechodzą przez cięciwę, która wyznacza odcinek kołowy.
    \item [3.9] Kąt w odcinku kołowym wygodniej jest nazywać kątem wpisanym opartym na łuku, który łączy końce odcinka kołowego (są dwa takie łuki).
    \item [3.10] Wycinek kołowy to figura ograniczona przez dwa promienie oraz łuk koła łączący te promienie.
    \item [3.11] Dwa odcinki kołowe są \emph{podobne}, kiedy mają te same kąty.
\end{enumerate}

\subsubsection{Twierdzenia}
\begin{enumerate}
    \item [3.1] Skonstruować środek danego okręgu.      
    \item [3.2] Cięciwa znajduje się wewnątrz koła.
    \item [3.3] Średnica (cięciwa przechodząca przez środek koła), która połowi inną cięciwę niebędącą średnicą, jest do niej prostopadła. Średnica, która jest prostopadła do innej cięciwy, połowi ją.
    \item [3.4] Dwie cięciwy, które nie są średnicami, nie połowią się nawzajem.
    \item [3.5] Dwa okręgi, które się przecinają, nie mogą być współśrodkowe. 
    \item [3.6] Dwa okręgi, które są styczne, nie są koncentryczne.
    \item [3.7] Dane jest koło oraz punkt $F$ leżący na jego średnicy $AD$, który nie jest środkiem. Rozpatrujemy odcinki łączące punkt $F$ z innymi punktami na brzegu koła. Wtedy:
        \begin{itemize}
        \item odcinek, który przechodzi przez środek jest najdłuższy;
        \item odcinek, który nie przechodzi przez środek, ale leży na średnicy, jest najkrótszy;
        \item odcinek, którego koniec leży dalej od wspomnianej średnicy jest dłuższy od odcinka, którego koniec leży bliżej.
        \end{itemize}
    (To twierdzenie nie jest później do niczego używane.)
    \item [3.8] Twierdzenie ...
    % Twierdzenie 8. % Jeżeli z punktu zewnątrz koła obranego, poprowadzone będą do okręgu linie proste, z których jedna przechodziła by przez środek koła a inne padały gdziekolwiek, z linii prostych padających na część okręgu wklęsłą, największa jest linia poprowadzona przez środek koła, z innych zaś linii każda bliższa przechodzącej przez środek jest większa od odleglejszej. Lecz z linii padających na cześć okręgu wypukłą, najmniejsza jest linia prosta zawarta między punktem zewnętrz koła i średnicą, z innych zaś linii prostych każda bliższa najmniejszej, mniejsza jest odleglejsza; na koniec dwie tylko równe linie proste z tego punktu po obydwu stronach najmniejszej linii prostej mogą być do okręgu poprowadzone.
    \item [3.9] Jeżeli przez punkt wewnątrz koła przechodzą trzy cięciwy równej długości, to jest środkiem koła.
    \item [3.10] Dwa okręgi, które się przecinają, przecinają się w dwóch punktach. 
    \item [3.11] Prosta łącząca środki kół stycznych wewnętrznie przechodzi przez punkt styczności kół.
    \item [3.12] Odcinek łączący środki kół stycznych zewnętrznie przechodzi przez punkt styczności kół.
    \item [! 3.13] Dwa różne okręgi nie mogą być styczne (wewnętrznie lub zewnętrznie) w więcej niż jednym punkcie.
    \item [3.14] Następujące warunki są równoważne:
        \begin{itemize}
        \item dwie cięciwy są równej długości; 
        \item dwie cięciwy są równoodległe od środka koła.
        \end{itemize}
    \item [! 3.15] Spośród dwóch cięciw ta jest dłuższa, która leży bliżej środka koła; średnica to najdłuższa cięciwa.
    \item [3.16] Prosta prostopadła do średnicy przechodząca przez jej środek nie przechodzi przez wnętrze koła. (To twierdzenie mówi też, że ,,kąt'' między prostopadłą a brzegiem koła jest mniejszy niż dowolny kąt płaski, co trudno sformalizować i nie jest później wykorzystywane).
    \item [3.17] Skonstruować styczną do danego okręgu, która przechodzi przez dany punkt.
    \item [3.18] Odcinek łączący środek koła z punktem styczności tego koła i pewnej prostej jest prostopadły do tej prostej.
    \item [3.19] Prosta prostopadła do stycznej do okręgu, przechodząca przez punkt styczności, przechodzi także przez środek koła.
    \item [3.20] Twierdzenie ...
    % Twierdzenie 20. % W kole, kąt mający wierzchołek we środku jest podwojeniem kata mającego swój wierzchołek na okręgu koła, gdyż tę samą podstawę okręgu mają za podstawę, czyli to samo gdy ramionami swymi tej samej części okręgu obejmują.
    \item [3.21] Twierdzenie ...
    % Twierdzenie 21. % Kąty w tym samym odcinku koła są między sobą równe.
    \item [3.22] Twierdzenie ...
    % Twierdzenie 22. % Kąty przeciwne czworokąta w koło wpisane są równe dwóm kątom prostym.
    \item [3.23] Twierdzenie ...
    % Twierdzenie 23. % Na tej samej linii prostej nie można wykreślić dwóch odcinków kół po tej samej stronie podobnych, które by nie przystawały do siebie.
    \item [3.24] Twierdzenie ...
    % Twierdzenie 24. % Wykreślone na równych liniach prostych podobne odcinki kół, są między sobą równe.
    \item [3.25] Twierdzenie ...
    % Twierdzenie 25. % Mając dany odcinek koła, opisać koła którego jest odcinkiem.
    \item [3.26] Twierdzenie ...
    % Twierdzenie 26. % W kołach równych, kąty równe w środkach lub przy okręgach wspierają się na równych łukach.
    \item [3.27] Twierdzenie ...
    % Twierdzenie 27. % W kołach równych, kąty we środkach lub przy okręgach, na równych łukach wspierające się, są między sobą równe.
    \item [3.28] Twierdzenie ...
    % Twierdzenie 28. % W kołach równych, cięciwy równe obejmują łuki równe, tak, że łuk większy większemu, mniejszy mniejszemu jest równy.
    \item [3.29] Twierdzenie ...
    % Twierdzenie 29. % W kołach równych, równe łuki obejmują cięciwy równe.
    \item [3.30] Podzielić dany Twierdzenie ...
    % Twierdzenie 30. % Dany łuk podzielić na dwie części.
    \item [3.31] Twierdzenie ...
    % Twierdzenie 31. % W kole, kąt w półkolu jest prosty; z katów zaś w odcinkach nierównych, kąt w większym odcinku mniejszy jest od prostego; a w mniejszym odcinku większy od prostego.
    \item [3.32] Twierdzenie ...
    % Twierdzenie 32. % Jeżeli okręgu koła dotyka linia prosta, z punktu zaś dotknięcia poprowadzona będzie cięciwa, kąty zawarte miedzy cięciwową i styczną, będą równe kątom w odcinkach koła na przemian.
    \item [3.33] Twierdzenie ...
    % Twierdzenie 33. % Na danej linii prostej wykreślić odcinek koła który by zawierał kąt równy kątowi danemu.
    \item [3.34] Twierdzenie ...
    % Twierdzenie 34. % Z koła danego oddzielić odcinek któryby zawierał kąt równy danemu kątowi.
    \item [3.35] Twierdzenie ...
    % Twierdzenie 35. % Jeżeli w kole dwie cięciwy przecinają się nawzajem, prostokąt zawarty odcinkami jednej cięciwy będzie równy prostokątowi zawartemu odcinkami drugiej cięciwy.
    \item [3.36] Twierdzenie ...
    % Twierdzenie 36. % Jeżeli z punktu za kołem obranego, poprowadzimy dwie linie proste, których jedna przecinałaby koło, a druga byłaby styczną; to prostokąt zawarty całą linia przecinającą i odcinkiem jej za kołem będzie równy kwadratowi ze stycznej.
    \item [3.37] Twierdzenie ...
    % Twierdzenie 37. % Jeżeli z dwóch linii prostych, od jednego punktu zewnątrz koła obranego poprowadzonych, jedna przecina koło, a druga pada na okrąg tego koła: i jeżeli prostokąt z całej linii przecinającej i odcinka jej za kołem będącego jest równy kwadratowi z linii padającej na okrąg koła, to linia będzie padająca na okrąg koła styczną.
\end{enumerate}

%
%

\subsection{Księga IV}
\subsubsection{Definicje}
\begin{enumerate}
	\item [4.1] Dwie figury prostokreślne są wpisane jedna w drugą, jeśli wierzchołki pierwszej leżą na bokach drugiej, a boki drugiej przechodzą przez wierzchołki pierwszej.
	\item [4.2] Podobnie definiuje się figurę, która jest opisana na innej figurze.
	\item [4.3] Figura prostokreślna jest wpisana w okrąg, kiedy wszystkie jej wierzchołki leżą na tym okręgu.
	\item [4.4] Figura prostokreślna jest opisana na okręgu, kiedy każdy jej bok jest styczny do tego okręgu.
	\item [4.5] Podobnie definiuje się, że okrąg jest wpisany w figurę prostokreślną.
	\item [4.6] Koło opisuje się na figurze prostokreślnej, kiedy okrąg (jego brzeg) przechodzi przez wszystkie wierzchołki figury.
	\item [4.7] Linia prosta (właściwie: odcinek) kreśli się w kole, gdy jej końca leżą na brzegu tego koła.
\end{enumerate}

\subsubsection{Twierdzenia}
\begin{enumerate}
	\item [4.1] Wpisać odcinek krótszy od średnicy w dany okrąg.
	\item [4.2] Wpisać trójkąt podobny do danego w okrąg. % equiangular
	\item [4.3] Opisać trójkąt podobny do danego na okręgu.
	\item [4.4] Wpisać okrąg w dany trójkąt.
	\item [4.5] Opisać okrąg na danym kwadracie. \index{okrąg!opisany} % TODO OKRG NA OKREGU???
	\item [4.6] Wpisać kwadrat w dany okrąg.
	\item [4.7] Opisać kwadrat na danym okręgu.
	\item [4.8] Wpisać okrąg w dany kwadrat.
	\item [4.9] Opisać okrąg na danym kwadracie.
	\item [4.10] Wykreślić trójkąt równoramienny, którego kąt przy podstawie jest podwojeniem kąta przy wierzchołku (o kątach $\pi/5$, $2\pi/5$, $2\pi/5$; taki trójkąt nazywa się czasami złotym). \index{trójkąt!złoty}
	\item [4.11] Wpisać pięciokąt foremny\footnote{Euklides mówi tu o pięciokącie równobocznym i jednocześnie równokątnym.} w dany okrąg.
	\item [4.12] Opisać pięciokąt foremny na danym okręgu. \index{pięciokąt|see{wielokąt}}
	\index{wielokąt!pięciokąt}
	\item [4.13] Wpisać okrąg w dany pięciokąt równoboczny i równokątny.
	\item [4.14] Opisać okrąg na danym pięciokącie równobocznym i równokątnym
	\item [4.15] Wpisać sześciokąt równoboczny i równokątny w dany okrąg.
	\index{sześciokąt|see{wielokąt}}
	\index{wielokąt!sześciokąt}
	\item [4.16] Wpisać piętnastokąt równoboczny i równokątny w dany okrąg.
	\index{piętnastokąt|see{wielokąt}}
	\index{wielokąt!piętnastokąt}
\end{enumerate}

%

\index{aksjomaty!Euklidesa|)}%

%