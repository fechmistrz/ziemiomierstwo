\subsection{Księga II}	
\subsubsection{Definicje}	
\begin{enumerate}
    \item [2.1] Definicja ...
    % Definicja 1 % Każdy równoległobok prostokątny wyraża i wykreśla się dwiema liniami prostymi które zawierają właściwy kąt.
    \item [2.2] Definicja ...
    % Definicja 2 % W równoległoboku jeżeli poprowadzimy przekątną i przez punkt gdziekolwiek obrany na tej przekątnej poprowadzimy dwie linie równoległe do boków równoległoboku, równoległobok podzieli się na cztery części, każda z dwóch części której przekątna jest częścią przekątnej całego równoległoboku, wzięta z dwiema jej przyległymi zwać będziemy węgielnicą.
\end{enumerate}	
	
\subsubsection{Twierdzenia}	
\begin{enumerate}	
    \item [2.1] Twierdzenie ...
    % Twierdzenie 1 % Jeżeli z dwóch linii prostych podzielimy jedną którąkolwiek na ilekolwiek części (które będziemy nazywać odcinkami), prostokąt zawarty dwiema liniami prostymi, równy będzie prostokątom wykreślonym z linii prostej nieprzecietej i z odcinków drugiej linii prostej.
    \item [2.2] Twierdzenie ...
    % Twierdzenie 2 % Jeżeli linie prostą podzielimy jakkolwiek, prostokąty zawarte całą linią i jej oddzielnymi odcinkami będą równe kwadratowi z całej linii.
    \item [2.3] Twierdzenie ...
    % Twierdzenie 3 % Jeżeli linie prostą podzielimy na dwa jakiekolwiek odcinki; prostokąt całą linią i jednym odcinkiem, zawarty, będzie równy prostokątowi odcinkami linii prostej zawartymi wraz z kwadratem wyrażonym na odcinku wziętym z boku drugiego prostokąta pierwszego.
    \item [2.4] Twierdzenie ...
    % Twierdzenie 4 % Jeżeli linię prostą podzielimy na dwa jakiekolwiek odcinki, kwadrat z całej linii będzie równy kwadratom z obydwu odcinków linii dwa razy wziętemu prostokątowi zawartemu odcinkami linii.
    \item [2.5] Twierdzenie ...
    % Twierdzenie 5 % Jeżeli linię prostą podzielimy na dwa równe odcinki i na dwa odcinki nierówne; to prostokąt odcinkami nierównymi zawartymi wraz z kwadratem wystawionym na odcinkach między podziałami zawartymi będzie równy kwadratowi wystawionemu na połowie linii.
    \item [2.6] Twierdzenie ...
    % Twierdzenie 6 % Jeżeli linię prostą na dwa różne odcinki podzieloną przedłużymy podług upodobani; prostokąt zawarty linią prostą wraz z przedłużeniem wziętą i samym przedłużeniem, wraz z kwadratem wystawionym na połowie linii, będzie równy kwadratowi wystawionemu na połowie linii wraz z przedłużeniem wziętym.
    \item [2.7] Twierdzenie ...
    % Twierdzenie 7 % Jeżeli linię prostą podzielimy na dwa różne odcinki nierówne; kwadraty: pierwszy z całej linii, drugi z jej odcinka, będą równe dwa razy wziętemu prostokątowi całą linią i tym samym odcinkiem zawartym wraz z kwadratem z odcinka drugiego.
    \item [2.8] Twierdzenie ...
    % Twierdzenie 8 % Jeżeli linię podzielimy na dwa odcinki nierówne; cztery razy wzięty prostokąt całą linią i jej jednym odcinkiem zawarty wraz z kwadratem z odcinka drugiego, będzie równy kwadratowi wystawionemu na linii złożonej z całej linii i z odcinka pierwszego.
    \item [2.9] Twierdzenie ...
    % Twierdzenie 9 % Jeżeli linię prostą podzielimy na dwa odcinki równe, i na dwa odcinki nierówne; kwadraty z odcinków nierównych będą dwa razy większe od kwadratów, z których jeden byłby wystawiony na połowie linii, drugi na linii miedzy podziałami zawartej.
    \item [2.10] Twierdzenie ...
    % Twierdzenie 10 % Jeżeli linię prostą na dwa odcinki równe podzieloną przedłużymy według upodobania; kwadraty: pierwszy z całej linii wraz z przedłużeniem, drugi z samego przedłużenia, będą dwa razy większe od kwadratów, z których pierwszy byłby wystawiony na połowie linii, a drugi na połowie linii wraz z przedłużeniem wziętym.
    \item [2.11] Twierdzenie ...
    % Twierdzenie 11 % Daną linię prostą podzielić na dwa odcinki tak, aby prostokąt całą linią i jednym jej odcinkiem zawarty, był równy kwadratowi z odcinka drugiego.
    \item [2.12] Twierdzenie ...
    % Twierdzenie 12 % W trójkątach rozwartokątnych, kwadrat z boku kątowi przeciwnemu rozwartemu, większy jest od kwadratów z ramion kąta rozwartego o dwa razy wzięty prostokąt, zawarty ramionami kąta rozwartego i przedłużeniem tego ramienia zamkniętym między wierzchołkiem kąta rozwartego i punktem w którym prostopadła z końca drugiego ramienia kąta rozwartego spuszczona na pierwsze ramie, spotyka przedłużenie odcinka.
    \item [2.13] Twierdzenie ...
    % Twierdzenie 13 % W każdym trójkącie, kwadrat z boku przeciwnego kątowi ostremu, mniejszy jest od kwadratów z ramion ten kąt obejmujących, o dwa razy wzięty prostokąt zawarty ramieniem tego kąta ostrego i odcinka, lub przedłużeniem tego ramienia zamkniętym między wierzchołkami kata ostrego i punktem, w którym linia prostopadła z końca drugiego ramienia kąta ostrego spuszczona na pierwsze ramię spotyka to ramię lub przedłużenie danego ramienia.
    \item [2.14] Twierdzenie ...
    % Twierdzenie 14 % Na danej figurze prostokreślnej równy kwadrat wykreślić.
\end{enumerate}	