%

\section{Geometrie neutralna oraz uporządkowana}
Geometrię opartą o ten sam zestaw aksjomatów, co nasz (czyli aksjomaty Hilberta), ale bez piątego postulatu, nazywamy geometrią neutralną; prawdziwe są w niej na przykład (I.1) do (I.28) oraz (I.31).
Jej twierdzenia obowiązują w geometrii euklidesowej oraz hiperbolicznej (gdzie przez każdy punkt poza daną prostą można przeprowadzić co najmniej dwie proste równległe!), ale nie eliptycznej lub sferycznej, ponieważ uporządkowanie punktów na prostej wszystko psuje.
Geometria absolutna jest przedłużeniem geometrii uporządkowanej, która stanowi trzon wcześniej wymienionej oraz geometrii afinicznej, czyli takiej, gdzie nie ma odległości ani kątów.
No i jest jeszcze geometria rzutowa...
Napiszą coś o tym lub części tego Coxeter \cite[s. 193, 194]{coxeter_1967}, Audin \cite[s. 13]{audin_2003}.

https://www.deltami.edu.pl/2019/04/co-to-jest-geometria-rzutowa/

\index{twierdzenie!Sylvestera-Gallaia|(}%

\subsection{Twierdzenie Sylvestera-Gallaia}

Podamy teraz (jedyny nam znany) przykład interesującego twierdzenia geometrii neutralnej.

\begin{theorem}[Sylvestera-Gallaia]
	Dla każdego skończonego zbioru punktów na płaszczyźnie istnieje prosta, która przechodzi przez dokładnie dwa albo wszystkie punkty.
\end{theorem}

W 1893 roku James Sylvester postawi ten problem, zainspirowany konfiguracją Hessego\footnote{Konfiguracja Hessego to 12 prostych przez 9 punktów na zespolonej płaszczyźnie rzutowej, gdzie każdy punkt leży na 4 prostych, a każda prosta przechodzi przez 3 punkty.} (albo czymś innym, kto go tam wie).
\index[persons]{Sylvester, James}%
\index{kofinguracja Hessego}%
Herbert Woodall szybko zaproponuje rozwiązanie z usterką, która zostanie równie szybko dostrzeżona.
\index[persons]{Woodall, Herbert}%
Dopiero w 1941 roku Eberhard Melchior udowodni trochę mocniejsze stwierdzenie niż rzutowy dual ówczesnej hipotezy (że prostych przez dokładnie dwa punkty jest co najmniej trzy).
\index[persons]{Melchior, Eberhard}%
Nieświadomy tego, Paul Erdős postawi hipotezę na nowo w~1943 roku, a Tibor Gallai w 1944 roku doda swój dowód (ponownie wykorzystując elementy geometrii rzutowej).
\index[persons]{Erdős, Paul}%
\index[persons]{Gallai, Tibor}%
Wraz z upływem czasu pojawią się inne, ciekawe rozumowania.
Na przykład Leroy Kelly wykorzysta własności metryki, co oburzy Harolda Coxetera \cite[s. 199-201]{coxeter_1967} i skłoni go do opublikowania kolejnego dowodu, korzystającego jedynie z aksjomatów geometrii uporządkowania.
\index{geometria!uporządkowania}
\index[persons]{Kelly, Leroy}%
\index[persons]{Coxeter, Harold}%
(Aigner, Ziegler uznają dowód Kelly'ego za najlepszy).

Niech $t_2(n)$ oznacza minimalną liczbę prostych przez dwa punkty w dowolnym ułożeniu $n$ punktów.
Melchior pokaże, że $t_2(n) \ge 3$.
Wynik sukcesywnie poprawią:
de Bruijn \cite{debruijn_1948} zapyta, czy $t_2(n)$ dąży do nieskończoności,
\index[persons]{de Bruijn, Nicolas}%
Theodore Motzkin \cite{motzkin_1951} udzieli twierdzącej odpowiedz, bo $t_2(n) \ge \sqrt{n}$.
\index[persons]{Motzkin, Theodore}%
Potem Gabriel Dirac \cite{dirac_1951} przypuści, że $t_2(n) \ge \lfloor n/2\rfloor$, co nie zostawia wiele miejsca na poprawki, bo dla parzystych $n \ge 6$ zachodzi $t_2(n) \le n/2$, jak pokaże pomysłową konstrukcją Károly Böröczky.
\index[persons]{Dirac, Gabriel}%
\index[persons]{Böröczky, Károly}%
Dla nieparzystych $n$ wiemy tylko, że ten kres jest realizowany dla $n = 7$ (Kelly, Moser \cite{kelly_1958} w 1958) i $n = 13$ (Crowe, McKee \cite{mckee_1968} w 1968).
\index[persons]{Moser, William}%
\index[persons]{Crowe, Donald}%
\index[persons]{McKee, Terry}%
Najpóźniejszy wynik, o jakim nam wiadomo, to Csimy, Sawyera \cite{csima_1993}: że $t_2(n) \ge \lceil 6n/13 \rceil$.
\index[persons]{Sawyer, Eric}%
\index[persons]{Csima, Joseph}%

\index{twierdzenie!Sylvestera-Gallaia|)}%

%