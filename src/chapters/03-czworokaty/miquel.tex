%

\begin{proposition}[o punkcie Miquela]
	\index{twierdzenie!Miquela}%
	Na bokach $AB$, $BC$, $AC$ trójkąta $\triangle ABC$ wybrano kolejno punkty $C'$, $A'$, $B'$.
	Okręgi opisane na trójkątach $AB'C'$, $A'BC'$ oraz $A'B'C$ mają punkt wspólny, zwany punktem Miquela.
\end{proposition}

Ostermann, Wanner (autorzy książki ,,Geometry by its History'') stwierdzą, że Auguste Miquel był nauczycielem szkoły średniej na francuskiej prowincji (Nantua).
% Ostermann, Alexander; Wanner, Gerhard (2012), Geometry by its History, Springer, ISBN 978-3-642-29162-3
% STRONA 94 TAMŻĘ
Podobną informację znajdziemy w ćwiczeniu u Hartshorne'a \cite[s. 61]{hartshorne2000} razem z datą: rokiem 1838.

\begin{proposition}
	Jeśli oznaczyć punkt Miquela przez $M$, to kąty $\angle MA'B$, $\angle MB'C$ oraz $\angle MC'A$ mają równe miary.	
\end{proposition}

Niektórzy używają nazwy ,,twierdzenie Miquela'' wobec innych wyników:
\begin{itemize}
	\item (o czterech okręgach) że okręgi opisane na czterech trójkątach wyznaczonych przez czworobok zupełny przechodzą przez jeden punkt; ogłoszone krótko przez Jakoba Steinera w wydaniu 1827/1828 \emph{,,Gergonne's Annales de Mathématiques''}, dowiedzione przez Miquela,
	\item (o pięciu okręgach) że okręgi opisane na trójkątach wyznaczonych przez przedłużenia boków pięciokąta wypukłego przecinają się w pięciu nowych punktach; leżą one na jednym okręgu, % converse: https://en.wikipedia.org/wiki/Five_circles_theorem
	\item (o sześciu okręgach) że pięć okręgów, które mają cztery potrójne punkty przecięcia, wyznaczają szósty okrąg, na którym leżą pozostałe cztery punkty przecięć.
\end{itemize}

Na przykład dla Audina \cite[s. 104]{audin_2003} ,,twierdzenie Miquela'' to trzeci wynik z listy, zaś ,,the pivot'' to określenie, jakiego używa wobec naszego twierdzenie Miquela.
Patrz też do Guzickiego \cite[s. 29, 32]{guzicki_2021}.
\index{punkt!Miquela}%

%