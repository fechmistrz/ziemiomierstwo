%

\index{czworobok zupełny|(}%
\index{czworokąt zupełny|see{czworobok zupełny}}%

Pojęcie opisane tutaj jest charakterystyczne dla geometrii incydencji albo rzutowej.

\begin{definition}
	Cztery punkty, z których żadne trzy nie są współliniowe, leżące na sześciu prostych przez sześć par punktów, nazywamy czworokątem zupełnym.
\end{definition}

Sześć prostych wyznacza dodatkowe trzy punkty, które nazywamy przekątniowymi.

\begin{definition}
	Cztery proste, z których żadne trzy nie są współpękowe, przecinające się parami w sześciu różnych punktach, nazywamy czworobokiem zupełnym.
\end{definition}

Robert Lachlan nazwie w 1893 roku czworoboki zupełne tetragramami, zaś czworokąty tetrastygmami.
\index[persons]{Lachlan, Robert}
Podobnież zdarzy się później napotkać jeszcze te słowa.
\index{tetragram}%
\index{tetrastygma}%
\index[persons]{Lachlan, Robert}%
% Lachlan, Robert (1893). An Elementary Treatise on Modern Pure Geometry. London, New York: Macmillan and Co.
Inne określenie na czworoboki zupełne to konfiguracje Pascha, szczególnie w kontekście potrójnych układów Steinera.
\index{układ Steinera, potrójny}%
\index{konfiguracja Pascha}%
Można je spotkać u Coxetera \cite[s. 34]{coxeter_1967}.
Eves \cite[s. 84]{eves1_1972} wprowadzi je przy okazji podziału harmonicznego.

\begin{theorem}[Newtona-Gaussa]
	Środki trzech przekątnych czworoboku zupełnego leżą na jednej prostej, zwaną prostą Newtona-Gaussa.
	\index{twierdzenie!Newtona-Gaussa}%
	\index{prosta!Newtona-Gaussa}%
\end{theorem}

\todofoot{https://www.deltami.edu.pl/2007/02/twierdzenie-newtona/}

\todofoot{https://www.deltami.edu.pl/media/issues/2017/11/delta-2017-11.pdf}

\todofoot{https://www.deltami.edu.pl/2021/09/zabawy-na-polu/, strona 9}

Różne dowody korzystają z własności pola, iloczynu zewnętrznego albo twierdzenie Menelaosa.
\index{iloczyn zewnętrzny}%
\index{twierdzenie!Menelaosa}%
\todofoot{en-wiki Newton-Gauss line}

W książkach geometrii afinicznej dowodzi się, że środki boków czworokąta zupełnego i trzy punkty przekątniowe leżą na stożkowej.

% TODO: https://en.wikipedia.org/wiki/Newton-Gauss_line#:~:text=In%20geometry%2C%20the%20Newton-Gauss,diagonals%20of%20a%20complete%20quadrilateral.
% 	\todofoot{Twierdzenie Newtona: środek okręgu ego w czworokąt i środki przekątnych tego czworokąta są współliniowe.}

% https://en.wikipedia.org/wiki/Newton%E2%80%93Gauss_line#Existence_of_the_Newton%E2%88%92Gauss_line

\begin{theorem}[Gaussa-Bodenmillera]
	Trzy okręgi, których średnicami są przekątne czworoboku zupełnego, są współosiowie.
	% The theorem of Gauss and Bodenmiller states that the three circles whose diameters are the diagonals of a complete quadrilateral are coaxal.[8]
	\index{twierdzenie!Gaussa-Bodenmillera}
\end{theorem}

\begin{theorem}[Jemieljanowa?]
	punkt Miquela właściwego czworoboku zupełnego leży na okręgu dziewięciu punktów trójkąta przekątnego tego czworoboku.?
	\index{twierdzenie!Jemieljanowa}
\end{theorem}

\begin{proposition}
	Neugebauer 262: w każdy właściwy czworobok zupełny da się wpisać dokładnie jedną parabolę, jej ogniskiem jest punkt Miquela czworoboku.
	\todofoot{Neugebauer 262: w każdy właściwy czworobok zupełny da się wpisać dokładnie jedną parabolę, jej ogniskiem jest punkt Miquela czworoboku.}
\end{proposition}

\index{czworobok zupełny|)}%

%