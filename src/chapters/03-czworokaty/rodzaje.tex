%

\begin{definition}[czworokąt]
\index{czworokąt}%
	Dane są cztery punkty $A$, $B$, $C$, $D$, z których żadne trzy nie są współliniowe, takimi że odcinki $AB$, $BC$, $CD$, $DA$ nie mają części wspólnej poza końcami.
    Wtedy sumę tych odcinków nazywamy czworokątem.	
\end{definition}

% https://en.wikipedia.org/wiki/Square
% https://en.wikipedia.org/wiki/Rhomboid

Po angielsku \emph{quadrangle}, z łacińskiego \emph{quadri} określającego liczbę cztery oraz \emph{latus}, czyli boku (a~więc technicznie czworobok, nie czworokąt); istnieje też rzadko używane słowo \emph{tetragon} (przez analogię do \emph{polygon, pentagon}) z greckiego \emph{tetra} oraz \emph{gon}, znaczącego tym razem kąt.
Niektórzy dopuszczają czworokąty z samoprzecięciem, nazywając je muchą lub motylem.
\index{motyl}%
\index{mucha}%
Hartshorne \cite[s.~80]{hartshorne2000} nie akceptuje motyli.

Czworokąty, które nie mają samoprzecięć, mogą być wypukłe albo wklęsłe.
Wypukłe tworzą bogatszą rodzinę.
W krajach anglosaskich wyróżnia się czworokąty nieregularne, które nie mają żadnej pary boków równoległych (\emph{trapezium} w amerykańskiej odmianie języka).

\begin{definition}[trapez]
\index{trapez}%
	Czworokąt, który posiada jedną lub dwie pary boków równoległych nazywamy trapezem.
\end{definition}

Po brytyjsku \emph{trapezium}, po amerykańsku \emph{trapezoid} -- widać więc kolizję z terminem na czworokąt ,,w ogólnym położeniu''.
Wyjaśnimy to po podaniu pozostałych definicji. % https://en.wikipedia.org/wiki/Trapezoid

\begin{definition}[trapez równoramienny]
\index{trapez!równoramienny}%
	Czworokąt, który posiada parę boków równoległych, którego kąty przy podstawie są równej miary, nazywamy trapezem równoramiennym.
\end{definition}

W szczególności, nie można definiować trapezu równoramiennego jako trapezu, który ma ramiona równej długości!
To czyniłoby wszystkie równoległoboki trapezami równoramiennymi, co nie ma żadnych zalet.
\index{równoległobok}%
Inne, równoważne definicje mówią, że jest to czworokąt z osią symetrii, która połowi parę przeciwległych boków albo trapez, który ma przekątne równej długości.

\begin{definition}[równoległobok]
\index{równoległobok}%
	Czworokąt, który posiada dwie pary boków równoległych, nazywamy równoległobokiem.
\end{definition}

Każdy równoległobok (po angielsku \emph{parallelogram}) jest też trapezem.
Alternatywne definicje głoszą, że ma przeciwległe boki równej długości, albo równe przeciwległe kąty, albo połowiące się wzajemnie przekątne.

\begin{definition}[romb]
\index{romb}%
	Czworokąt, który posiada cztery boki tej samej długości, nazywamy rombem.
\end{definition}

Równoważnie, romby to te czworokąty, których prostopadłe przekątne połowią się i rozcinają romb na cztery przystające trójkąty (prostokątne).
Angielski nazywa je \emph{rhombus}.

\begin{definition}[prostokąt]
\index{prostokąt}%
	Czworokąt, który posiada cztery kąty tej samej miary, nazywamy prostokątem.
\end{definition}

Nazwa bierze się z tego, że wszystkie jego kąty są proste.
Alternatywnie, prostokąty to dokładnie te czworokąty, których przekątne równej długości połowią się.
Szczególnymi prostokątami są

\begin{definition}[kwadrat]
\index{kwadrat}%
	Czworokąt, który posiada cztery kąty tej samej miary i wszystkie boki tej samej długości, nazywamy kwadratem.
\end{definition}

Kwadraty (\emph{squares}) to dokładnie te prostokąty (\emph{rectangles}), które są też rombami.

\begin{definition}[latawiec]
\index{latawiec}%
\index{deltoid|see{latawiec}}%
	Czworokąt, który ma dwie pary sąsiednich boków równej długości, nazywamy latawcem.	
\end{definition}

W Polsce bardziej popularne będzie określenie deltoid.
Latawce mają jedną przekątną, która dzieli je na dwa przystające trójkąty, w szczególności jego przekątne są prostopadłe.
Na latawcu, który ma dwa kąty proste naprzeciw siebie można opisać okrąg.
Romby są szczególnym przypadkiem latawców.

Czworokąty wklęsłe mają jeden kąt, którego miara wynosi między $\pi$ i $2\pi$ radianów.
Wyróżniamy wśród nich rzutki (z angielskiego \emph{dart, arrowhead}), które mają oś symetrii albo równoważnie, dwie pary sąsiednich boków równej długości, tak jak latawce.
\index{rzutka}%

W starożytnej Grecji czworokąty podzieli się na rozłączne klasy: kwadraty, podłużne prostokąty (po angielsku \emph{oblong}), niekwadratowe romby, romboidy (równoległoboki, które nie są ani rombami, ani prostokątami) oraz \emph{trapezia} (τραπέζιον znaczy dosłownie stolik), które nie pasowały do wcześniej wymienionych.
Filozof neoplatoński Proklos zwany Diadochem napisze wpływowy komentarz do prac Euklidesa, gdzie wprowadzi trochę inny podział zgodnie z tym, co obmyśli około 100 lat p.n.e. Posejdonios z Rodos (nauczyciel Cycerona).
% Proklos: jeden z epigonów matematyki greckiej, 
Czworokąt będzie równoległobokiem (kwadratem, podłużnym prostokątem, rombem, romboidem) albo nie (z parą boków równoległych, jako trapezium równoramienne lub ukośne albo bez niej -- trapezoidem).
Wszystkie języki europejskie odziedziczą klasyfikację czworokątów po Proklosie... poza angielskim, gdzie znaczenie trapezoidu i trapezium odwróci słownik wydany przez Charlesa Huttona w 1795 roku.
Porządek przywróci się około 1875 roku, ale amerykańskiego angielskiego już nikt nie uratuje.