%%% POLE
Eves \cite[s.4]{eves_1963} napisze, że w starożytnym Babilonie będzie używany niepoprawny wzór na pole czworokąta o przeciwległych bokach $a, c$ oraz $b, d$:
\begin{equation}
	S = \frac{a + c}{2} \cdot \frac{b + d}{2}.
\end{equation}
Wzór ten zostanie umieszczony w grobowcu Ptolemeusza XI, króla Egiptu żyjącego krótko przed narodzinami Chrystusa.
Jak łatwo zauważyć, pole otrzymane w ten sposób jest zawyżone, chyba że czworokąt był prostokątem: jeśli oznaczymy kolejne kąty przez $\alpha, \beta, \gamma, \delta$, to 
\begin{align}
	S & = \frac 1 2 \left(\frac 1 2 ab \sin \alpha + \frac 1 2 bc \sin \beta + \frac 1 2 cd \sin \gamma + \frac 1 2 ad \sin \delta\right) \\
	& \le \frac 1 4 (ab + bc + cd + ad) = \frac{a + c}{2} \cdot \frac{b + d}{2}.
\end{align}
z równością, kiedy wszystkie cztery kąty są proste.
%%% POLE
% https://www.ime.usp.br/~toscano/disc/2022/GreenbergGeometry.pdf ps. 34