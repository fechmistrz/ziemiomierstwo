
Na rombie, który nie jest kwadratem, nie można opisać okręgu.

\begin{proposition}[okrąg opisany na czworokącie]
\index{okrąg!opisany}%
\label{prp_incircle}
	Niech $A$, $B$, $C$, $D$ będą czterema punktami na płaszczyźnie takimi, że $A$ i $B$ leżą po tej samej stronie prostej $CD$.
	Wtedy następujące warunki są równoważne:
	\begin{itemize}
		\item punkty $A$, $B$, $C$, $D$ leżą na jednym okręgu;
		\item kąty $\angle ACB$ i $\angle ADB$ są sobie równe;
		\item suma dwóch przeciwległych kątów czworokąta wypukłego wyznaczonego przez wierzchłki $A$, $B$, $C$, $D$ ma miarę kąta półpełnego.
	\end{itemize}
\end{proposition}

Jedna ze wspomnianych implikacji to wniosek \ref{ab_twice_pi}.
Piszą o tym Guzicki \cite[s. 11-13, 16, 17]{guzicki_2021}.
Neugebauer \cite[s. 25]{neugebauer_2018} wykorzystuje równoważność 1. i 3. warunku w dowodzie twierdzenia Wallace'a o trzech spodkach punktu na boki trójkąta.

W prostokąt, który nie jest kwadratem, nie można wpisać okręgu.

\begin{proposition}[okrąg wpisany w czworokąt]
	\label{prp_excircle}
	Dany jest czworokąt wypukły $ABCD$.
	Wtedy następujące warunki są równoważne:
	\begin{itemize}
		\item w czworokąt $ABCD$ można wpisać okrąg,
		\item $AB + CD = AD + BC$.
	\end{itemize}
	\index{okrąg!wpisany}%
\end{proposition}

Piszą o tym Guzicki \cite[s. 231-237]{guzicki_2021}, Neugebauer \cite[s. 35]{neugebauer_2018}.
https://www.deltami.edu.pl/2023/02/o-okregach-wpisanych-w-czworokaty/ Pitot, Steiner

\begin{proposition}[twierdzenie Newtona?]
	Dany jest czworokąt $ABCD$, w który wpisano okrąg, styczny do boków $AB$, $BC$, $CD$, $DA$ w punktach $K$, $L$, $M$, $N$.
	Wówczas proste $AC$, $BD$, $KM$ i $LN$ są współpękowe.
\end{proposition}

Piszą o tym Guzicki \cite[s. 237, 238]{guzicki_2021}, że jest to szczególny przypadek twierdzenia Brianchona, którego dowodzi najpierw dla sześciokąta.

\begin{proposition}[twierdzenie Miquela]
	Niech $ABC$ będzie trójkątem, na bokach którego ($AB$, $BC$, $AC$) wybrano punkty $C'$, $A'$, $B'$.
	Wtedy okręgi opisane na trójkątach $AB'C'$, $A'BC'$, $A'B'C$ przecinają się w~jednym punkcie.
\end{proposition}

Sformułowanie mówi tylko o trójkątach, ale dowód wykorzystuje własności czworokątów cyklicznych.
Hartshorne jako ćwiczenie \cite[s. 61]{hartshorne2000} napisze, że tym razem punkt Miquela zostanie nazwany na cześć osoby, która go odkryje w 1838 roku.
\index{punkt!Miquela}%
\todofoot{Guzicki ps. 29, 32}
Audin \cite[s. 104]{audin_2003} jako ,,the pivot'' (dla niego twierdzenie Miquela mówi o czterech okręgach, czyli nasz fakt \ref{miquel6}).
Patrz też \cite[s. 11]{komisarczyk_2024}.

\begin{proposition}[twierdzenie Miquela o pięciokącie]
	Niech $ABCDE$ będzie pięciokątem wypukłym, którego przedłużenia boków przecinają się w punktach $F$, $G$, $H$, $I$, $K$.
	Na trójkątach $CFD$, $DGE$, $EHA$, $AIB$, $BKC$ opisano okręgi.
	Wtedy ich nowe punkty przecięcia (różne od $A$, $B$, $C$, $D$, $E$) leżą na jednym okręgu. 
\end{proposition}

% Miquel, A. "Mémoire de Géométrie." J. de mathématiques pures et appliquées de Liouville 1, 485-487, 1838. ???

\begin{proposition}[twierdzenie Miquela o sześciu okręgach]
\label{miquel6}%
	Niech $\Gamma$ będzie okręgiem opisanym na czworokącie $ABCD$.
	Niech $\Gamma_1$, $\Gamma_2$, $\Gamma_3$, $\Gamma_4$ będą dowolnymi okręgami, które przechodzą przez $AB$, $BC$, $CD$, $DA$.
	Wtedy ich cztery nowe punkty przecięcia tworzą czworokąt cykliczny.
	\index{czworokąt!cykliczny}%
	% https://en.wikipedia.org/wiki/Miquel%27s_theorem#Miquel's_six_circle_theorem
	% It is also known as the four circles theorem and while generally attributed to Jakob Steiner the only known published proof was given by Miquel.[11]
	% Ostermann, Alexander; Wanner, Gerhard (2012), Geometry by its History, Springer, ISBN 978-3-642-29162-3 STRONA=352
\end{proposition}

https://www.deltami.edu.pl/2018/06/jesli/

%