
\section{Przybliżenia $\pi$}

% Eves s. 7
W papirusie Rhinda pojawi się wskazówka, że pole koła to w przybliżeniu pole kwadratu o boku tak długim jak $8/9$ średnica koła, co jest równoważne przybliżeniu $\pi \approx (4/3)^4 \approx 3.16049$.
Następne przybliżenie (dokładniejsze za cenę większego mianownika) to dopiero $31/35$, więć Egipcjanie mieli nosa.

Śulbasūtras, starożytny tekst hinduistyczny przybliży $\pi$ jako $36 / (6 + 4 \sqrt 2) \approx 3.08831$.

\todofoot{260 BC - Archimedes proved that the value of π lies between 3 + 1/7 (approx. 3.1429) and 3 + 10/71 (approx. 3.1408), that the area of a circle was equal to π multiplied by the square of the radius of the circle and that the area enclosed by a parabola and a straight line is 4/3 multiplied by the area of a triangle with equal base and height. He also gave a very accurate estimate of the value of the square root of 3.}

%