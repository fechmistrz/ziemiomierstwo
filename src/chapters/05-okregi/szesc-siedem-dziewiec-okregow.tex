%

\section{Twierdzenia o sześciu, siedmiu i dziewięciu okręgach}
\label{sssection_6_7_9_circles}
% TODO: pięciu, Guzicki ps. 36
Cecil John Alvin Evelyn, G. B. (pełne imiona nieznane) Money-Coutts i John Alfred Tyrrell znaleźli około 1974 roku piękne twierdzenia o okręgach wpisanych, które pokazują nam, jak wiele wyników geometrii elementarnej oczekuje na swoje odkrycie.

\begin{proposition}[o sześciu okręgach]
\index{twierdzenie!o sześciu okręgach}%
	Dany jest trójkąt $\triangle ABC$.
	Niech $\Gamma_1, \Gamma_2, \ldots, \Gamma_6$ będą okręgami wpisanymi kolejno w kąty przy wierzchołkach $A$, $B$, $C$, $A$, $B$, $C$, $A$ takimi, że każdy jest styczny do poprzedniego.
	Wtedy okręgi $\Gamma_6$ i $\Gamma_1$ też są do siebie styczne.
\end{proposition}

Pisze o tym Bogdańska, Neugebauer \cite[s. 101]{neugebauer_2018}: wykorzystują punkt Crelle'a-Brocarda, wzór Herona i trochę trygonometrii.
Tabacznikow, Iwanow \cite{ivanov_tabachnikov_2016} pokazali, że jeśli osłabimy założenia: okręgi nie muszą zawierać się w~trójkącie i~wystarczy, że będą styczne do prostych zawierających boki trójkąta, to nadal ciąg okręgów jest od pewnego miejsca okresowy z okresem równym sześć, ale osiągnięcie tego stanu może wymagać dowolnie wielu kroków.
\index[persons]{Tabacznikow, Siergiej (Табачников, Сергей Львович)}%
\index[persons]{Iwanow, Denis (Иванов, Денис)}%

\begin{proposition}[o siedmiu okręgach]
\index{twierdzenie!o siedmiu okręgach}%
	Dany jest okrąg $\Gamma$ oraz sześć okręgów stycznych do niego tak, że każdy jest też styczny do swoich dwóch sąsiadów.
	Wtedy trzy proste łączące przeciwległe punkty styczności przecinają się w jednym punkcie.
\end{proposition}

\begin{proposition}[o dziewięciu okręgach]
\index{twierdzenie!o dziewięciu okręgach}%
	Niech $\Gamma_1$, $\Gamma_2$, $\Gamma_3$ będą trzema okręgami na płaszczyźnie, zaś okrąg $\Gamma_4$ będzie styczny do $\Gamma_2$ i $\Gamma_3$.
	Kreślimy ciąg okręgów stycznych do poprzednich:
	$\Gamma_5$ do $\Gamma_1$, $\Gamma_3$ i $\Gamma_4$,
	$\Gamma_6$ do $\Gamma_1$, $\Gamma_2$ i $\Gamma_5$,
	$\Gamma_7$ do $\Gamma_2$, $\Gamma_3$ i $\Gamma_6$,
	$\Gamma_8$ do $\Gamma_1$, $\Gamma_3$ i $\Gamma_7$,
	$\Gamma_9$ do $\Gamma_1$, $\Gamma_2$ i $\Gamma_8$.
	Wtedy przy dobrym wyborze, który okrąg styczny wziąć, okrąg $\Gamma_4$ jest też styczny do $\Gamma_2$, $\Gamma_3$ i $\Gamma_9$.
\end{proposition}

%