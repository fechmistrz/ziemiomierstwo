%

\section{Styczne i dwustyczne}
\begin{proposition}
    Niech $\Gamma$ będzie okręgiem o środku $O$ oraz promieniu $OA$.
    Wtedy prosta prostopadła do $OA$, która przechodzi przez $A$, jest styczną do okręgu, leżącą (poza punktem $A$) na zewnątrz okręgu $\Gamma$.
    Odwrotnie, każda prosta, która jest styczna w punkcie $A$ do okręgu $\Gamma$, musi być prostopadła do prostej $OA$.
    \index{styczność}
\end{proposition} % Hartshorne 105

\begin{corollary}
    Przez każdy punkt okręgu przechodzi dokładnie jedna styczna do tego okręgu.
\end{corollary} % Hartshorne 105

Przez każdy punkt leżący na zewnątrz okręgu przechodzą dwie proste -- Neugebauer \cite[s. 34]{neugebauer_2018} z rozmachem nazywa to \emph{,,zasadniczym twierdzeniem planimetrii''}

\begin{corollary}
    Prosta, która nie jest styczna do okręgu i nie jest z nim rozłączna, musi przecinać go w dokładnie dwóch punktach.
\end{corollary} % Hartshorne 106

\begin{proposition}
    Niech $O_1, O_2, A$ będą trzema punktami.
    Następujące warunki są równoważne: punkty $A, O_1, O_2$ są współliniowe; okręgi o promieniach $O_1A$, $O_2A$ są styczne.
\end{proposition} % Hartshorne 105

\begin{corollary}
    Dwa okręgi, które nie są rozłączne i nie są styczne, mają dokładnie dwa punkty wspólne.
\end{corollary} % Hartshorne 106

\todofoot{Dwustyczne (do dwóch okręgów).}
\index{dwustyczna}

%