%

\index{nierówność!Snelliusa-Huygensa}%
\begin{proposition}[nierówność Snelliusa-Huygensa]
	\begin{equation}
		2 \sin x + \tan x > 3x.
		\label{nierownosc_snelliusa}
	\end{equation}
\end{proposition}

Willebrod Snellius poda zawiły dowód w książce Cyclometicus (1631); schludniejsze uzasadnienie znajdzie Christian Huygens.
\index[persons]{Snellius, Willebrod}%
\index[persons]{Huygens, Christian}%
W \emph{,,De circuli magnitudine inventa''} użyje nierówności \ref{nierownosc_mikolaja_z_kuzy} oraz \ref{nierownosc_snelliusa} do oszacowania wartości $\pi$.

%