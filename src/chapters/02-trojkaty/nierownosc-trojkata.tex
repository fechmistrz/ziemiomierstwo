%

% PRZECZYTANO: https://en.wikipedia.org/wiki/Triangle_inequality

\section{Nierówność trójkąta}

\todofoot{https://www.deltami.edu.pl/2012/06/punkt-w-trojkacie/ zadanie 4 z odnośnikiem do tw. Ptolemeusza.}

Zajmiemy się teraz (I.20).

\begin{proposition}[nierówność trójkąta]
\index{nierówność!trójkąta}%
	Niech $ABC$ będzie trojkątem.
	Wtedy suma odcinków $AB$ i $BC$ jest dłuższa niż $AC$.
\end{proposition}

% TODO: https://www.deltami.edu.pl/2013/09/siatka-czworoscianu/ to samo w R3

\begin{proof}
	Wynika to ze wzoru Herona (fakt \ref{prp_heron}) i tego, że pole trójkąta jest nieujemne.
\end{proof}

\begin{corollary}
	Niech $a \ge b \ge c$ będą bokami trójkąta.
	Wtedy
	% \begin{equation}
	% 	1 < \frac{a + c}{b} < 3 
	% \end{equation}
	% oraz
	\begin{equation}
		1 \le \min \left(\frac ab, \frac bc\right) \le \phi = \frac {1 + \sqrt 5}{2}.
		% American Mathematical Monthly, pp. 49-50, 1954. 
	\end{equation}
\end{corollary}

Nierówność trójkąta nie jest wnioskiem z aksjomatów I1-I3, B1-B4, C1-C3, ponieważ nie zachodzi w następującym modelu (ukradniętym Hartshorne'owi \cite[s. 90]{hartshorne2000}):

\begin{example}
	Rozpatrujemy zbiór $\mathbb R^2$ jako płaszczyznę ze standardowymi punktami oraz prostymi, ale niestandardową metryką
	\begin{equation}
		d((x_1, y_1), (x_2, y_2)) = \begin{cases}
			\sqrt{(x_1-x_2)^2 + (y_1-y_2)^2} & \text{jeśli } x_1 = x_2 \vee y_1 = y_2, \\
			2 \sqrt{(x_1-x_2)^2 + (y_1-y_2)^2} & \text{w przeciwnym wypadku}
		\end{cases}.
	\end{equation}
	Wtedy nierówność trójkąta nie zachodzi.
\end{example}

Przytoczone niżej dwa zadania: Fagnano i Fermata będą prawie zawsze omawiane razem.

%

% https://en.wikipedia.org/wiki/Fagnano's_problem

\begin{problem}[zadanie Fagnano]
	Dany jest trójkąt ostrokątny $ABC$.
	Wpisać w niego trójkąt $DEF$ (czyli wybrać punkt $D$ na boku $BC$, $E$ na $AC$, $F$ na $AB$ tak, by tworzyły trójkąt) o możliwie najmniejszym obwodzie.
\index{zadanie!Fagnano}%
\end{problem}

Coxeter \cite[s. 36, 37]{coxeter_1967} pokaże tak jak Fejer, że rozwiązaniem zadania jest trójkąt spodkowy (którego wierzchołkami są spodki wysokości danego trójkąta; zwany brzydko ortycznym).
Zetel \cite[s. 97-100]{zetel_2020} przytoczy oprócz dowodu Fejera także dowód Izwolskiego, oparty na twierdzeniu: promienie okręgu opisanego na trójkącie, przechodzące przez wierzchołki trójkąta, są prostopadłe do odpowiednich boków trójkąta spodkowego.
Pompe \cite[s. 16-18]{pompe_2022} wykorzysta własności obrotów.
Audin \cite[s. 101]{audin_2003} podaje ten fakt w~formie ćwiczenia. % todo: fagnano czy gemrat?

Zadanie traci sens dla trójkątów nieostrokątnych: wtedy dwa z trzech wierzchołków trójkąta wpisanego o najmniejszym obwodzie pokrywają się z wierzchołkiem kąta rozwartego.

\begin{corollary}
	Wysokości trójkąta $ABC$ są dwusiecznymi kątów jego trójkąta spodkowego. % Pompe, Wokół obrotów, wniosek 2.3
\end{corollary}

Ten wniosek pojawia się u Zetela \cite[s. 89]{zetel_2020} i Pompego \cite[s. 19]{pompe_2022}.

\begin{proposition}
    Boki trójkąta spodkowego są antyrównoległe do boków trójkąta danego.
\end{proposition}

(Zetel \cite[s. 89]{zetel_2020}).

\begin{proposition}
	Niech wysokości trójkąta ABC przecinają sięw punkcie H.
	Każdy z czterech punktów A, B, C, H jest ortocentrum trójkąta, którego wierzchołkami są trzy pozostałe punkty.
\end{proposition}

(Zetel \cite[s. 95]{zetel_2020}).

%
%

\todofoot{https://www.deltami.edu.pl/2013/05/kacik-przestrzenny-17-punkt-fermata-torricellego/}
\todofoot{https://www.deltami.edu.pl/2018/01/zagadnienie-fermata-w-jednej-linijce/
}
\todofoot{https://www.deltami.edu.pl/2019/09/w-poszukiwaniu-trojkata-rownobocznego/ 14}
\todofoot{Zadanie Fermata -- Neugebauer, s. 117.}

\begin{problem}[zadanie Fermata]
	\label{punkt_fermata}
	Dany jest trójkąt ostrokątny $ABC$.
	Znaleźć punkt $P$ taki, by suma $|PA| + |PB| + |PC|$ była możliwie najmniejsza.
\index{zadanie!Fermata}%
\end{problem}

Powyższe zadanie rozwiąże Evangelista Torricelli (dlatego też punkt $P$ nazywa się czasem punktem Torricellego; robi tak Guzicki \cite[s. 224-228]{guzicki_2021}), który dostanie je w formie wyzwania od Fermata.
\index[persons]{Torricelli, Evangelista}%.
Rozwiązanie opublikuje student Torricelliego, Vincenzo Viviani, w 1659 roku.
\index[persons]{Viviani, Vincenzo}
% TODO: Johnson, R. A. Modern Geometry: An Elementary Treatise on the Geometry of the Triangle and the Circle. Boston, MA: Houghton Mifflin, pp. 221-222, 1929.
Coxeter \cite[s. 37]{coxeter_1967} przytoczy rozwiązanie Hofmanna\todofoot{J E Hoffman, Elementare Losung einer Mimimumsaufgabe 1929}
Patrz też Pompe \cite[s. 19-22]{pompe_2022}, Neugebauer \cite[s. 117]{neugebauer_2018}.
Audin \cite[s. 105]{audin_2003} podaje ten fakt w formie ćwiczenia.

Założenie ostrokątności można osłabić, wystarczy, żeby każdy kąt wewnętrzny trójkąta miał miarę mniejszą niż $2\pi/3$.

\begin{proposition}
    Dany jest trójkąt $ABC$, w którym żaden kąt nie ma miary większej niż $2\pi/3$.
    Wtedy wewnątrz istnieje dokładnie jeden punkt $P$ taki, że kąty $\angle APB$, $\angle BPC$, $\angle CPA$ są przystające (i miarę $2\pi/3$). 
\end{proposition}

Punkt $P$ jest punktem Fermata i można go łatwo skontruować.
Na zewnątrz boków trójkąta $ABC$ kreślimy trzy trójkąty równoboczne $ABD$, $BCE$, $CAF$, po czym prowadzimy proste $AE$, $BF$, $CD$.
Punkt $P$ leży na przecięciu wszystkich trzech, co wynika z twierdzenia Jacobiego:

\begin{proposition}
    Na bokach trójkąta $ABC$, po jego zewnętrznej stronie, zbudowano takie trójkąty $ABD$, $BCE$, $CAF$, że kąty: $\angle BAD = \angle CAF$, $\angle ABD = \angle CBE$, $\angle BCE = ACF$ są równe.
    Wówczas proste $AE$, $BF$, $CD$ przecinają się w jednym punkcie.
\end{proposition}

O twierdzeniu tym wspomni Pompe \cite[s. 21]{pompe_2022}, a następnie przejdzie do twierdzenia \ref{twierdzenie_ptolemeusza} (czyli Ptolemeusza).

%

%