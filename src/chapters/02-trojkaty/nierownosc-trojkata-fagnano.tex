%

% https://en.wikipedia.org/wiki/Fagnano's_problem

\begin{problem}[zadanie Fagnano]
	Dany jest trójkąt ostrokątny $ABC$.
	Wpisać w niego trójkąt $DEF$ (czyli wybrać punkt $D$ na boku $BC$, $E$ na $AC$, $F$ na $AB$ tak, by tworzyły trójkąt) o możliwie najmniejszym obwodzie.
\index{zadanie!Fagnano}%
\end{problem}

Coxeter \cite[s. 36, 37]{coxeter_1967} pokaże tak jak Fejer, że rozwiązaniem zadania jest trójkąt spodkowy (którego wierzchołkami są spodki wysokości danego trójkąta; zwany brzydko ortycznym).
Zetel \cite[s. 97-100]{zetel_2020} przytoczy oprócz dowodu Fejera także dowód Izwolskiego, oparty na twierdzeniu: promienie okręgu opisanego na trójkącie, przechodzące przez wierzchołki trójkąta, są prostopadłe do odpowiednich boków trójkąta spodkowego.
Pompe \cite[s. 16-18]{pompe_2022} wykorzysta własności obrotów.
Audin \cite[s. 101]{audin_2003} podaje ten fakt w~formie ćwiczenia. % todo: fagnano czy gemrat?

Zadanie traci sens dla trójkątów nieostrokątnych: wtedy dwa z trzech wierzchołków trójkąta wpisanego o najmniejszym obwodzie pokrywają się z wierzchołkiem kąta rozwartego.

\begin{corollary}
	Wysokości trójkąta $ABC$ są dwusiecznymi kątów jego trójkąta spodkowego. % Pompe, Wokół obrotów, wniosek 2.3
\end{corollary}

Ten wniosek pojawia się u Zetela \cite[s. 89]{zetel_2020} i Pompego \cite[s. 19]{pompe_2022}.

\begin{proposition}
    Boki trójkąta spodkowego są antyrównoległe do boków trójkąta danego.
\end{proposition}

(Zetel \cite[s. 89]{zetel_2020}).

\begin{proposition}
	Niech wysokości trójkąta ABC przecinają sięw punkcie H.
	Każdy z czterech punktów A, B, C, H jest ortocentrum trójkąta, którego wierzchołkami są trzy pozostałe punkty.
\end{proposition}

(Zetel \cite[s. 95]{zetel_2020}).

%