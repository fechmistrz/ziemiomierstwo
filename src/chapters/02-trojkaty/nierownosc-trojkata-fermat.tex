%

\todofoot{https://www.deltami.edu.pl/2013/05/kacik-przestrzenny-17-punkt-fermata-torricellego/}
\todofoot{https://www.deltami.edu.pl/2018/01/zagadnienie-fermata-w-jednej-linijce/
}
\todofoot{https://www.deltami.edu.pl/2019/09/w-poszukiwaniu-trojkata-rownobocznego/ 14}
\todofoot{Zadanie Fermata -- Neugebauer, s. 117.}

\begin{problem}[zadanie Fermata]
	\label{punkt_fermata}
	Dany jest trójkąt ostrokątny $ABC$.
	Znaleźć punkt $P$ taki, by suma $|PA| + |PB| + |PC|$ była możliwie najmniejsza.
\index{zadanie!Fermata}%
\end{problem}

Powyższe zadanie rozwiąże Evangelista Torricelli (dlatego też punkt $P$ nazywa się czasem punktem Torricellego; robi tak Guzicki \cite[s. 224-228]{guzicki_2021}), który dostanie je w formie wyzwania od Fermata.
\index[persons]{Torricelli, Evangelista}%.
Rozwiązanie opublikuje student Torricelliego, Vincenzo Viviani, w 1659 roku.
\index[persons]{Viviani, Vincenzo}
% TODO: Johnson, R. A. Modern Geometry: An Elementary Treatise on the Geometry of the Triangle and the Circle. Boston, MA: Houghton Mifflin, pp. 221-222, 1929.
Coxeter \cite[s. 37]{coxeter_1967} przytoczy rozwiązanie Hofmanna\todofoot{J E Hoffman, Elementare Losung einer Mimimumsaufgabe 1929}
Patrz też Pompe \cite[s. 19-22]{pompe_2022}, Neugebauer \cite[s. 117]{neugebauer_2018}.
Audin \cite[s. 105]{audin_2003} podaje ten fakt w formie ćwiczenia.

Założenie ostrokątności można osłabić, wystarczy, żeby każdy kąt wewnętrzny trójkąta miał miarę mniejszą niż $2\pi/3$.

\begin{proposition}
    Dany jest trójkąt $ABC$, w którym żaden kąt nie ma miary większej niż $2\pi/3$.
    Wtedy wewnątrz istnieje dokładnie jeden punkt $P$ taki, że kąty $\angle APB$, $\angle BPC$, $\angle CPA$ są przystające (i miarę $2\pi/3$). 
\end{proposition}

Punkt $P$ jest punktem Fermata i można go łatwo skontruować.
Na zewnątrz boków trójkąta $ABC$ kreślimy trzy trójkąty równoboczne $ABD$, $BCE$, $CAF$, po czym prowadzimy proste $AE$, $BF$, $CD$.
Punkt $P$ leży na przecięciu wszystkich trzech, co wynika z twierdzenia Jacobiego:

\begin{proposition}
    Na bokach trójkąta $ABC$, po jego zewnętrznej stronie, zbudowano takie trójkąty $ABD$, $BCE$, $CAF$, że kąty: $\angle BAD = \angle CAF$, $\angle ABD = \angle CBE$, $\angle BCE = ACF$ są równe.
    Wówczas proste $AE$, $BF$, $CD$ przecinają się w jednym punkcie.
\end{proposition}

O twierdzeniu tym wspomni Pompe \cite[s. 21]{pompe_2022}, a następnie przejdzie do twierdzenia \ref{twierdzenie_ptolemeusza} (czyli Ptolemeusza).

%