\begin{proposition}[nierówność Eulera]
\index{nierówność!Eulera}%
	Niech $R, r, d$ oznaczają kolejno promienie okręgu opisanego i wpisanego w~trójkąt oraz odległość między środkami tych okręgów.
	Wtedy
	\begin{equation}
	\frac{1}{R-d} + \frac{1}{R+d} = \frac 1 r,
	\end{equation}
	lub równoważnie $d^2 = R \cdot (R - 2r)$.
	Wynika stąd, że $R \ge 2r$.
\end{proposition}

https://www.deltami.edu.pl/2019/03/twierdzenie-o-trojzebie/

Leonhard Euler opublikuje ten wynik w 1765 roku, chociaż wcześniej zrobi to William Chapple, w pracy \emph{,,An essay on the properties of triangles inscribed in and circumscribed about two given circles''} z 1746 roku, na dole strony 123.
% Leversha, Gerry; Smith, G. C. (November 2007), "Euler and triangle geometry", The Mathematical Gazette, 91 (522): 436–452, doi:10.1017/S0025557200182087, JSTOR 40378417, S2CID 125341434
% TODO: https://en.wikipedia.org/wiki/Euler%27s_theorem_in_geometry a stornger version... [6]
Wspomni go Zetel \cite[s. 75]{zetel_2020}

Trójwymiarowym odpowiednikiem jest:

\begin{proposition}[nierówność Grace'a-Danielssona]
\index{nierówność!Grace'a-Danielssona}%
	Dla każdej pary sfer o promieniach $r, R$ takich, że $r < R$ istnieje czworościan zawarty w dużej sferze i zawierający małą sferę wtedy i tylko wtedy, gdy
	\begin{equation}d^2 \le (R+r)(R - 3r).\end{equation}
\end{proposition}
% TODO: Grace, J.H. (1918), Proc. London Math. (ed.), Tetrahedra in relation to spheres and quadrics, vol. Soc.17, pp. 259–271
% Danielsson, G. (1952), Johan Grundt Tanums Forlag (ed.), Proof of the inequality d2≤(R+r)(R−3r) for the distance between the centres of the circumscribed and inscribed spheres of a tetrahedron, pp. 101–105