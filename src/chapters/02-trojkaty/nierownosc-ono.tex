%

\subsection{Nierówność Ono}

% https://en.wikipedia.org/wiki/Ono%27s_inequality

W 1914 roku Toda Ono wysnuje hipotezę (nieprawdziwą w ogólności, ale prawdziwą dla wszystkich trójkątów ostrokątnych): % 小野藤太
\index[persons]{Ono, Toda}%

\begin{proposition}[nierówność Ono]
    W trójkącie ostrokątnym o bokach długości $a, b, c$ oraz polu powierzchni $S$ zachodzi nierówność
    \begin{equation}
        \frac{27}{4096} \cdot (b^2 + c^2 - a^2)^2 (c^2 + a^2 - b^2)^2 (a^2 + b^2 - c^2)^2 \le S^6.
    \end{equation}
\end{proposition}

Dowód znajdzie F. Balitrand w 1916 roku.
\index[persons]{Balitrand, F?}%
% Ono, T. (1914). "Problem 4417". Interméd. Math. 21: 146.
% https://zbmath.org/?format=complete&q=an:46.0859.06
Nierówność nie zachodzi dla trójkąta o bokach długości $a = 2$, $b = 3$, $c = 4$, wtedy $S^2 = 135/16$.

%