%

\section{Twierdzenie Talesa}

\todofoot{Guzicki-3}

\begin{theorem}[Talesa]
\index{twierdzenie!Talesa}%
    Jeśli ramiona kąta płaskiego przetnie się 2 równoległymi prostymi:
    \begin{center}
\begin{comment}
        \begin{tikzpicture}
            \tkzDefPoint(0, 0.5){O}
            \tkzDefPoint(1.5, 0){A}
            \tkzDefPoint(2, 1){Ap}
            \tkzDefPointBy[homothety=center O ratio 1.618](A) \tkzGetPoint{B}
            \tkzDefLine[parallel=through B](A,Ap) \tkzGetPoint{Bp}
            \tkzInterLL(O,Ap)(B,Bp) \tkzGetPoint{Bpp}
            \tkzDrawPoints[fill=gray,opacity=.9](O,A,B,Ap,Bpp)
            \tkzLabelPoint[above](O){$O$}
            \tkzLabelPoint[below](A){$A$}
            \tkzLabelPoint[below](B){$A'$}
            \tkzLabelPoint[above left](Bpp){$B'$}
            \tkzLabelPoint[above left](Ap){$B$}
            \tkzDrawLine[thick](O,B)
            \tkzDrawLine[thick](O,Bpp)
            \tkzDrawLine[color=blue, thick](A,Ap)
            \tkzDrawLine[color=blue, thick](B,Bpp)
        \end{tikzpicture}
\end{comment}
        \end{center}
    to długości odcinków wyznaczonych przez te proste na jednym z ramion kąta są proporcjonalne do długości odpowiednich odcinków na drugim ramieniu kąta, a zatem
    \begin{equation}
        \label{thales_ratio}
        \frac{|OA|}{|OA'|} = \frac{|OB|}{|OB'|} = \frac{|AB|}{|A'B'|}.
    \end{equation}
\end{theorem}
% TODO: https://en.wikipedia.org/wiki/Thales's_theorem <- to nie jest to XD

https://www.deltami.edu.pl/2017/04/dedukcja-lokalna-na-przykladzie/

Tradycja przypisuje jego sformułowanie Talesowi z Miletu, chociaż znane będzie starożytnym Babilończykom i Egipcjanom.
\index[persons]{Tales z Miletu}%
Najstarszy zachowany dowód twierdzenia Talesa poda Euklides (VI.2).
Napiszą o nim Neugebauer, Bogdańska \cite[s. 48-56]{neugebauer_2018}; Audin \cite[s. 24, 173]{audin_2003}.

Po angielsku znane będzie jako \emph{intercept theorem}, \emph{basic proportionality theorem} albo \emph{side splitter theorem}; natomiast nazwy \emph{Thales's theorem} użyje się o~dziwo wobec innego wyniku, że kąt oparty na półokręgu jest prosty (!).
Po niemiecku będą dominować \emph{Strahlensatz} (twierdzenie o półprostych) albo \emph{Vierstreckensatz (o czterech odcinkach)}.
W innych językach tak jak po polsku.

Prawdziwe jest również twierdzenie odwrotne:

\begin{proposition}[twierdzenie odwrotne do tw. Talesa]
    Jeżeli pewna prosta przecina boki $OA'$, $OB'$ trójkąta $OA'B'$ w różnych punktach $A$ i $B$ odpowiednio, a przy tym zachodzi równość \ref{thales_ratio}, to prosta ta jest równoległa do prostej $A'B'$.
\end{proposition}

Prostym wnioskiem z twierdzenia Talesa jest fakt \ref{hartshorne_52}, znajduje on zastosowanie w dowodzie:
% Neugebauer s. 52

\begin{theorem}[Varignona]
    \label{theorem_varignon}
    Czworokąt $PQRS$, którego wierzchołki leżą na środkach boków $AB$, $BC$, $CD$, $DA$ czworokąta $ABCD$, jest równoległobokiem.
    Jego znakowane (!) pole jest równe połowie pola czworokąta $ABCD$. % Neugebauer s. 61
\end{theorem}

% TODO po skasowaniu Neugebauer-s-61 wyżej: The area of the Varignon parallelogram equals half the area of the original quadrilateral. This is true in convex, concave and crossed quadrilaterals provided the area of the latter is defined to be the difference of the areas of the two triangles it is composed of. => [[Varignon's theorem]]

W szczególności, czworokąt $ABCD$ nie musi być wypukły; może być nawet ,,motylkiem'', czyli łamaną zamkniętą z samoprzecięciami.
Twierdzenie zostanie nazwane na cześć Pierre'a Varignona (pośmiertnie) w 1731 roku.
\index[persons]{Varignon, Pierre}%
Co więcej,

\begin{proposition}
    Następujące warunki są równoważne:
    \begin{itemize}
        \item przekątne czworokąta $ABCD$ są równej długości,
        \item równoległobok Varignona $PQRS$ związany z czworokątem $ABCD$ jest rombem.
    \end{itemize}
\index{równoległobok!Varignona}%
\index{romb}%
\end{proposition}

\begin{proposition}
    Następujące warunki są równoważne:
    \begin{itemize}
        \item przekątne czworokąta $ABCD$ są prostopadłe do siebie,
        \item równoległobok Varignona $PQRS$ związany z czworokątem $ABCD$ jest prostokątem.
    \end{itemize}
\index{równoległobok!Varignona}%
\index{prostokąt}%
\end{proposition}

% Michael de Villiers "Some Adventures in Euclidean Geometry", strona 58 wspomina to, ale nie podaje dowodu, bo zostawił go dla Czytelnika.

%