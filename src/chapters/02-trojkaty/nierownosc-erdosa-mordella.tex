%

\index{nierówność!Erdősa-Mordella|(}

\subsection{Nierówność Erdősa-Mordella i jej uogólnienie}

\begin{theorem}[nierówność Erdősa-Mordella]
    Niech $P$ będzie punktem wewnątrz trójkąta $\triangle ABC$, zaś $A_p, B_p, C_p$ spodkami punktu $P$ na boki trójkąta jak na rysunku \ref{erdos_mordell_barrowa}.
    Wtedy
    \begin{equation}
        |PA| + |PB| + |PC| \ge 2 (|PA_p| + |PB_p| + |PC_p|).
    \end{equation}
\end{theorem}

\begin{figure}[H] \centering
\begin{minipage}[b]{.45\linewidth}
\begin{center}
    \begin{comment}
    \begin{tikzpicture}[scale=.4]
    \tkzDefPoint(0, 0){A}
    \tkzDefPoint(10, 2){B}
    \tkzDefPoint(6, 7){C}
    \tkzDefPoint(5, 3){P}
    \tkzLabelPoint[below left](A){$A$}
    \tkzLabelPoint[below right](B){$B$}
    \tkzLabelPoint[above](C){$C$}
    \tkzLabelPoint[below left](P){$P$}
    \tkzDefPointsBy[projection=onto A--B](P){Pc}
    \tkzDefPointsBy[projection=onto B--C](P){Pa}
    \tkzDefPointsBy[projection=onto C--A](P){Pb}
    \tkzLabelPoint[above right](Pa){$A_p$}
    \tkzLabelPoint[above left](Pb){$B_p$}
    \tkzLabelPoint[below](Pc){$C_p$}

    \tkzDrawSegments[line width=0.2mm,dashed](P,Pa P,Pb P,Pc)
    \tkzDrawPolygon[line width=0.3mm](A,B,C)
    \tkzMarkRightAngles[size=0.5](P,Pa,C P,Pb,A P,Pc,B)
    \tkzDrawPoints[size=3,color=black,fill=black!50](A,B,C,P,Pc,Pb,Pa)
\end{tikzpicture}
\end{comment}
    \end{center}
    \subcaption{nierówność Erdősa-Mordella}
    \label{erdos_mordell_barrowa}
\end{minipage}
%
\begin{minipage}[b]{.45\linewidth}
\begin{center}\begin{comment}
    \begin{tikzpicture}[scale=.4]
    \tkzDefPoint(0, 0){A}
    \tkzDefPoint(10, 2){B}
    \tkzDefPoint(6, 7){C}
    \tkzDefPoint(5, 3){P}

    \tkzDefLine[bisector](A,P,B) \tkzGetPoint{prePc}
    \tkzInterLL(P,prePc)(A,B) \tkzGetPoint{Pc}
    \tkzDefLine[bisector](B,P,C) \tkzGetPoint{prePa}
    \tkzInterLL(P,prePa)(B,C) \tkzGetPoint{Pa}
    \tkzDefLine[bisector](C,P,A) \tkzGetPoint{prePb}
    \tkzInterLL(P,prePb)(C,A) \tkzGetPoint{Pb}

    \tkzLabelPoint[below left](A){$A$}
    \tkzLabelPoint[below right](B){$B$}
    \tkzLabelPoint[above](C){$C$}
    %\tkzLabelPoint[below left](P){$P$}
    \tkzLabelPoint[above right](Pa){$A_p$}
    \tkzLabelPoint[above left](Pb){$B_p$}
    \tkzLabelPoint[below](Pc){$C_p$}

    \tkzMarkAngle[arc=lll,size=1.2,mark=|||](A,P,Pc)
    \tkzMarkAngle[arc=lll,size=1.2,mark=|||](Pc,P,B)
    \tkzMarkAngle[arc=ll,size=1.2,mark=||](B,P,Pa)
    \tkzMarkAngle[arc=ll,size=1.2,mark=||](Pa,P,C)
    \tkzMarkAngle[arc=l,size=1.2,mark=|](C,P,Pb)
    \tkzMarkAngle[arc=l,size=1.2,mark=|](Pb,P,A)

    \tkzDrawSegments[line width=0.2mm](P,A P,B P,C)
    \tkzDrawSegments[line width=0.2mm,dashed](P,Pa P,Pb P,Pc)
    \tkzDrawPolygon[line width=0.3mm](A,B,C)
    \tkzDrawPoints[size=3,color=black,fill=black!50](A,B,C,P,Pc,Pb,Pa)
\end{tikzpicture}
\end{comment}
    \end{center}
    \subcaption{nierówność Barrowa}
    \label{erdos_mordell_barrowb}
\end{minipage}
\caption{}
\end{figure}

\label{subsection_erdos_mordell}
Nierówność pojawi się w wakacyjnym wydaniu American Mathematical Monthly z 1935 roku jako problem 3740 od Erdősa.
\index[persons]{Erdős, Paul}%
Na dowód przyjdzie poczekać.
Pierwszy podadzą Louis Mordell i~David Barrow \cite{mordell_barrow_1937} w 1937 roku.
\index[persons]{Mordell, Louis Joel}%
\index[persons]{Barrow, David}%
Po dwudziestoletniej przerwie pojawi się prostszy dowód Kazarinowa \cite{kazarinoff_1957};
\index[persons]{Kazarinoff, Donat}% Donat Konstantinovich Kazarinoff
Leon Bankoff \cite{bankoff_1958} uzasadni nierówność wykorzystując jedynie podobieństwo trójkątów oraz fakt, że $t + 1/t \ge 2$ dla $t \ge 0$; cały dowód zmieści się na połowie strony w $\Delta_{90}^8$.
\index[persons]{Bankoff, Leon}%
Ostatni współczesny dowód pochodzi od Claudi Alsiny i Rogera Nelsena \cite{alsina_2007}.
\index[persons]{Alsina, Claudi}%
\index[persons]{Nelsen, Roger}%

% https://en.wikipedia.org/wiki/Erdős-Mordell_inequality
Coxeter \cite[s. 25]{coxeter_1967} przetworzy je po latach w ćwiczenie dla Czytelnika;
podobnie postąpi Audin \cite[s. 102]{audin_2003}, nie szczędząc przy tym wskazówek.
W geometrii absolutnej nierówność Erdősa-Mordella jest równoważna temu, że suma kątów trójkąta nie przekracza $\pi$ (patrz Pambuccian \cite{pambuccian_2008}).
\index[persons]{Pambuccian, Victor}%

Wzmocnieniem nierówności Erdősa-Mordella będzie nierówność Barrowa:
\index{nierówność!Barrowa}

% TODO: https://en.wikipedia.org/wiki/Barrow%27s_inequality

\begin{theorem}[nierówność Barrowa]
    Niech $P$ będzie punktem wewnątrz trójkąta $\triangle ABC$, zaś $A_p$, $B_p$, $C_p$ punktami przecięć dwusiecznych trzech kątów wyznaczanych przez $P$ i pary wierzchołków trójkąta; tak jak na rysunku \ref{erdos_mordell_barrowb}.
    Wtedy
    \begin{equation}
        |PA| + |PB| + |PC| \ge 2 (|PA_p| + |PB_p| + |PC_p|).
    \end{equation}
\end{theorem}

Dowód Barrowa i Mordella \cite{mordell_barrow_1937} zostanie opublikowany w 1937 roku, ale nazwa ,,nierówność Barrowa'' przyjmie się nieco później, na pewno w 1961 roku Alexander Oppenheim będzie już jej używać.
\index[persons]{Oppenheim, Alexander}%
% Oppenheim, A. (1961), "New inequalities for a triangle and an internal point", Annali di Matematica Pura ed Applicata, 53: 157–163, doi:10.1007/BF02417793, MR 0124774
Mordell \cite{mordell_1962} wróci do nierówności po latach i poda prostszy dowód.
\index[persons]{Mordell, Louis Joel}%

\index{nierówność!Erdősa-Mordella|)}

%