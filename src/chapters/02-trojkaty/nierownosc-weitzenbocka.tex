%

\subsection{Nierówność Weitzenböcka i jej uogólnienia}

% https://en.wikipedia.org/wiki/Weitzenböck's_inequality

Roland Weitzenböck \cite{weitzenbock_1919} udowodni w 1919 roku:
\index[persons]{Weitzenböck, Roland}%

\begin{proposition}[nierówność Weitzenböcka]
	W trójkącie o bokach długości $a, b, c$ oraz polu powierzchni $S$ zachodzi nierówność
    \begin{equation}
        a^2 + b^2 + c^2 \ge 4 \sqrt{3} S,
    \end{equation}
    z równością wtedy i tylko wtedy, gdy trójkąt jest równoboczny: $a = b = c$.
	\index{nierówność!Weitzenböcka}%
\end{proposition}

% https://en.wikipedia.org/wiki/Weitzenböck%27s_inequality#Further_proofs
Dowód wykorzystuje punkt Fermata, wzór Herona, nierówność między średnią arytmetyczną i~geometryczną, własności funkcji cotangens albo nierówność
\begin{equation}
    (a^2 - b^2)^2 + (b^2 - c^2)^2 + (c^2 - a^2)^2 \ge 0.
\end{equation}
\index{punkt Fermata}%
\index{wzór Herona}%
% https://www.ime.usp.br/~toscano/disc/2022/GreenbergGeometry.pdf ps. 34

Są dwa uogólnienia: nierówność Hadwigera-Finslera oraz nierówność (Neuberga-)Pedoego.

\begin{proposition}[nierówność Hadwigera-Finslera]
    W trójkącie o bokach długości $a, b, c$ oraz polu powierzchni $S$ zachodzi nierówność
    \begin{equation}
        a^2 + b^2 + c^2 \ge (a-b)^2 + (b-c)^2 + (a-c)^2 + 4 \sqrt{3} S,
    \end{equation}
\end{proposition}

\begin{proof}
    Z twierdzenia cosinusów mamy $a^2 = b^2 + c^2 - 2bc \cos \alpha$.
    Ponieważ $S = \frac 1 2 b c \sin \alpha$, poprzednią równość możemy przepisać do 
    \begin{align}
        a^2 & = (b-c)^2 + 4S \cdot \frac{1 - \cos \alpha}{\sin \alpha} \\
        & = (b-c)^2 + 4S \cdot \frac{2 \sin^2 (\alpha / 2)}{2 \sin (\alpha / 2) \cos (\alpha / 2)} \\
        & = (b-c)^2 + 4S \cdot \tan \frac \alpha 2.
    \end{align}
    Po powtórzeniu tego dla pozostałych boków i dodaniu do siebie dostaniemy
    \begin{equation}
        a^2 + b^2 + c^2 = (a-b)^2 + (b-c)^2 + (c-a)^2 + 4S \left(\tan \frac \alpha 2 + \tan \frac \beta 2 + \tan \frac \gamma 2\right).
    \end{equation}
    Ale połówki kątów trójkąta nie przekraczają $\pi/2$, zaś funkcja $\tan$ jest wypukła, więc nawias można ograniczyć z dołu:
    \begin{equation}
        \tan \frac \alpha 2 + \tan \frac \beta 2 + \tan \frac \gamma 2 \ge 3 \tan \frac{\alpha + \beta + \gamma}{6} = 3\tan \frac \pi 6 = \sqrt 3,
    \end{equation}
    co kończy dowód.
\end{proof}

Paul Finsler oraz Hugo Hadwiger \cite{finsler_1937} udowodnią to twierdzenie w 1937 roku.
\index[persons]{Finsler, Paul}%
\index[persons]{Hadwiger, Hugo}%

\begin{proposition}[nierówność Neuberga-Pedoego]
    W dwóch trójkątach o bokach $a, b, c$ oraz polu powierzchni $S$ i bokach $a_*, b_*, c_*$ oraz polu powierzchni $S_*$ zachodzi nierówność
    \begin{equation}
        a^2_*(b^2 + c^2 - a^2) + 
        b^2_*(a^2 + c^2 - b^2) + 
        c^2_*(a^2 + b^2 - c^2) \ge 16 SS_*
    \end{equation}
    z równością wtedy i tylko wtedy, gdy trójkąty są podobne: $a/a_* = b/b_* = c/c_*$.
\end{proposition}

(Aby dostać nierówność Weitzenböcka, wystarczy wziąć $a = b = c = 1$).
Daniel Pedoe \cite{pedoe_1941} odkryje nierówność w 1941, a później dowie się, że była znana luksemburskiemu geometrze Józefowi Neubergowi.
\index[persons]{Pedoe, Daniel}%
\index[persons]{Neuberg, Joseph}%
Ten jednak nie pokaże, że z równości dwóch strony wynika podobieństwo trójkątów.
Jeden z dowodów wykorzystuje wzór Herona oraz nierówność Cauchy'ego-Schwarza.

%