\begin{proposition}[nierówność izoperymetryczna]
	Dany jest trójkąt o połowie obwodu $p$ oraz polu $S$.
	Wtedy 
	\begin{equation}
		S \le \frac{p^2}{3 \sqrt 3},
	\end{equation}
	zatem wśród trójkątów o ustalonym obwodzie największe pole ma trójkąt równoboczny.
	\index{nierówność!izoperymetryczna}
	% TODO: https://www.deltami.edu.pl/2011/04/symetryzacja-steinera/
\end{proposition}

https://www.deltami.edu.pl/2023/01/nierownosc-izoperymetryczna/

Guzicki \cite[s. 169, 170]{guzicki_2021} wyprowadza nierówność izoperymetryczną ze wzoru Herona oraz nierówności między średnią arytmetyczną i~geometryczną.
\index{wzór!Herona}%
Odpowiednik tego w trzech wymiarach znany będzie jako hipoteza (Zdzisława Aleksandra) Melzaka: podana około roku 1965, będzie musiała długo czekać na swoje rozwiązanie.
\index[persons]{Melzak, Zdzisław Aleksander}%
% trójwymiarowy odpowiednik to hipoteza: https://math.stackexchange.com/questions/4044670/what-is-the-largest-volume-of-a-polyhedron-whose-skeleton-has-total-length-1-is