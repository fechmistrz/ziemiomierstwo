%

\begin{proposition}[twierdzenie Napoleona]
    Dany jest trójkąt, na bokach którego zbudowano na zewnątrz trzy trójkąty równoboczne.
    Ich środki są wierzchołkami czwartego trójkąta równobocznego.
\end{proposition}

Napisze o tym Guzicki \cite[s. 307]{guzicki_2021}.
\index[persons]{Bonaparte, Napoleon}%
Chociaż twierdzenie przypisuje się Napoleonowi Bonapartemu, prawdziwym odkrywcą mógł być Lorenzo Mascheroni albo William Rutherford, matematyk angielski, który zasłynął z obliczenia 208 cyfr $\pi$ w 1841 roku (tylko 152 cyfry były poprawne).

% TODO: https://www.deltami.edu.pl/2009/05/liczby-zespolone-w-geometrii/

https://www.deltami.edu.pl/2015/12/o-obrotach/

https://www.deltami.edu.pl/2025/07/napoleon-thebault-barlotti-i-wielokaty-foremne/

\todofoot{https://www.deltami.edu.pl/2015/12/od-kwadratu/}


%