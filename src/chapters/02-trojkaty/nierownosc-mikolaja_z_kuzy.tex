%

\index{nierówność!Mikołaja z Kuzy|(}

\begin{proposition}[nierówność Mikołaja z Kuzy]
\begin{equation}
	\label{nierownosc_mikolaja_z_kuzy}
	\frac{\sin x}{x} < \frac{2 + \cos x}{3}.
\end{equation}
\end{proposition}

Mikołaj z Kuzy (znany też jako Kuzańczyk) w wielu pismach zajmie się problemem kwadratury koła i obliczania obwodu koła.
\index[persons]{Mikołaj z Kuzy}%
Jego przybliżenie $\pi \approx 6\sqrt{48/7}/5 = 3.1423...$ (z \emph{De Geometricis Transmutationibus} z 1445 roku) będzie zgodne z~oszacowaniem $223/71 < \pi < 22/7$ pochodzącym jeszcze od Archimedesa.
Niestety, późniejsze pomysły będą mniej trafione; 
Regiomontanus omówi po kolei wszystkie konstrukcje Mikołaja i~przejdzie do liczb (,,\emph{nunc ad numeros descendendum}'').
Policzy na przykład, że Mikołaj w \emph{Quadratura Circuli} z 1450 roku sugeruje $\pi = 3.154\ldots > 22/7$. 

\index{nierówność!Mikołaja z Kuzy|)}

https://www.deltami.edu.pl/2020/03/o-nierownosci-miedzy-srednia-arytmetyczna-sinusa-i-tangensa-kata-ostrego-a-jego-miara/

%