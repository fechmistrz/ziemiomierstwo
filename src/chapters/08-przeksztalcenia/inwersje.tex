%

\section{Inwersja względem okręgu}

Inwersję odkryje wiele osób niezależnie od siebie, na przykład Jakob Steiner (1824), Lambert Adolphe Jacques Quetelet (1825), Giusto Bellavitis (1836), John William Stubbs i John Kells Ingram (1842?) albo lord Kelvin (właściwie William Thomson 1845).

\todofoot{https://www.deltami.edu.pl/2013/05/w-krzywym-zwierciadle/}

\todofoot{https://www.deltami.edu.pl/2014/07/symetria-wzgledem-okregu/}

\begin{definition}
    Dany jest okrąg $\Gamma$ o środku $O$ i promeniu $r$ oraz punkt $P$, różny od $O$:
    \begin{figure}
    % TODO: rysunek tutaj
    \end{figure}
    Wtedy jedyny punkt $P'$ leżący na półprostej $OP$ taki, że $|OP| \cdot |OP'| = r^2$ nazywamy inwersją punktu $P$ względem okręgu $\Gamma$.
\end{definition}

Czasami do płaszczyzny dokłada się $\infty$, punkt w nieskończoności: przyjmujemy, że inwersja odwzorowuje $\infty$ na punkt $O$, zaś punkt $O$ na $\infty$; czyni to ją automorfizmem płaszczyzny, który przenosi wnętrze okręgu na jego zewnętrze.
O inwersji piszą Coxeter \cite[s. 93-101, 107-112]{coxeter_1967}.

Cyrkiel i linijka wystarczają do znalezienia inwersji dowolnego punktu.
Bez straty ogólności załóżmy, że $P$ leży na zewnątrz okręgu.
Wtedy jego obraz znajduje się na przecięciu prostej $OP$ oraz cięciwy, która łączy punkty styczności stycznych do okręgu poprowadzonych z punktu $P$. % Eves 122
Surajit Dutta około 2014 roku uprze się, by nie rozpatrywać przypadków ,,punkt leży wewnątrz okręgu'' oraz ,,punkt leży na zewnątrz okręgu'' osobno i poda swój przepis.
% Dutta, Surajit (2014) A simple property of isosceles triangles with applications Archived 2018-04-21 at the Wayback Machine, Forum Geometricorum 14: 237–240 
\index[persons]{Dutta, Surajit}%

Coxeter \cite[s. 98]{coxeter_1967} wspomina dwa mechaniczne przyrządy kreślące obraz inwersyjny trajektorii: inwersor Peaucelliera, a właściwie Peaucelliera-Lipkina, a chwilę poźniej także urządzenie Harta\footnote{Po angielsku odpowiednio Peaucellier's cell, Harts' linkage.} służące do tego samego celu. % EVES 135

\begin{proposition}
    Inwersje są kątowierne: każdy kąt ma taką samą miarę jak jego obraz względem inwersji.
\end{proposition}

\begin{proposition}
    Inwersje przenoszą okręgi ortogonalne na okręgi ortogonalne, włączając proste jako przypadki szczególne.
\end{proposition}

To, że inwersje przenoszą proste oraz okręgi na proste oraz okręgi będzie wiedzieć Robert Simson w 1749 roku, kiedy odtworzy zagubioną pracę ,,Plane Loci'' Apoloniusza na podstawie komentarza Pappusa.
\index[persons]{Simson, Robert}%
\index[persons]{Apoloniusz}%
\index[persons]{Pappus}%

Wiele trudnych problemów geometrycznych staje się prostsze po zastosowaniu inwersji.
Na przykład środek okręgu przed i po odwróceniu są współliniowe ze środkiem inwersji, skąd wynika, że prosta Eulera trójkąta między punktami styczności okręgów dopisanych przechodzi przez pewne ważne punkty.
% https://en.wikipedia.org/wiki/Inversive_geometry#Application

Proste prostopadłe do $PP'$, które przechodzą przez jeden z punktów, są biegunowymi drugiego końca odcinka (bieguna).
Bieguny i biegunowe mają swoje ciekawe własności.

% Coxeter s. 77: Magnus 1831 wymyślił ten termin % Ludwig Immanuel Magnus
% % https://en.wikipedia.org/wiki/Sacred_Mathematics

Istnieje odpowiednik klasyfikacji \ref{podobienstwa_klasyfikacja}:

\begin{proposition}
    Każde przekształcenie płaszczyzny inwersyjnej zachowujące okręgi jest albo podobieństwem, albo złożeniem inwersji i izometrii.
\end{proposition}

Stąd wynika, że każde takie przekształcenie jest złożeniem co najwyżej czterech inwersji.
Coxeter \cite[s. 108]{coxeter_1967} pisze o trzech rodzajach przekształceń: eliptycznych, parabolicznych i hiperbolicznych, co sprowadza się w szczególnym przypadku do obrotów, translacji i dylatacji.

\begin{proposition}
    Złożenie czterech inwersji, które nie jest złożeniem dwóch inwersji, nazywamy przekształceniem loksodromicznym.
\end{proposition}

Nazwa ma greckie korzenie: loksodroma to linia przecinająca wszystkie południki pod tym samym kątem (λοξός ukośny, δρόμος linia).

%