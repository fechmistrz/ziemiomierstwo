%

\subsection{Klasyfikacja izometrii}

\begin{theorem}[Chasles'a, 1830]
\index{twierdzenie!Chasles'a}%
\label{theorem_chasles}
    Każda izometria płaszczyzny jest złożeniem co najwyżej trzech symetrii osiowych.
\end{theorem}

\todofoot{https://www.deltami.edu.pl/2012/09/wszystko-moze-sie-przydac/}
\todofoot{https://www.deltami.edu.pl/2015/11/maly-wybor-i-dobrze/}
\todofoot{https://www.deltami.edu.pl/2015/11/rzut-butem-czyli-twierdzenie-chaslesa/}
\todofoot{https://www.deltami.edu.pl/2017/01/uczniowie/}

Piszą o nim Eves \cite[s. 113]{eves1_1972}; Audin \cite[s. 49]{audin_2003} (w ogólniejszej wersji dla przestrzeni $\mathbb R^n$ i $n+1$ odbić).
Poza symetriami osiowymi oraz złożeniami dwóch takich, pozostał przypadek trzech symetrii.
% TODO: https://en.wikipedia.org/wiki/Cartan–Dieudonné_theorem

\begin{definition}[symetria z poślizgiem]
    Niech $p$ będzie prostą, zaś $w$ wektorem równoległym do niej.
    Złożenie $T_w \circ S_p$ nazywamy symetrią osiową z poślizgiem.
\end{definition}

Wektor $w$ nie musi być równoległy do prostej $p$, ale jeśli założymy, że tak jest, to nie musimy pamiętać kolejności składania, bo mamy równość $T_w \circ S_p = S_p \circ T_w$.

\begin{corollary}
    Każda izometria płaszczyzny jest przesunięciem, obrotem albo symetrią z poślizgiem, przy czym orientację zachowują dokładnie pierwsze dwa typy izometrii.
\end{corollary}

Klasyfikacja izometrii płaszczyzny eliminuje paradoks Banacha-Tarskiego z płaszczyzny, bo nie ma podgrupy wolnej w grupie jej izometrii.
Oto szczegóły.
Jeśli $f$ jest izometrią, to $f^2$ jest przesunięciem albo obrotem.
Gdy $g$ też jest izometrią, to $f^2g^2f^{-2}g^{-2}$ jest zawsze przesunięciem; dla dowodu tego należy rozpatrzeć cztery przypadki, z których najtrudniejszy to para obrót-obrót.
Pomocne może być zauważenie, że jeśli mamy dwa obroty $f$, $g$ o różnych środkach i ich złożenie jest obrotem, to $fg$ i $gf$ są obrotami o ten sam kąt, ale różnych środkach.
A więc 
\begin{align}
    \ldots & = f^2 g^2 f^{-2} g^{-2} f^{-2} g^2 f^4 g^{-2} f^{-2} g^2 f^{-2} g^{-2} f^2 \label{slowo_banacha} \\
    & = (f^2 g^2 f^{-2} g^{-2})( f^{-2} g^2 f^2 g^{-2}) (g^2 f^2g^{-2} f^{-2} ) (g^2 f^{-2} g^{-2} f^2) \\
    & = T_v T_w T_{-v} T_{-w} \\
    & = \operatorname{Id},
\end{align}
co kończy dowód.
(Po drodze skorzystaliśmy z faktów \ref{delta_1997_9_start} do \ref{delta_1997_9_end}, twierdzenia \ref{theorem_chasles}, że złożenie przesunięć jest przesunięciem oraz faktu \ref{for_banach_11}).
Napis \ref{slowo_banacha} to tak zwane słowem Banacha.

Istnieje podobna klasyfikacja izometrii przestrzeni $\mathbb R^3$, jak u Coxetera \cite[s. 113-122]{coxeter_1967}, Evesa \cite[s. 148]{eves1_1972}:

\begin{proposition}
    Każda izometria przestrzeni $\mathbb R^3$
    zachowuje orientację (i jest odwzorowaniem tożsamościowym; przesunięciem; obrotem wokół prostej albo ruchem śrubowym, czyli obrotem wokół prostej z przesunięciem)
    albo nie (i jest odbiciem względem płaszczyzny; odbiciem z poślizgiem; obrotem niewłaściwym, czyli obrotem wokół prostej z odbiciem w płaszczyźnie prostopadłej do niej).
\end{proposition}

%