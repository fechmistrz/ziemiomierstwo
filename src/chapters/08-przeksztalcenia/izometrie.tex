%

\section{Izometrie}

\todofoot{Izometrie Coxetera, s. 29-36, 47 -- Hjelmslev}

\begin{definition}
    Odwzorowanie płaszczyzny w siebie, które zachowuje długość każdego odcinka, nazywamy izometrią płaszczyzny albo krócej izometrią.
\end{definition}

Izometrie to odwzorowania, które zachowują odległości.
Izometrie można składać, każda jest bijekcją i funkcja odwrotna do niej też jest izometrią.
Patrz do Coxetera \cite[s. 45-52, 56-63]{coxeter_1967}.

\begin{proposition}
\label{delta_1997_9_start}%
    Izometria, która ma punkt stały, przenosi wszystkie okręgi wokół tego punktu na siebie.
\end{proposition}

\begin{proposition}
    Izometria, która ma dwa punkty stałe, ma ich nieskończenie wiele (wszystkie punkty prostej przechodzącej przez dwa punkty).
\end{proposition}

\begin{proposition}
    Izometria, która ma trzy niewspółliniowe punkty stałe, jest odwzorowaniem tożsamościowym i~wszystkie punkty są jej punktami stałymi.
\end{proposition}

\begin{proposition}
\label{delta_1997_9_end}%
    Izometria, której wartości w trzech niewspółliniowych punktach są znane, jest jednoznacznie wyznaczona przez nie.
\end{proposition}

Cztery powyższe fakty to początek przepisu, jaki Marek Kordos poda w $\Delta_{97}^9$ na wynik związany z paradoksalnym rozkładem kuli.

Mamy też:

\begin{theorem}[Hjelmsleva]
    Jeżeli wszystkie punkty $P$ prostej przekształcają się izometrycznie na wszystkie punkty $P'$ innej prostej, to wszystkie środki odcinków $PP'$ są różne i współliniowe, albo wszystkie sprowadzają się do jednego punktu.
\end{theorem}

% TODO: https://www.deltami.edu.pl/2013/06/slowa-slowa-slowa/ => słowo Banacha, Chasles

% https://deltami.edu.pl/media/issues/1984/08/delta-1984-08.pdf ?

Patrz też do Coxetera \cite[s. 63, 64]{coxeter_1967}.


%

\subsection{Translacje}

\begin{definition}[translacja]
    Niech $w = (w_x, w_y)$ będzie wektorem.
    Funkcję $T_w$ zadaną przez
    \begin{equation}
        T_w(x, y) = (x + w_x, y + w_y)
    \end{equation}
    nazywamy translacją o wektor $w$.
\end{definition}

Translacje to inaczej przesunięcia równoległe.
Rodzina wszystkich translacji tworzy grupę, przesunięcie o wektor zerowy to odwzorowanie tożsamościowe.

%

%

\subsection{Symetrie osiowe}

\begin{definition}[symetria osiowa]
    Niech $k$ będzie ustaloną prostą, $X$ punktem, zaś $X'$ jego rzutem na prostą $k$.
    Oznaczmy przez $w$ wektor $2X X'$.
    Wtedy funkcję $S_k$ zadaną przez
    \begin{equation}
        S_k(X) = T_{w}(X)
    \end{equation}
    nazywamy symetrią osiową o osi $k$.
\end{definition}

(Chodzi o to, by rzut punktu $X$ na prostą $k$ był środkiem odcinka łączącego argument oraz wartość funkcji $S_k$).
Symetrie osiowe nie mają estetycznego wzoru, jeśli nie chcemy wprowadzać do książki macierzy: punkt $(x_0, y_0)$ odbity względem osi o równaniu $ax + by + c = 0$ przechodzi na
\begin{equation}
    \left(
    x_0 - 2a \cdot \frac{ax_0 + by_0 + c}{a^2 + b^2},
    y_0 - 2b \cdot \frac{ax_0 + by_0 + c}{a^2 + b^2}
    \right),
\end{equation}

\begin{proposition}
\label{kordos_banach_6}%
    Złożenie dwóch symetrii osiowych, których osie są równoległe, jest translacją (o wektor dwa razy dłuższy niż odległość między osiami, prostopadły do obydwu).
\end{proposition}

\begin{proposition}
\label{klkm1}%
    Niech $k, l$ będą prostymi.
    Wtedy $S_k \circ S_l \circ S_k = S_m$, gdzie $m$ to prosta będąca odbiciem prostej $l$ względem $k$.
\end{proposition}

Jeśli figura ma dokładnie dwie osi symetri, to muszą być prostopadłe.

%

%

\subsection{Symetrie środkowe}

\begin{definition}[symetria środkowa]
    Niech $A = (x_A, y_A)$ będzie ustalonym punktem.
    Funkcję $S_A$ zadaną następująco:
    \begin{equation}
        S_A(x_0, y_0) = (2x_A - x_0, 2y_A - y_0)
    \end{equation}
    nazywamy symetrią środkową o środku $A$.
\end{definition}

(Chodzi o to, by punkt $A$ był środkiem odcinka łączącego argument oraz wartość funkcji $S_A$).

\begin{example}
    Symetrie osiowe oraz środkowe to przykłady inwolucji.
\end{example}

\begin{proposition}
    Złożenie dwóch symetrii środkowych jest translacją o wektor dwa razy dłuższy od tego, który łączy ich środki.
\end{proposition}

Symetrie środkowe można emulować dwoma symetriami osiowymi:

\begin{proposition}
\label{two_axial_one_point}
    Niech $k, l$ będą dwiema prostymi prostopadłymi do siebie, które przechodzą przez punkt $A$.
    Wtedy $S_k \circ S_l = S_l \circ S_k = S_A$.
\end{proposition}

Równoważnie: jeżeli punkt $A$ leży na prostej $k$, to $S_A \circ S_k = S_k \circ S_A = S_l$, gdzie prosta $l$ jest prostopadła do prostej $k$ i przechodzi przez $A$.
% Ćwiczenie 4.26, Neugebauer s. 203

\begin{proposition}
    Złożenie trzech symetrii środkowych jest symetrią środkową: $S_A \circ S_B \circ S_C = S_D$, gdzie punkt $D$ jest przesunięciem punktu $A$ o wektor $BC$.
\end{proposition}

To jest, po krótkim zastanowieniu, treść twierdzenia Varignona.
\index{twierdzenie!Varignona}%
Mamy odpowiednik faktu \ref{klkm1}:

\begin{proposition}
\label{klkm2}
    Niech $K, L$ będą punktami.
    Wtedy $S_K \circ S_L \circ S_K = S_M$, gdzie $M$ to punkt będący odbiciem punktu $L$ względem $K$.
\end{proposition}

Jeśli figura ma dwa środki symetrii, to ma ich nieskończenie wiele.

%

%

\subsection{Obroty}

\begin{definition}[obrót]
    Niech $A$ będziez punktem, zaś $\alpha$ miarą kąta.
    Funkcję $R_A^\alpha$ zadaną następująco: $R_A^\alpha(X) = Y$ wtedy i tylko wtedy, gdy kąt $\angle XAY$ ma miarę $\alpha$ i $|AY| = |AX|$ nazywamy obrotem o kąt $\alpha$ wokół punktu $A$ zwanego środkiem obrotu.
\end{definition}

Oczywiście i tutaj można podać wzór.
Wyznacza się go najpierw dla obrotów wokół zera, by następnie zauważyć, że dowolny obrót sprowadza się do tamtego przy użyciu dwóch translacji (jednej, by przenieść się na początek i drugiej, by wrócić):

\begin{equation}
    R_A^\alpha(x,y) = \begin{pmatrix}
        x_A + (x_0 - x_A) \cos \alpha - (y_0 - y_A) \sin \alpha\\
        y_A + (x_0 - x_A) \sin \alpha - (y_0 - y_A) \cos \alpha
    \end{pmatrix}.
\end{equation}

Symetrie środkowe nazywa się (rzadko) półobrotem, czyli obrotem o kąt $\pi$; robi tak Eves \cite[s. 105]{eves1_1972}.

\begin{proposition}
    Złożenie dwóch symetrii osiowych, których osie przecinają się pod kątem $\alpha$ jest obrotem o kąt $2 \alpha$ wokół punktu przecięcia wspomnianych osi.
\end{proposition}

To uogólnienie faktu \ref{two_axial_one_point}, tam było $\alpha = \pi$, a półobrót (obrót o kąt $\pi$) jest tym samym, co symetria środkowa.

\begin{proposition}
\label{for_banach_11}%
    Niech $A \neq B$ będą dwoma punktami, zaś $\alpha, \beta$ dwiema miarami kątów.
    Jeśli $\alpha + \beta$ nie jest wielokrotnością kąta pełnego, $2 \pi$, to $R_B^\beta \circ R_A^\alpha = R_C^{\alpha + \beta}$, gdzie $C$ jest takim punktem, by trójkąt $\triangle ABC$ miał kąty $\alpha/2$ przy $A$, $\beta / 2$ przy $B$.

    W przeciwnym przypadku złożenie tych dwóch obrotów jest translacją.
\end{proposition}

%

%

\subsection{Klasyfikacja izometrii}

\begin{theorem}[Chasles'a, 1830]
\index{twierdzenie!Chasles'a}%
\label{theorem_chasles}
    Każda izometria płaszczyzny jest złożeniem co najwyżej trzech symetrii osiowych.
\end{theorem}

\todofoot{https://www.deltami.edu.pl/2012/09/wszystko-moze-sie-przydac/}
\todofoot{https://www.deltami.edu.pl/2015/11/maly-wybor-i-dobrze/}
\todofoot{https://www.deltami.edu.pl/2015/11/rzut-butem-czyli-twierdzenie-chaslesa/}
\todofoot{https://www.deltami.edu.pl/2017/01/uczniowie/}

Piszą o nim Eves \cite[s. 113]{eves1_1972}; Audin \cite[s. 49]{audin_2003} (w ogólniejszej wersji dla przestrzeni $\mathbb R^n$ i $n+1$ odbić).
Poza symetriami osiowymi oraz złożeniami dwóch takich, pozostał przypadek trzech symetrii.
% TODO: https://en.wikipedia.org/wiki/Cartan–Dieudonné_theorem

\begin{definition}[symetria z poślizgiem]
    Niech $p$ będzie prostą, zaś $w$ wektorem równoległym do niej.
    Złożenie $T_w \circ S_p$ nazywamy symetrią osiową z poślizgiem.
\end{definition}

Wektor $w$ nie musi być równoległy do prostej $p$, ale jeśli założymy, że tak jest, to nie musimy pamiętać kolejności składania, bo mamy równość $T_w \circ S_p = S_p \circ T_w$.

\begin{corollary}
    Każda izometria płaszczyzny jest przesunięciem, obrotem albo symetrią z poślizgiem, przy czym orientację zachowują dokładnie pierwsze dwa typy izometrii.
\end{corollary}

Klasyfikacja izometrii płaszczyzny eliminuje paradoks Banacha-Tarskiego z płaszczyzny, bo nie ma podgrupy wolnej w grupie jej izometrii.
Oto szczegóły.
Jeśli $f$ jest izometrią, to $f^2$ jest przesunięciem albo obrotem.
Gdy $g$ też jest izometrią, to $f^2g^2f^{-2}g^{-2}$ jest zawsze przesunięciem; dla dowodu tego należy rozpatrzeć cztery przypadki, z których najtrudniejszy to para obrót-obrót.
Pomocne może być zauważenie, że jeśli mamy dwa obroty $f$, $g$ o różnych środkach i ich złożenie jest obrotem, to $fg$ i $gf$ są obrotami o ten sam kąt, ale różnych środkach.
A więc 
\begin{align}
    \ldots & = f^2 g^2 f^{-2} g^{-2} f^{-2} g^2 f^4 g^{-2} f^{-2} g^2 f^{-2} g^{-2} f^2 \label{slowo_banacha} \\
    & = (f^2 g^2 f^{-2} g^{-2})( f^{-2} g^2 f^2 g^{-2}) (g^2 f^2g^{-2} f^{-2} ) (g^2 f^{-2} g^{-2} f^2) \\
    & = T_v T_w T_{-v} T_{-w} \\
    & = \operatorname{Id},
\end{align}
co kończy dowód.
(Po drodze skorzystaliśmy z faktów \ref{delta_1997_9_start} do \ref{delta_1997_9_end}, twierdzenia \ref{theorem_chasles}, że złożenie przesunięć jest przesunięciem oraz faktu \ref{for_banach_11}).
Napis \ref{slowo_banacha} to tak zwane słowem Banacha.

Istnieje podobna klasyfikacja izometrii przestrzeni $\mathbb R^3$, jak u Coxetera \cite[s. 113-122]{coxeter_1967}, Evesa \cite[s. 148]{eves1_1972}:

\begin{proposition}
    Każda izometria przestrzeni $\mathbb R^3$
    zachowuje orientację (i jest odwzorowaniem tożsamościowym; przesunięciem; obrotem wokół prostej albo ruchem śrubowym, czyli obrotem wokół prostej z przesunięciem)
    albo nie (i jest odbiciem względem płaszczyzny; odbiciem z poślizgiem; obrotem niewłaściwym, czyli obrotem wokół prostej z odbiciem w płaszczyźnie prostopadłej do niej).
\end{proposition}

%

%