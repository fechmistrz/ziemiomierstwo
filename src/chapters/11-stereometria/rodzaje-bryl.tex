%

\section{Rodzaje brył}

Panuje ogromny bałagan, jeśli chodzi o nazewnictwo wielościanów.
Spróbujemy więc zaprowadzić porządek.

\begin{definition}
    Wielościan nazywamy przechodnim wierzchołkowo (odpowiednio: krawędziowo, ściennie, flagowo), jeśli działanie grupy symetrii tego wielościanu na zbiorze jego wierzchołków (odpowiednio: krawędzi, ścian, flag) jest przechodnie.
\end{definition}

\begin{definition}
    Wielościany platońskie to te, które są flagowo  (lub równoważnie, wierzchołkowo, krawędziowo i ściennie) przechodnie, zaś ich ściany są wielokątami foremnymi.
\end{definition}

\begin{definition}
    Wielościany Keplera-Poinsota (gwiaździste foremne) to te, które są flagowo przechodnie, zaś ich ściany są wielokątami gwiaździstymi foremnymi.
\end{definition}

\begin{definition}
    Wielościany archimedesowe to te, które są wierzchołkowo przechodnie, ale nie ściennie przechodnie i wszystkie ich ściany są wielokątami foremnymi.
\end{definition}

Wielościany archimedesowe oraz dwie serie: graniastosłupów i antygraniastosłupów nazywa się czasami półforemnymi.

\begin{definition}
    Wielościany Catalana to te, które są dualne do wielościanów archimedesowych.
\end{definition}

\begin{definition}
    Wielościany jednorodne to te, które są wierzchołkowo przechodnie i mają foremne ściany.
    Dzielą się na foremne (ściennie i krawędziowo przechodnie), kwasiforemne (krawędziowo, ale nie ściennie przechodnie) i semiforemne (ani krawędziowo, ani ściennie przechodnie).
\end{definition}

\begin{definition}
    Wielościany Johnsona-Zalgallera to te, które są wypukłe, mają foremne ściany i nie są jednorodne.
\end{definition}


% The elongated square gyrobicupola or pseudo­rhombi­cub­octa­hedron is an extra polyhedron with regular faces and congruent vertices. Still, it is not generally counted as an Archimedean solid because it is not vertex-transitive.



%

\subsection{Czworościany} % tetrahedron

W każdym czworościanie istnieje wierzchołek, przy którym trzy kąty płąskie są ostre. % https://www.deltami.edu.pl/2013/12/katy-trojscienne/

\begin{proposition}
    Niech $ABCD$ będzie czworościanem.
    Wtedy następujące warunki są równoważne:
    \begin{enumerate}
        \item Wszystkie ściany są przystające.
        \item Wszystkie ściany są trójkątami ostrokątnymi o takim samym promieniu okręgu opisanego.
        \item Suma kątów płaskich przy każdym wierzchołku wynosi $\pi$.
        \item Sumy kątów płaskich przy trzech dowolnych wierzchołkach wynosi $\pi$.
        \item Siatka czworościanu jest trójkątem ostrokątnym podzielonym na cztery przystające trójkąty.
        \item Kąty $\angle BAC$, $\angle ABD$, $\angle ACD$, $\angle BDC$ są równe.
        \item Przeciwległe krawędzie są równe.
        \item Trzy odcinki łączące środki przeciwległych krawędzi są parami prostopadłe.
        \item Wszystkie ściany mają równe pola.
        \item Rzut czworościanu $ABCD$ na dowolną płaszczyznę równoległą do dwóch przeciwległych krawędzi jest prostokątem.
        \item Każdy odcinek łączący środki przeciwległych krawędzi jest prostopadły do tych krawędzi.
    \end{enumerate} 
\end{proposition}

Dowód wszystkich 11 implikacji znajdziemy w artykulu Pompego, $\Delta_{94}^3$. % https://www.deltami.edu.pl/1994/03/o-czworoscianie-rownosciennym/

Czworościan foremny ma trzy osie symetii: to proste przechodzące przez środki przeciwległych krawędzi. % https://www.deltami.edu.pl/1993/04/rozne-roznosci/

% % 11.2.1 czworościany
\subsection{Ostrosłupy} % pyramid

\begin{definition}[ostrosłup]
    A pyramid is a polyhedron that may be formed by connecting each vertex in a planar polygon to a point lying outside that plane. This point is called the pyramid's apex, and the planar polygon is the pyramid's base. Each other face of the pyramid is a triangle[1] consisting of one of the base's edges, and the two edges connecting that edge's endpoints to the apex. These faces are called the pyramid's lateral faces, and each edge connected to the apex is called a lateral edge.[2] 
\end{definition}

The terms "right pyramid" and "regular pyramid" are used to describe special cases of pyramids. Their common notions are as follows. A regular pyramid is one with a regular polygon as its base. A right pyramid is one where the axis (the line joining the centroid of the base and the apex) is perpendicular to the base.[6][7][8] An oblique pyramid is one where the axis is not perpendicular to the base.[9] However, there are no standard definitions for these terms, and different sources use them somewhat differently.

The truncated pyramid is a pyramid cut off by a plane; if the truncation plane is parallel to the base of a pyramid, it is called a frustum.

Jeżeli spodek wysokości ostrosłupa pokrywa się ze środkiem okręgu opisanego na jego podstawie, to taki ostrosłup nazywamy ostrosłupem prostym. J


Ostrosłupy są szczególnymi przypadkami pryzmatoidów:

\begin{definition}
A prismatoid is defined as a polyhedron where its vertices lie on two parallel planes, with its lateral faces as triangles, trapezoids, and parallelograms.[4] \end{definition}

Ostrosłup ścięty – bryła powstała w wyniku przecięcia ostrosłupa płaszczyzną równoległą do podstawy ostrosłupa i odrzucenia punktów leżących po stronie jego wierzchołka[1].
PO ANGIELSKU JEST INACZEJ
The truncated pyramid is a pyramid cut off by a plane; if the truncation plane is parallel to the base of a pyramid, it is called a frustum.



Eves \cite[s.4]{eves1_1972} napisze, że około 1850 lat przed Chrystusem znany będzie wzór na dokładną objętość ostrosłupa czworokątnego ściętego, o podstawie długości $a$, $b$ i wysokości $h$:
\begin{equation}
	V = \frac 1 3 h (a^2 + ab + b^2).
\end{equation}
(Po angielsku taką bryłę nazywa się \emph{frustum}, co w łacinie znaczy \emph{kęs, kawałek}).
Wzór pojawi się bez dowodu w papirusie moskiewskim, nazwanym tak, ponieważ jego pierwszym właścicielem spoza Egiptu będzie Władimir Goleniszczew, rosyjski egiptolog i kolekcjoner sztuki.
\index[persons]{Goleniszczew, Władimir}%






 % 11.2.2 ostrosłupy
%

\subsubsection{Pryzmatoidy} % prism? graniastosłup?
Pryzmatoid to bryła, której wierzchołki leżą na dwóch płaszczyznach równoległych.

\begin{example}[pyramids]
    Ostrosłup, gdzie jedna płaszczyzna zawiera tylko wierzchołek.
\end{example}

\begin{example}[wedges]
    (Klin?), gdzie jedna płaszczyzna zawiera tylko dwa wierzchołki.
\end{example}

\begin{example}[parallelepipeds]
    Quadrilateral-faced hexahedral prismatoids:
        Parallelepipeds – six parallelogram faces
        Rhombohedrons – six rhombus faces
        Trigonal trapezohedra – six congruent rhombus faces
        Cuboids – six rectangular faces
        Quadrilateral frusta – an apex-truncated square pyramid
        Cube – six square faces
\end{example}

\begin{example}[prisms]
    Graniastosłup, gdzie płaszczyzny są przystającymi wielokątami, zaś ściany boczne to równoległoboki.
\end{example}

\begin{example}[antiprisms]
    % Antiprisms, whose polygons in each plane are congruent and joined by an alternating strip of triangles;[5]
    % Star antiprisms;
\end{example}

\begin{example}[cupolae]
    % Cupolae, in which the polygon in one plane contains twice as many points as the other and is joined to it by alternating triangles and rectangles;
    % https://en.wikipedia.org/wiki/Cupola_(geometry)
\end{example}

\begin{example}[frusta]
    Frusta obtained by truncation of a pyramid or a cone;
    % https://en.wikipedia.org/wiki/Frustum
\end{example}

\begin{proposition}
    Objętość pryzmatoidu wyraża wzór
    \begin{equation}
        V = \frac 1 6 h (S_l + 4S_m + 4S_h),
    \end{equation}
    gdzie $h$ oznacza wysokość (odległość między płaszczyznami podstaw), $S_l$ pole podstawy dolnej, $S_h$ pole podstawy górnej, zaś $S_m$ to pole przekroju poprzecznego płaszczyzną, która leży w połowie między podstawami. % eves s. 40
\end{proposition}

\begin{proof}
    Wzór wynika z całkowania metodą Simsona, która daje dokładny wynik dla wielomianów stopnia co najwyżej trzeciego; pole przekroju poprzecznego stanowi funkcję kwadratową wysokości.
\end{proof}

% TODO https://en.wikipedia.org/wiki/Prismatoid#Prismatoid_families

% % 11.2.3 graniastoslupy
%

\subsection{Wielościany foremne} % regular

\begin{definition}
    Wielościan foremny to taki wielościan, którego grupa izometrii działa przechodnio na zbiorze flag.
\end{definition}

Niektóre osoby, przede wszystkim te nieznające definicji flagi, będą używać innych definicji, z~czego najpowszechniejsza będzie głosiła, że wielościan foremny ma mieć identyczne foremne ściany oraz identyczne naroża.

\begin{proposition}
    Istnieje czternaście wielościanów foremnych:
    \begin{itemize}
        \item pięć platońskich,
        \item cztery gwieździste (Keplera-Poinsota),
        \item pięć złożeń.
    \end{itemize}
\end{proposition}
Niektórzy nie uznają złożeń za wielościany foremne.

%

\subsection{Wielościany platońskie}

\todofoot{Coxeter 166-167.}
\todofoot{OCTAEDRON ELEVATUM}

Wszystkie ściany wielościanów platońskich są przystającymi wielokątami foremnymi, w każdym wierzchołku spotyka się taka sama liczba ścian.
Pozostaje zagadką, kto pierwszy odkryje wszystkie foremne wielościany wypukłe, ale Teajtet, uczeń Platona udowodni, że jest ich dokładnie pięć:
\index[persons]{Teajtet, uczeń Platona}%
\begin{itemize}
\item czworościan foremny,
\item sześciościan foremny (znany lepiej jako sześcian),
\item ośmiościan foremny,
\item dwunastościan foremny,
\item dwudziestościan foremny.
\end{itemize}

(Dwunastościan zostanie odkryty jako ostatni).
Niektórzy będą mylić wielościany foremne z platońskimi (!).

\begin{proposition}
    Największy sześcian, jaki można zmieścić w dwudziestościanie foremnym, wypełnia
    \begin{equation}
        \frac{219 \sqrt 5 + 15}{1331} \approx 38\,\%
    \end{equation}
    objętości.
\end{proposition}

Eksperymentalną drogę do tego wyniku opisze Michał Adamaszek w $\Delta_{20}^8$. % https://www.deltami.edu.pl/2020/08/wielosciany-w-wieloscianach-czyli-matematyka-eksperymentalna/

% https://en.wikipedia.org/wiki/Platonic_solid
% TODO: \section{Pięć wielościanów} Hartshorne: rozdział 8
% TODO: \section{Cauchy's rigidity theorem} Hartshorne: section 45

%

%

\subsection{Wielościany Keplera-Poinsota}

Definiując wielokąt (foremny) możemy opuścić warunek, by tworząca go łamana była zwyczajna.
Prowadzi to do szerszej klasy figur, nadających się na bycie ścianami wielościanów.
Wielokąty gwiaździste takie jak pentagram będą znane starożytnym, ale z jakiegoś powodu nie użyją ich tak, jak byśmy chcieli i jak można przeczytać w $\Delta_{74}^1$. % https://www.deltami.edu.pl/1974/01/wielosciany-gwiazdziste/

W 1619 roku Johannes Kepler odkryje dwa wielościany, które mają właśnie takie ściany; dwa następne zawdzięczymy Louisowi Poinsotowi w 1806 roku.
Dowód, że piątej takiej bryły już nie ma, poda Cauchy w 1813 roku; wykorzysta stellacje brył platońskich.
Stellacja to proces, podczas którego przedłuża się krawędzie albo ściany bryły tak długo, aż nie spotkają się ze sobą.
Bardziej eleganckie rozwiązanie znajdzie Joseph Bertrand przez cięcie faset dwunastościanu i dwudziestościanu.

W 1859 roku Arthur Cayley nada całej rodzinie nazwę ,,wielościany Keplera-Poinsota''.
Składają się na nią:
\begin{itemize}
\item dwunastościan gwiaździsty mały i wielki (znalezione przez Keplera),
\item dwunastościan i dwudziestościan wielki (znalezione przez Poinsota).
\end{itemize}

Dwunastościan gwiaździsty mały pojawia się na posadzce bazyliki św. Marka w Wenecji, datuje się go na XV wiek.
Wenzel Jamnitzer przedstawi dwunastościan gwiaździsty wielki oraz dwudziestościan wielki w \emph{Perspectiva corporum regularium}, księdze drzeworytów z 1568.
Większość, jeśli nie wszystkie wielościany Keplera-Poinsota, będą znane przed Keplerem.

%


\subsection{Złożenia}
Opiszemy teraz trójwymiarowe odpowiedniki heksagramu.
Wielościan złożony to taki, który składa się z kilku wielościanów o wspólnym środku.
Są wierzchołkowo, krawędziowo i ściennie przechodnie, ale w przeciwieństwie do wielościanów platońskich, poza złożeniem dwóch czworościanów nie (!) są flagowo przechodnie.

Ich lista obejmuje dwa czworościany, pięć czworościanów, dziesięć czworościanów, pięć sześcianów i pięć ośmiościanów.
Dwa czworościany znane są lepiej jako \emph{stella octangula}.
\index{stella octangula}

% https://en.wikipedia.org/wiki/Compound_of_four_cubes -> to nie jest regular


% % 11.2.4 foremne

\subsection{Wielościany archimedesowe}

Wszystkie naroża wielościanów archimedesowych są identyczne, ich ściany to wielokąty foremne (ale nie muszą mieć tyle samo boków), a one same są wypukłe.
Zostaną odnalezione przez Archimedesa i chociaż jego książka na ten temat zaginie, Pappus z Aleksandrii wspomni, że Archimedes znał wszystkie trzynaście.

Niektórzy (zaskakująco wiele osób) mieszają wielościany archimedesowe z półforemnymi, które oprócz wspomnianych trzynastu brył obejmują dwie nieskończone serie:
\label{dwie_serie}%
\begin{itemize}
\item póforemne graniastosłupy (o ścianach bocznych będących kwadratami)
\item półforemne antygraniastosłupy (o ścianach bocznych będących trójkątami równobocznymi).
\end{itemize} 

Branko Grünbaum uzna także $J_{37}$, wielościan Johnsona, za wielościan półforemny.
W każdym wierzchołku spotykają się trzy kwadraty i trójkąt, ale nie jest wierzchołkowo przechodni;  zapewne przez to pojawi się w druku dopiero w 1905 roku, a potem zostanie przez przypadek odkryty na nowo: przez Jeffreya Charlesa Percy'ego Millera w 1930 i niezależnie przez Ashkinuze.
% TODO: Ashkinuse, V.G.: On the number of semiregular polyhedra. -> rosyjski mathscinet

% % 11.2.5 archimedesowe
%

\subsection{Wielościany Catalana}

\begin{definition}[wielościan dualny]
    W ustalonym wielościanie usuńmy wierzchołki i krawędzie, zastąpmy każdą ścianę jej środkiem, po czym połączmy te punkty, które należały do sąsiednich ścian.
    Jeśli otrzymany obiekt jest wielościanem, to nazywamy go wielościanem dualnym.
\end{definition}

Bryły dualne do wielościanów platońskich są platońskie, a dualne do Keplera-Poinsota są bryłami Keplera-Poinsota, więc nie dają żadnych nowych obiektów.
Bryła dualna do bryły dualnej jest tego samego kształtu, co wyjściowa.

\begin{definition}[wielościan Catalana]
    Wielościan dualny do wielościanu archimedesowego nazywamy wielościanem Catalana.
\end{definition}

Pierwsze wielościany Catalana znajdzie Kepler, ale dopiero Eugene Catalan skompletuje je w 1865 roku. % Some Catalan solids were discovered by Johannes Kepler during his study of zonohedra
Nazwę sześćdziestościan deltoidowy zaproponuje profesor Roman Duda podczas tłumaczenia książki ,,Modele matematyczne'' Cundy'ego i Rolleta; brzmi to trochę dziwnie, ale przyjmie się.

\begin{proposition}
    Wielościany Catalana są ściennie przechodnie, ale nie wierzchołkowo przechodnie.
\end{proposition}

\begin{proposition}
    W wielościanach Catalana wszystkie kąty dwuścienne są równe. % Each Catalan solid has constant dihedral angles, meaning the angle between any two adjacent faces is the same.[1]
\end{proposition}



% % 11.2.6 catalana
%

\subsection{Wielościany Johnsona-Zalgallera}

\begin{definition}
    Wielościan wypukły, który nie jest jednorodny, ale wszystkie jego ściany są foremne, nazywamy wielościanem Johnsona albo Johnsona-Zalgallera.
\end{definition}

Norman Woodason Johnson \cite{johnson_1966} opublikuje w pracy doktorskiej z 1966 roku, a spisanej pod opieką samego Coxetera, listę 92 brył, które spełniają ten warunek, zaś Wiktor Abramowicz Zalgaller \cite{zalgaller_1969} udowodni po trzech latach (1969), że lista ta jest kompletna.
\index[persons]{Johnson, Norman}%
\index[persons]{Coxeter, ???}%
\index[persons]{Zalgaller, Wiktor}%
Często oznacza się je po prostu $J_{\ldots}$, gdzie indeks dolny mówi, którą pozycję na liście zajmuje.

\begin{compactitem}
    \item piramidy: \begin{compactitem}
        \item \textbf{piramida} (czworokątna: $J_1$, pięciokątna: $J_2$),
        \item wydłużona \textbf{piramida} (trójkątna: $J_7$, czworokątna: $J_8$, pięciokątna: $J_9$),
        \item skrętnie wydłużona \textbf{piramida} (czworokątna: $J_{10}$, pięciokątna $J_{11}$),
        \item \textbf{dwupiramida} (trójkątna: $J_{12}$, pięciokątna: $J_{13}$),
        \item wydłużona \textbf{dwupiramida} (trójkątna: $J_{14}$, czworokątna: $J_{15}$, pięciokątna: $J_{16}$),
        \item skrętnie wydłużona \textbf{dwupiramida} czworokątna: $J_{17}$,
    \end{compactitem}

    \item kopuły: \begin{compactitem}
        \item \textbf{kopuła} (trójkątna: $J_3$, czworokątna: $J_4$, pięciokątna: $J_5$),
        \item wydłużona \textbf{kopuła} (trójkątna: $J_{18}$, czworokątna: $J_{19}$, pięciokątna: $J_{20}$),
        \item skrętnie wydłużona \textbf{kopuła} (trójkątna: $J_{22}$, czworokątna: $J_{23}$, pięciokątna: $J_{24}$),
        \item podwójna \textbf{kopuła} (trójkątna: $J_{27}$, czworokątna: $J_{28}$, czworokątna skręcona: $J_{29}$,\\
        pięciokątna: $J_{30}$),
        \item \textbf{dwukopuła} pięciokątna skręcona: $J_{31}$,
        \item wydłużona \textbf{dwukopuła} (trójkątna: $J_{35}$, trójkątna skręcona: $J_{36}$, czworokątna skręcona: $J_{37}$, pięciokątna: $J_{38}$, pięciokątna skręcona: $J_{39}$),
        \item skrętnie wydłużona \textbf{dwukopuła} (trójkątna: $J_{44}$, czworokątna: $J_{45}$, pięciokątna: $J_{46}$),
    \end{compactitem}

    \item kopuło-rotundy: \begin{compactitem}
        \item \textbf{kopuło-rotunda} (pięciokątna: $J_{32}$, pięciokątna skręcona: $J_{33}$),
        \item wydłużona \textbf{kopuło-rotunda} (pięciokątna: $J_{40}$, pięciokątna skręcona: $J_{41}$),
        \item skrętnie wydłużona \textbf{kopuło-rotunda} pięciokątna: $J_{47}$,
    \end{compactitem}

    \item rotundy: \begin{compactitem}
        \item \textbf{rotunda} pięciokątna: $J_6$
        \item wydłużona \textbf{rotunda} pięciokątna: $J_{21}$,
        \item skrętnie wydłużona \textbf{rotunda} pięciokątna: $J_{25}$,
        \item \textbf{dwurotunda} pięciokątna: $J_{34}$,
        \item wydłużona \textbf{dwurotunda} (pięciokątna: $J_{42}$, pięciokątna skręcona: $J_{43}$),
        \item skrętnie wydłużona \textbf{dwurotunda} pięciokątna: $J_{48}$,
        \item dwusoczewkowa \textbf{rotunda} podwójna: $J_{91}$,
        \item \textbf{hebeklinorotunda} trójkątna: $J_{92}$,
    \end{compactitem}

    \item klino-... \begin{compactitem}
        \item \textbf{dwuklinoid} przycięty: $J_{84}$,
        \item \textbf{klinocingulum} podwójne: $J_{90}$,
    \end{compactitem}

    \item korony: \begin{compactitem}
        \item \textbf{klinokorona}: $J_{86}$,
        \item powiększona \textbf{klinokorona}: $J_{87}$,
        \item \textbf{klinomegakorona}: $J_{88}$, 
        \item \textbf{hebeklinomegakorona}: $J_{89}$,
    \end{compactitem}

    \item graniastosłupy: \begin{compactitem}
        \item podwójny \textbf{graniastosłup} trójkątny: $J_{26}$,
        \item powiększony \textbf{graniastosłup} (trójkątny: $J_{49}$, pięciokątny: $J_{52}$, sześciokątny: $J_{54}$),
        \item podwójnie powiększony \textbf{graniastosłup} (trójkątny: $J_{50}$, pięciokątny: $J_{53}$, \\
        sześciokątny: $J_{56}$),

        \item podwójnie osiowo powiększony \textbf{graniastosłup} sześciokątny: $J_{55}$,
        \item potrójnie powiększony \textbf{graniastosłup} (trójkątny: $J_{51}$, sześciokątny: $J_{57}$),
    \end{compactitem}

    \item \textbf{antygraniastosłup} czworokątny przycięty: $J_{85}$,

    \item powiększony \textbf{czworościan} ścięty $J_{65}$,

    \item sześciany: \begin{compactitem}
        \item powiększony \textbf{sześcian} ścięty $J_{66}$,
        \item podwójnie powiększony \textbf{sześcian} ścięty $J_{67}$,
    \end{compactitem}

    \item dwunastościany: \begin{compactitem}
        \item powiększony \textbf{dwunastościan} $J_{58}$,
        \item (podwójnie osiowo: $J_{59}$, podwójnie: $J_{60}$, potrójnie: $J_{61}$) powiększony \textbf{\textbf{dwunastościan}},
        \item powiększony \textbf{dwunastościan} ścięty $J_{68}$,
        \item (podwójnie osiowo:$J_{69}$, podwójnie: $J_{70}$, potrójnie: $J_{71}$) powiększony \textbf{dwunastościan} ścięty,
    \end{compactitem}

    \item dwudziesto-dwunastościany: \begin{compactitem}
        \item \textbf{dwudziesto-dwunastościan} rombowy (skręcony: $J_{72}$, podwójnie osiowo skręcony $J_{73}$, podwójnie skośnie skręcony $J_{74}$, potrójnie skręcony $J_{75}$, obcięty $J_{76}$, podwójnie osiowo obcięty $J_{80}$, podwójnie skośnie obcięty $J_{81}$, potrójnie obcięty $J_{83}$),
        \item przekręcony osiowo \textbf{dwudziesto-dwunastościan} rombowy obcięty: $J_{77}$,
        \item przekręcony skośnie \textbf{dwudziesto-dwunastościan} rombowy obcięty: $J_{78}$,
        \item podwójnie przekręcony \textbf{dwudziesto-dwunastościan} rombowy obcięty: $J_{79}$,
        \item przekręcony \textbf{dwudziesto-dwunastościan} rombowy podwójnie obcięty: $J_{82}$,
    \end{compactitem}

    \item dwudziestościany: \begin{compactitem}
        \item podwójnie obcięty \textbf{dwudziestościan}: $J_{62}$,
        \item potrójnie obcięty \textbf{dwudziestościan}: $J_{63}$,
        \item powiększony potrójnie obcięty \textbf{dwudziestościan}: $J_{64}$.
    \end{compactitem}
\end{compactitem}

% https://en.wikipedia.org/wiki/Johnson_solid
% http://matematyka.wroc.pl/book/wielosciany-johnsona
% {Siamese dodecahedron} https://en.wikipedia.org/wiki/Snub_disphenoid

% % 11.2.7 johnsona-zalgallera
%

\subsection{Wielościany jednorodne}

\begin{definition}
    Wierzchołkowo przechodni wielościan, którego ściany są foremne, nazywamy jednorodnym.
\end{definition}

Podobnież dzieli się je na foremne (jeśli są ściennie i krawędziowo przechodnie), kwazi-foremne (tylko krawędziowo przechodnie) i semiforemne (jeśli nie są ani takie, ani takie).
Znamy wszystkie wielościany jednorodne:
\begin{itemize}
\item 5 brył platońskich,
\item 13 wielościanów archimedesowych (2 kwazi-regularne, 11 semi-regularnych),
\item 4 wielościany Keplera-Poinsota,
\item 53 jednorodne wielościany gwiaździste (14 kwazi, 39 semi).
\end{itemize}

Oprócz tego istnieją dwie nieskończone serie wielościanów jednorodnych: graniastosłupy oraz antygraniastosłupy, tak jak na stronie \pageref{dwie_serie}.

% % 11.2.8 jednorodne, uniform

% TEGO NIE BĘDZIE: {Wielościany Goldberga}

\subsection{Wielościany Szilassiego}
O wielościanach Szilassiego pisze $\Delta_{24}^4$.

\subsection{Stożki}
Stożki.

\subsection{Walce}
Walce. % https://en.wikipedia.org/wiki/Cylinder

\subsection{Kule}
Kule. % https://en.wikipedia.org/wiki/On_the_Sphere_and_Cylinder pierwszy raz wzór objętość kuli

%