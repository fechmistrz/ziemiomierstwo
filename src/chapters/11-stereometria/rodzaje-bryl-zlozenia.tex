
\subsection{Złożenia}
Opiszemy teraz trójwymiarowe odpowiedniki heksagramu.
Wielościan złożony to taki, który składa się z kilku wielościanów o wspólnym środku.
Są wierzchołkowo, krawędziowo i ściennie przechodnie, ale w przeciwieństwie do wielościanów platońskich, poza złożeniem dwóch czworościanów nie (!) są flagowo przechodnie.

Ich lista obejmuje dwa czworościany, pięć czworościanów, dziesięć czworościanów, pięć sześcianów i pięć ośmiościanów.
Dwa czworościany znane są lepiej jako \emph{stella octangula}.
\index{stella octangula}

% https://en.wikipedia.org/wiki/Compound_of_four_cubes -> to nie jest regular
