%

\section{Zliczanie małych wielościanów}

https://www.deltami.edu.pl/2015/08/jakich-wieloscianow-nie-ma-a-jakie-sa/

\subsubsection{Deltościany}
%

Martyn Cundy zaproponuje przez podobieństwo greckiej litery $\Delta$ do trójkąta równobocznego, by rozważać pojęcie takie jak \emph{deltahedron} (nie mylić z \emph{deltohedron}, którego ściany są latawcami):
\index{Cundy, Henry Martyn}%

\begin{definition}[deltościan]
    Wielościan, którego wszystkie ściany są trójkątami równobocznymi, nazywamy deltościanem.
\end{definition}

\begin{proposition}
    Jest osiem deltościanów wypukłych: trzy bryły platońskie (o czterech, ośmiu albo dwudziestu ścianach) oraz pięć wielościanów Johnsona ($J_{12}$, $J_{13}$, $J_{17}$, $J_{51}$, $J_{84}$).
\end{proposition}

Pokażą to najpierw Otto Rausenberger \cite{rausenberger_1915} i niezależnie, choć dużo później Hans Freudenthal oraz Bartel Leendert van der Waerden \cite{freudenthal_1947} w obskurnym duńskim żurnalu.
% https://mathscinet.ams.org/mathscinet/relay-station?mr=0021687
\index[persons]{Freudenthal, Hans}%
\index[persons]{van der Warden, Bartel}%
Ich wynik zreferuje później Adam Gajda w $\Delta_{84}^{4}$ (bez wzmianki o Rausenbergerze).
Niewypukłych brył o tej własności jest nieskończnie wiele, mogą mieć dowolną parzystą liczbę ścian większą niż sześć.

https://www.deltami.edu.pl/2016/08/brzydka-prawda/

% % http://matematyka.wroc.pl/book/deltosciany

\subsubsection{Czworościany}
Ilejesttypówczworościanów? https://www.deltami.edu.pl/2011/01/ile-jest-typow-czworoscianow/

\subsubsection{Sześciościany}
Jest siedem sześciościanów wypukłych (i trzy niewypukłe).
Heinz Schumann oraz Bronisław Pabich napiszą krótki artykuł w $\Delta_{24}^{11}$, gdzie uzasadnią te liczby diagramami Schlegela.
\index{diagram Schlegela}%
(Wielościany wypukłe mają 6/0/0, 5/0/1, 4/2/0, 3/2/1, 2/4/0, 2/2/2, 0/6/0 ścian o trzech, czterech, pięciu krawędziach).
% TODO: DELTA 2001 luty Kordos
Oprócz tego są trzy typy niewypukłych sześciościanów, które powstają przez wycięcie czworościanu z czworościanu.
% Cvetković Dragoš i Milenko Petrić, ,,A table of connected graphs on six vertices”, Discrete Mathematics 50 (1984): 37–49.

\subsubsection{Wielościany z $n$ przekątnymi}
Jerzy Bednarczuk w $\Delta_{23}^{12}$ pokaże bez dowodu:

\begin{proposition}
    Istnieją co najmniej 3 wielościany wypukłe z jedną przekątną.
\end{proposition}

\begin{proposition}
    Istnieje co najmniej 8 wielościanów wypukłych z dwiema przekątnymi.     
\end{proposition}

%