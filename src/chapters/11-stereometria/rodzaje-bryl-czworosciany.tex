%

\subsection{Czworościany} % tetrahedron

W każdym czworościanie istnieje wierzchołek, przy którym trzy kąty płąskie są ostre. % https://www.deltami.edu.pl/2013/12/katy-trojscienne/

\begin{proposition}
    Niech $ABCD$ będzie czworościanem.
    Wtedy następujące warunki są równoważne:
    \begin{enumerate}
        \item Wszystkie ściany są przystające.
        \item Wszystkie ściany są trójkątami ostrokątnymi o takim samym promieniu okręgu opisanego.
        \item Suma kątów płaskich przy każdym wierzchołku wynosi $\pi$.
        \item Sumy kątów płaskich przy trzech dowolnych wierzchołkach wynosi $\pi$.
        \item Siatka czworościanu jest trójkątem ostrokątnym podzielonym na cztery przystające trójkąty.
        \item Kąty $\angle BAC$, $\angle ABD$, $\angle ACD$, $\angle BDC$ są równe.
        \item Przeciwległe krawędzie są równe.
        \item Trzy odcinki łączące środki przeciwległych krawędzi są parami prostopadłe.
        \item Wszystkie ściany mają równe pola.
        \item Rzut czworościanu $ABCD$ na dowolną płaszczyznę równoległą do dwóch przeciwległych krawędzi jest prostokątem.
        \item Każdy odcinek łączący środki przeciwległych krawędzi jest prostopadły do tych krawędzi.
    \end{enumerate} 
\end{proposition}

Dowód wszystkich 11 implikacji znajdziemy w artykulu Pompego, $\Delta_{94}^3$. % https://www.deltami.edu.pl/1994/03/o-czworoscianie-rownosciennym/

Czworościan foremny ma trzy osie symetii: to proste przechodzące przez środki przeciwległych krawędzi. % https://www.deltami.edu.pl/1993/04/rozne-roznosci/

%