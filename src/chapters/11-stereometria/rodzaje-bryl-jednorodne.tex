%

\subsection{Wielościany jednorodne}

\begin{definition}
    Wierzchołkowo przechodni wielościan, którego ściany są foremne, nazywamy jednorodnym.
\end{definition}

Podobnież dzieli się je na foremne (jeśli są ściennie i krawędziowo przechodnie), kwazi-foremne (tylko krawędziowo przechodnie) i semiforemne (jeśli nie są ani takie, ani takie).
Znamy wszystkie wielościany jednorodne:
\begin{itemize}
\item 5 brył platońskich,
\item 13 wielościanów archimedesowych (2 kwazi-regularne, 11 semi-regularnych),
\item 4 wielościany Keplera-Poinsota,
\item 53 jednorodne wielościany gwiaździste (14 kwazi, 39 semi).
\end{itemize}

Oprócz tego istnieją dwie nieskończone serie wielościanów jednorodnych: graniastosłupy oraz antygraniastosłupy, tak jak na stronie \pageref{dwie_serie}.

%