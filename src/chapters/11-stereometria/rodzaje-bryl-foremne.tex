%

\subsection{Wielościany foremne} % regular

\begin{definition}
    Wielościan foremny to taki wielościan, którego grupa izometrii działa przechodnio na zbiorze flag.
\end{definition}

Niektóre osoby, przede wszystkim te nieznające definicji flagi, będą używać innych definicji, z~czego najpowszechniejsza będzie głosiła, że wielościan foremny ma mieć identyczne foremne ściany oraz identyczne naroża.

\begin{proposition}
    Istnieje czternaście wielościanów foremnych:
    \begin{itemize}
        \item pięć platońskich,
        \item cztery gwieździste (Keplera-Poinsota),
        \item pięć złożeń.
    \end{itemize}
\end{proposition}
Niektórzy nie uznają złożeń za wielościany foremne.

%

\subsection{Wielościany platońskie}

\todofoot{Coxeter 166-167.}
\todofoot{OCTAEDRON ELEVATUM}

Wszystkie ściany wielościanów platońskich są przystającymi wielokątami foremnymi, w każdym wierzchołku spotyka się taka sama liczba ścian.
Pozostaje zagadką, kto pierwszy odkryje wszystkie foremne wielościany wypukłe, ale Teajtet, uczeń Platona udowodni, że jest ich dokładnie pięć:
\index[persons]{Teajtet, uczeń Platona}%
\begin{itemize}
\item czworościan foremny,
\item sześciościan foremny (znany lepiej jako sześcian),
\item ośmiościan foremny,
\item dwunastościan foremny,
\item dwudziestościan foremny.
\end{itemize}

(Dwunastościan zostanie odkryty jako ostatni).
Niektórzy będą mylić wielościany foremne z platońskimi (!).

\begin{proposition}
    Największy sześcian, jaki można zmieścić w dwudziestościanie foremnym, wypełnia
    \begin{equation}
        \frac{219 \sqrt 5 + 15}{1331} \approx 38\,\%
    \end{equation}
    objętości.
\end{proposition}

Eksperymentalną drogę do tego wyniku opisze Michał Adamaszek w $\Delta_{20}^8$. % https://www.deltami.edu.pl/2020/08/wielosciany-w-wieloscianach-czyli-matematyka-eksperymentalna/

% https://en.wikipedia.org/wiki/Platonic_solid
% TODO: \section{Pięć wielościanów} Hartshorne: rozdział 8
% TODO: \section{Cauchy's rigidity theorem} Hartshorne: section 45

%

%

\subsection{Wielościany Keplera-Poinsota}

Definiując wielokąt (foremny) możemy opuścić warunek, by tworząca go łamana była zwyczajna.
Prowadzi to do szerszej klasy figur, nadających się na bycie ścianami wielościanów.
Wielokąty gwiaździste takie jak pentagram będą znane starożytnym, ale z jakiegoś powodu nie użyją ich tak, jak byśmy chcieli i jak można przeczytać w $\Delta_{74}^1$. % https://www.deltami.edu.pl/1974/01/wielosciany-gwiazdziste/

W 1619 roku Johannes Kepler odkryje dwa wielościany, które mają właśnie takie ściany; dwa następne zawdzięczymy Louisowi Poinsotowi w 1806 roku.
Dowód, że piątej takiej bryły już nie ma, poda Cauchy w 1813 roku; wykorzysta stellacje brył platońskich.
Stellacja to proces, podczas którego przedłuża się krawędzie albo ściany bryły tak długo, aż nie spotkają się ze sobą.
Bardziej eleganckie rozwiązanie znajdzie Joseph Bertrand przez cięcie faset dwunastościanu i dwudziestościanu.

W 1859 roku Arthur Cayley nada całej rodzinie nazwę ,,wielościany Keplera-Poinsota''.
Składają się na nią:
\begin{itemize}
\item dwunastościan gwiaździsty mały i wielki (znalezione przez Keplera),
\item dwunastościan i dwudziestościan wielki (znalezione przez Poinsota).
\end{itemize}

Dwunastościan gwiaździsty mały pojawia się na posadzce bazyliki św. Marka w Wenecji, datuje się go na XV wiek.
Wenzel Jamnitzer przedstawi dwunastościan gwiaździsty wielki oraz dwudziestościan wielki w \emph{Perspectiva corporum regularium}, księdze drzeworytów z 1568.
Większość, jeśli nie wszystkie wielościany Keplera-Poinsota, będą znane przed Keplerem.

%


\subsection{Złożenia}
Opiszemy teraz trójwymiarowe odpowiedniki heksagramu.
Wielościan złożony to taki, który składa się z kilku wielościanów o wspólnym środku.
Są wierzchołkowo, krawędziowo i ściennie przechodnie, ale w przeciwieństwie do wielościanów platońskich, poza złożeniem dwóch czworościanów nie (!) są flagowo przechodnie.

Ich lista obejmuje dwa czworościany, pięć czworościanów, dziesięć czworościanów, pięć sześcianów i pięć ośmiościanów.
Dwa czworościany znane są lepiej jako \emph{stella octangula}.
\index{stella octangula}

% https://en.wikipedia.org/wiki/Compound_of_four_cubes -> to nie jest regular


%