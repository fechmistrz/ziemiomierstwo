%

\subsection{Wielościany Johnsona-Zalgallera}

\begin{definition}
    Wielościan wypukły, który nie jest jednorodny, ale wszystkie jego ściany są foremne, nazywamy wielościanem Johnsona albo Johnsona-Zalgallera.
\end{definition}

Norman Woodason Johnson \cite{johnson_1966} opublikuje w pracy doktorskiej z 1966 roku, a spisanej pod opieką samego Coxetera, listę 92 brył, które spełniają ten warunek, zaś Wiktor Abramowicz Zalgaller \cite{zalgaller_1969} udowodni po trzech latach (1969), że lista ta jest kompletna.
\index[persons]{Johnson, Norman}%
\index[persons]{Coxeter, ???}%
\index[persons]{Zalgaller, Wiktor}%
Często oznacza się je po prostu $J_{\ldots}$, gdzie indeks dolny mówi, którą pozycję na liście zajmuje.

\begin{compactitem}
    \item piramidy: \begin{compactitem}
        \item \textbf{piramida} (czworokątna: $J_1$, pięciokątna: $J_2$),
        \item wydłużona \textbf{piramida} (trójkątna: $J_7$, czworokątna: $J_8$, pięciokątna: $J_9$),
        \item skrętnie wydłużona \textbf{piramida} (czworokątna: $J_{10}$, pięciokątna $J_{11}$),
        \item \textbf{dwupiramida} (trójkątna: $J_{12}$, pięciokątna: $J_{13}$),
        \item wydłużona \textbf{dwupiramida} (trójkątna: $J_{14}$, czworokątna: $J_{15}$, pięciokątna: $J_{16}$),
        \item skrętnie wydłużona \textbf{dwupiramida} czworokątna: $J_{17}$,
    \end{compactitem}

    \item kopuły: \begin{compactitem}
        \item \textbf{kopuła} (trójkątna: $J_3$, czworokątna: $J_4$, pięciokątna: $J_5$),
        \item wydłużona \textbf{kopuła} (trójkątna: $J_{18}$, czworokątna: $J_{19}$, pięciokątna: $J_{20}$),
        \item skrętnie wydłużona \textbf{kopuła} (trójkątna: $J_{22}$, czworokątna: $J_{23}$, pięciokątna: $J_{24}$),
        \item podwójna \textbf{kopuła} (trójkątna: $J_{27}$, czworokątna: $J_{28}$, czworokątna skręcona: $J_{29}$,\\
        pięciokątna: $J_{30}$),
        \item \textbf{dwukopuła} pięciokątna skręcona: $J_{31}$,
        \item wydłużona \textbf{dwukopuła} (trójkątna: $J_{35}$, trójkątna skręcona: $J_{36}$, czworokątna skręcona: $J_{37}$, pięciokątna: $J_{38}$, pięciokątna skręcona: $J_{39}$),
        \item skrętnie wydłużona \textbf{dwukopuła} (trójkątna: $J_{44}$, czworokątna: $J_{45}$, pięciokątna: $J_{46}$),
    \end{compactitem}

    \item kopuło-rotundy: \begin{compactitem}
        \item \textbf{kopuło-rotunda} (pięciokątna: $J_{32}$, pięciokątna skręcona: $J_{33}$),
        \item wydłużona \textbf{kopuło-rotunda} (pięciokątna: $J_{40}$, pięciokątna skręcona: $J_{41}$),
        \item skrętnie wydłużona \textbf{kopuło-rotunda} pięciokątna: $J_{47}$,
    \end{compactitem}

    \item rotundy: \begin{compactitem}
        \item \textbf{rotunda} pięciokątna: $J_6$
        \item wydłużona \textbf{rotunda} pięciokątna: $J_{21}$,
        \item skrętnie wydłużona \textbf{rotunda} pięciokątna: $J_{25}$,
        \item \textbf{dwurotunda} pięciokątna: $J_{34}$,
        \item wydłużona \textbf{dwurotunda} (pięciokątna: $J_{42}$, pięciokątna skręcona: $J_{43}$),
        \item skrętnie wydłużona \textbf{dwurotunda} pięciokątna: $J_{48}$,
        \item dwusoczewkowa \textbf{rotunda} podwójna: $J_{91}$,
        \item \textbf{hebeklinorotunda} trójkątna: $J_{92}$,
    \end{compactitem}

    \item klino-... \begin{compactitem}
        \item \textbf{dwuklinoid} przycięty: $J_{84}$,
        \item \textbf{klinocingulum} podwójne: $J_{90}$,
    \end{compactitem}

    \item korony: \begin{compactitem}
        \item \textbf{klinokorona}: $J_{86}$,
        \item powiększona \textbf{klinokorona}: $J_{87}$,
        \item \textbf{klinomegakorona}: $J_{88}$, 
        \item \textbf{hebeklinomegakorona}: $J_{89}$,
    \end{compactitem}

    \item graniastosłupy: \begin{compactitem}
        \item podwójny \textbf{graniastosłup} trójkątny: $J_{26}$,
        \item powiększony \textbf{graniastosłup} (trójkątny: $J_{49}$, pięciokątny: $J_{52}$, sześciokątny: $J_{54}$),
        \item podwójnie powiększony \textbf{graniastosłup} (trójkątny: $J_{50}$, pięciokątny: $J_{53}$, \\
        sześciokątny: $J_{56}$),

        \item podwójnie osiowo powiększony \textbf{graniastosłup} sześciokątny: $J_{55}$,
        \item potrójnie powiększony \textbf{graniastosłup} (trójkątny: $J_{51}$, sześciokątny: $J_{57}$),
    \end{compactitem}

    \item \textbf{antygraniastosłup} czworokątny przycięty: $J_{85}$,

    \item powiększony \textbf{czworościan} ścięty $J_{65}$,

    \item sześciany: \begin{compactitem}
        \item powiększony \textbf{sześcian} ścięty $J_{66}$,
        \item podwójnie powiększony \textbf{sześcian} ścięty $J_{67}$,
    \end{compactitem}

    \item dwunastościany: \begin{compactitem}
        \item powiększony \textbf{dwunastościan} $J_{58}$,
        \item (podwójnie osiowo: $J_{59}$, podwójnie: $J_{60}$, potrójnie: $J_{61}$) powiększony \textbf{\textbf{dwunastościan}},
        \item powiększony \textbf{dwunastościan} ścięty $J_{68}$,
        \item (podwójnie osiowo:$J_{69}$, podwójnie: $J_{70}$, potrójnie: $J_{71}$) powiększony \textbf{dwunastościan} ścięty,
    \end{compactitem}

    \item dwudziesto-dwunastościany: \begin{compactitem}
        \item \textbf{dwudziesto-dwunastościan} rombowy (skręcony: $J_{72}$, podwójnie osiowo skręcony $J_{73}$, podwójnie skośnie skręcony $J_{74}$, potrójnie skręcony $J_{75}$, obcięty $J_{76}$, podwójnie osiowo obcięty $J_{80}$, podwójnie skośnie obcięty $J_{81}$, potrójnie obcięty $J_{83}$),
        \item przekręcony osiowo \textbf{dwudziesto-dwunastościan} rombowy obcięty: $J_{77}$,
        \item przekręcony skośnie \textbf{dwudziesto-dwunastościan} rombowy obcięty: $J_{78}$,
        \item podwójnie przekręcony \textbf{dwudziesto-dwunastościan} rombowy obcięty: $J_{79}$,
        \item przekręcony \textbf{dwudziesto-dwunastościan} rombowy podwójnie obcięty: $J_{82}$,
    \end{compactitem}

    \item dwudziestościany: \begin{compactitem}
        \item podwójnie obcięty \textbf{dwudziestościan}: $J_{62}$,
        \item potrójnie obcięty \textbf{dwudziestościan}: $J_{63}$,
        \item powiększony potrójnie obcięty \textbf{dwudziestościan}: $J_{64}$.
    \end{compactitem}
\end{compactitem}

% https://en.wikipedia.org/wiki/Johnson_solid
% http://matematyka.wroc.pl/book/wielosciany-johnsona
% {Siamese dodecahedron} https://en.wikipedia.org/wiki/Snub_disphenoid

%