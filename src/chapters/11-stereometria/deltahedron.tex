%

Martyn Cundy zaproponuje przez podobieństwo greckiej litery $\Delta$ do trójkąta równobocznego, by rozważać pojęcie takie jak \emph{deltahedron} (nie mylić z \emph{deltohedron}, którego ściany są latawcami):
\index{Cundy, Henry Martyn}%

\begin{definition}[deltościan]
    Wielościan, którego wszystkie ściany są trójkątami równobocznymi, nazywamy deltościanem.
\end{definition}

\begin{proposition}
    Jest osiem deltościanów wypukłych: trzy bryły platońskie (o czterech, ośmiu albo dwudziestu ścianach) oraz pięć wielościanów Johnsona ($J_{12}$, $J_{13}$, $J_{17}$, $J_{51}$, $J_{84}$).
\end{proposition}

Pokażą to najpierw Otto Rausenberger \cite{rausenberger_1915} i niezależnie, choć dużo później Hans Freudenthal oraz Bartel Leendert van der Waerden \cite{freudenthal_1947} w obskurnym duńskim żurnalu.
% https://mathscinet.ams.org/mathscinet/relay-station?mr=0021687
\index[persons]{Freudenthal, Hans}%
\index[persons]{van der Warden, Bartel}%
Ich wynik zreferuje później Adam Gajda w $\Delta_{84}^{4}$ (bez wzmianki o Rausenbergerze).
Niewypukłych brył o tej własności jest nieskończnie wiele, mogą mieć dowolną parzystą liczbę ścian większą niż sześć.

https://www.deltami.edu.pl/2016/08/brzydka-prawda/

%