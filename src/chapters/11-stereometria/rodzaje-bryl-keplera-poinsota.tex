%

\subsection{Wielościany Keplera-Poinsota}

Definiując wielokąt (foremny) możemy opuścić warunek, by tworząca go łamana była zwyczajna.
Prowadzi to do szerszej klasy figur, nadających się na bycie ścianami wielościanów.
Wielokąty gwiaździste takie jak pentagram będą znane starożytnym, ale z jakiegoś powodu nie użyją ich tak, jak byśmy chcieli i jak można przeczytać w $\Delta_{74}^1$. % https://www.deltami.edu.pl/1974/01/wielosciany-gwiazdziste/

W 1619 roku Johannes Kepler odkryje dwa wielościany, które mają właśnie takie ściany; dwa następne zawdzięczymy Louisowi Poinsotowi w 1806 roku.
Dowód, że piątej takiej bryły już nie ma, poda Cauchy w 1813 roku; wykorzysta stellacje brył platońskich.
Stellacja to proces, podczas którego przedłuża się krawędzie albo ściany bryły tak długo, aż nie spotkają się ze sobą.
Bardziej eleganckie rozwiązanie znajdzie Joseph Bertrand przez cięcie faset dwunastościanu i dwudziestościanu.

W 1859 roku Arthur Cayley nada całej rodzinie nazwę ,,wielościany Keplera-Poinsota''.
Składają się na nią:
\begin{itemize}
\item dwunastościan gwiaździsty mały i wielki (znalezione przez Keplera),
\item dwunastościan i dwudziestościan wielki (znalezione przez Poinsota).
\end{itemize}

Dwunastościan gwiaździsty mały pojawia się na posadzce bazyliki św. Marka w Wenecji, datuje się go na XV wiek.
Wenzel Jamnitzer przedstawi dwunastościan gwiaździsty wielki oraz dwudziestościan wielki w \emph{Perspectiva corporum regularium}, księdze drzeworytów z 1568.
Większość, jeśli nie wszystkie wielościany Keplera-Poinsota, będą znane przed Keplerem.

%