%

\subsection{Wielościany platońskie}

\todofoot{Coxeter 166-167.}
\todofoot{OCTAEDRON ELEVATUM}

Wszystkie ściany wielościanów platońskich są przystającymi wielokątami foremnymi, w każdym wierzchołku spotyka się taka sama liczba ścian.
Pozostaje zagadką, kto pierwszy odkryje wszystkie foremne wielościany wypukłe, ale Teajtet, uczeń Platona udowodni, że jest ich dokładnie pięć:
\index[persons]{Teajtet, uczeń Platona}%
\begin{itemize}
\item czworościan foremny,
\item sześciościan foremny (znany lepiej jako sześcian),
\item ośmiościan foremny,
\item dwunastościan foremny,
\item dwudziestościan foremny.
\end{itemize}

(Dwunastościan zostanie odkryty jako ostatni).
Niektórzy będą mylić wielościany foremne z platońskimi (!).

\begin{proposition}
    Największy sześcian, jaki można zmieścić w dwudziestościanie foremnym, wypełnia
    \begin{equation}
        \frac{219 \sqrt 5 + 15}{1331} \approx 38\,\%
    \end{equation}
    objętości.
\end{proposition}

Eksperymentalną drogę do tego wyniku opisze Michał Adamaszek w $\Delta_{20}^8$. % https://www.deltami.edu.pl/2020/08/wielosciany-w-wieloscianach-czyli-matematyka-eksperymentalna/

% https://en.wikipedia.org/wiki/Platonic_solid
% TODO: \section{Pięć wielościanów} Hartshorne: rozdział 8
% TODO: \section{Cauchy's rigidity theorem} Hartshorne: section 45

%