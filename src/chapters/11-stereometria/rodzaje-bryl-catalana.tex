%

\subsection{Wielościany Catalana}

\begin{definition}[wielościan dualny]
    W ustalonym wielościanie usuńmy wierzchołki i krawędzie, zastąpmy każdą ścianę jej środkiem, po czym połączmy te punkty, które należały do sąsiednich ścian.
    Jeśli otrzymany obiekt jest wielościanem, to nazywamy go wielościanem dualnym.
\end{definition}

Bryły dualne do wielościanów platońskich są platońskie, a dualne do Keplera-Poinsota są bryłami Keplera-Poinsota, więc nie dają żadnych nowych obiektów.
Bryła dualna do bryły dualnej jest tego samego kształtu, co wyjściowa.

\begin{definition}[wielościan Catalana]
    Wielościan dualny do wielościanu archimedesowego nazywamy wielościanem Catalana.
\end{definition}

Pierwsze wielościany Catalana znajdzie Kepler, ale dopiero Eugene Catalan skompletuje je w 1865 roku. % Some Catalan solids were discovered by Johannes Kepler during his study of zonohedra
Nazwę sześćdziestościan deltoidowy zaproponuje profesor Roman Duda podczas tłumaczenia książki ,,Modele matematyczne'' Cundy'ego i Rolleta; brzmi to trochę dziwnie, ale przyjmie się.

\begin{proposition}
    Wielościany Catalana są ściennie przechodnie, ale nie wierzchołkowo przechodnie.
\end{proposition}

\begin{proposition}
    W wielościanach Catalana wszystkie kąty dwuścienne są równe. % Each Catalan solid has constant dihedral angles, meaning the angle between any two adjacent faces is the same.[1]
\end{proposition}



%