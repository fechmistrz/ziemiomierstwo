%

\subsubsection{Pryzmatoidy} % prism? graniastosłup?
Pryzmatoid to bryła, której wierzchołki leżą na dwóch płaszczyznach równoległych.

\begin{example}[pyramids]
    Ostrosłup, gdzie jedna płaszczyzna zawiera tylko wierzchołek.
\end{example}

\begin{example}[wedges]
    (Klin?), gdzie jedna płaszczyzna zawiera tylko dwa wierzchołki.
\end{example}

\begin{example}[parallelepipeds]
    Quadrilateral-faced hexahedral prismatoids:
        Parallelepipeds – six parallelogram faces
        Rhombohedrons – six rhombus faces
        Trigonal trapezohedra – six congruent rhombus faces
        Cuboids – six rectangular faces
        Quadrilateral frusta – an apex-truncated square pyramid
        Cube – six square faces
\end{example}

\begin{example}[prisms]
    Graniastosłup, gdzie płaszczyzny są przystającymi wielokątami, zaś ściany boczne to równoległoboki.
\end{example}

\begin{example}[antiprisms]
    % Antiprisms, whose polygons in each plane are congruent and joined by an alternating strip of triangles;[5]
    % Star antiprisms;
\end{example}

\begin{example}[cupolae]
    % Cupolae, in which the polygon in one plane contains twice as many points as the other and is joined to it by alternating triangles and rectangles;
    % https://en.wikipedia.org/wiki/Cupola_(geometry)
\end{example}

\begin{example}[frusta]
    Frusta obtained by truncation of a pyramid or a cone;
    % https://en.wikipedia.org/wiki/Frustum
\end{example}

\begin{proposition}
    Objętość pryzmatoidu wyraża wzór
    \begin{equation}
        V = \frac 1 6 h (S_l + 4S_m + 4S_h),
    \end{equation}
    gdzie $h$ oznacza wysokość (odległość między płaszczyznami podstaw), $S_l$ pole podstawy dolnej, $S_h$ pole podstawy górnej, zaś $S_m$ to pole przekroju poprzecznego płaszczyzną, która leży w połowie między podstawami. % eves s. 40
\end{proposition}

\begin{proof}
    Wzór wynika z całkowania metodą Simsona, która daje dokładny wynik dla wielomianów stopnia co najwyżej trzeciego; pole przekroju poprzecznego stanowi funkcję kwadratową wysokości.
\end{proof}

% TODO https://en.wikipedia.org/wiki/Prismatoid#Prismatoid_families

%