
\section{Do przerobienia}

\todofoot{https://www.deltami.edu.pl/2011/01/kacik-przestrzenny-6-czworosciany-ortocentryczne/}
\todofoot{https://www.deltami.edu.pl/2011/02/flexor-connellyego/}
\todofoot{https://www.deltami.edu.pl/2013/03/kacik-przestrzenny-16-sfera-styczna-do-krawedzi-czworoscianu/}
\todofoot{https://www.deltami.edu.pl/2013/06/kacik-przestrzenny-18-o-pozytku-ze-sfery-wpisanej/}
\todofoot{https://www.deltami.edu.pl/2013/09/wielosciany-drzace-i-wielosciany-multistabilne/}
\todofoot{https://www.deltami.edu.pl/2013/12/kacik-przestrzenny-20-sfery-dandelina/}
\todofoot{https://www.deltami.edu.pl/2013/12/stozkowe/}
\todofoot{https://www.deltami.edu.pl/2014/08/dwa-w-jednym/}
\todofoot{https://www.deltami.edu.pl/2014/10/a-jednak-istnieje/}
\todofoot{https://www.deltami.edu.pl/2014/12/kazdy-trojkat-jest-rownoboczny/}
\todofoot{https://www.deltami.edu.pl/2016/03/w-k-s-2/}
\todofoot{https://www.deltami.edu.pl/2017/04/niemozliwe-wycinanki/}
\todofoot{https://www.deltami.edu.pl/2017/06/dwie-sfery-w-jednym-miejscu/}
\todofoot{https://www.deltami.edu.pl/2017/07/najpiekniejsze-zadanie-geometryczne/}
\todofoot{https://www.deltami.edu.pl/2022/11/konstrukcja-sfer-dopisanych-do-czworoscianu/}

KratkaKratkaKratka historia

KratkaKratkaKratkaKratka Regular convex polyhedra

* The[Platonic solids](https://en.wikipedia.org/wiki/Platonic%_solid "Platonic solid") date back to the classical Greeks and were studied by the[Pythagoreans](https://en.wikipedia.org/wiki/Pythagoreanism "Pythagoreanism"),[Plato](https://en.wikipedia.org/wiki/Plato "Plato") (c. 424 – 348 BC),[Theaetetus](https://en.wikipedia.org/wiki/Theaetetus%_(mathematician)) "Theaetetus (mathematician)") (c. 417 BC – 369 BC),[Timaeus of Locri](https://en.wikipedia.org/wiki/Timaeus%_of%_Locri "Timaeus of Locri") (c. 420–380 BC), and[Euclid](https://en.wikipedia.org/wiki/Euclid "Euclid") (fl. 300 BC). The[Etruscans](https://en.wikipedia.org/wiki/Etruscans "Etruscans") discovered the regular dodecahedron before 500 BC.^[**[**3**]**](https://en.wikipedia.org/wiki/Uniform%_polyhedronKratkacite%_note-3)^

KratkaKratkaKratkaKratka Nonregular uniform convex polyhedra

* The[cuboctahedron](https://en.wikipedia.org/wiki/Cuboctahedron "Cuboctahedron") was known by[Plato](https://en.wikipedia.org/wiki/Plato "Plato").

* [Archimedes](https://en.wikipedia.org/wiki/Archimedes "Archimedes") (287 BC – 212 BC) discovered all of the 13[Archimedean solids](https://en.wikipedia.org/wiki/Archimedean%_solid "Archimedean solid"). His original book on the subject was lost, but[Pappus of Alexandria](https://en.wikipedia.org/wiki/Pappus%_of%_Alexandria "Pappus of Alexandria") (c. 290 – c. 350 AD) mentioned Archimedes listed 13 polyhedra.

* [Piero della Francesca](https://en.wikipedia.org/wiki/Piero%_della%_Francesca "Piero della Francesca")

(1415 – 1492) rediscovered the five truncations of the Platonic

solids—truncated tetrahedron, truncated octahedron, truncated cube,

truncated dodecahedron, and truncated icosahedron—and included

illustrations and calculations of their metric properties in his book*[De quinque corporibus regularibus](https://en.wikipedia.org/wiki/De%_quinque%_corporibus%_regularibus "De quinque corporibus regularibus")* . He also discussed the cuboctahedron in a different book.^[**[**4**]**](https://en.wikipedia.org/wiki/Uniform%_polyhedronKratkacite%_note-4)^

* [Luca Pacioli](https://en.wikipedia.org/wiki/Luca%_Pacioli "Luca Pacioli") plagiarized Francesca's work in*[De divina proportione](https://en.wikipedia.org/wiki/De%_divina%_proportione "De divina proportione")* in 1509, adding the[rhombicuboctahedron](https://en.wikipedia.org/wiki/Rhombicuboctahedron "Rhombicuboctahedron"), calling it an*icosihexahedron* for its 26 faces, which was drawn by[Leonardo da Vinci](https://en.wikipedia.org/wiki/Leonardo%_da%_Vinci "Leonardo da Vinci").

* [Johannes Kepler](https://en.wikipedia.org/wiki/Johannes%_Kepler "Johannes Kepler") (1571–1630) was the first to publish the complete list of[Archimedean solids](https://en.wikipedia.org/wiki/Archimedean%_solid "Archimedean solid"), in 1619. He also identified the infinite families of uniform[prisms and antiprisms](https://en.wikipedia.org/wiki/Prismatic%_uniform%_polyhedron "Prismatic uniform polyhedron").

KratkaKratkaKratka Other 53 nonregular star polyhedra

* Of the remaining 53,[Edmund Hess](https://en.wikipedia.org/wiki/Edmund%_Hess "Edmund Hess")

(1878) discovered 2, Albert Badoureau (1881) discovered 36 more, and

Pitsch (1881) independently discovered 18, of which 3 had not previously

been discovered. Together these gave 41 polyhedra.

* The geometer[H.S.M. Coxeter](https://en.wikipedia.org/wiki/Harold%_Scott%_MacDonald%_Coxeter "Harold Scott MacDonald Coxeter") discovered the remaining twelve in collaboration with[J. C. P. Miller](https://en.wikipedia.org/wiki/J.%_C.%_P.%_Miller "J. C. P. Miller") (1930–1932) but did not publish.[M.S. Longuet-Higgins](https://en.wikipedia.org/wiki/Michael%_S.%_Longuet-Higgins "Michael S. Longuet-Higgins") and[H.C. Longuet-Higgins](https://en.wikipedia.org/wiki/H.C.%_Longuet-Higgins "H.C. Longuet-Higgins") independently discovered eleven of these. Lesavre and Mercier rediscovered five of them in 1947.

* [Coxeter, Longuet-Higgins % &amp; Miller (1954)](https://en.wikipedia.org/wiki/Uniform%_polyhedronKratkaCITEREFCoxeterLonguet-HigginsMiller1954) published the list of uniform polyhedra.

* [Sopov (1970)](https://en.wikipedia.org/wiki/Uniform%_polyhedronKratkaCITEREFSopov1970) proved their conjecture that the list was complete.

* In 1974,[Magnus Wenninger](https://en.wikipedia.org/wiki/Magnus%_Wenninger "Magnus Wenninger") published his book[*Polyhedron models*](https://en.wikipedia.org/wiki/List%_of%_Wenninger%_polyhedron%_models "List of Wenninger polyhedron models"), which lists all 75 nonprismatic uniform polyhedra, with many previously unpublished names given to them by[Norman Johnson](https://en.wikipedia.org/wiki/Norman%_Johnson%_(mathematician)) "Norman Johnson (mathematician)").

* [Skilling (1975)](https://en.wikipedia.org/wiki/Uniform%_polyhedronKratkaCITEREFSkilling1975)

independently proved the completeness and showed that if the definition

of uniform polyhedron is relaxed to allow edges to coincide then there

is just one extra possibility (the[great disnub dirhombidodecahedron](https://en.wikipedia.org/wiki/Great%_disnub%_dirhombidodecahedron "Great disnub dirhombidodecahedron")).

* In 1987,[Edmond Bonan](https://en.wikipedia.org/wiki/Edmond%_Bonan "Edmond Bonan") drew all the uniform polyhedra and their duals in 3D with a Turbo Pascal program called**Polyca** .

Most of them were shown during the International Stereoscopic Union

Congress held in 1993, at the Congress Theatre, Eastbourne, England; and

again in 2005 at the Kursaal of Besançon, % France.^[**[**5**]**](https://en.wikipedia.org/wiki/Uniform%_polyhedronKratkacite%_note-5)^

* In 1993, Zvi Har'El % (1949–2008)^[**[**6**]**](https://en.wikipedia.org/wiki/Uniform%_polyhedronKratkacite%_note-6)^ produced a complete kaleidoscopic construction of the uniform polyhedra and duals with a computer program called**Kaleido** and summarized it in a paper*Uniform Solution for Uniform Polyhedra* , counting figures 1-80.^[**[**7**]**](https://en.wikipedia.org/wiki/Uniform%_polyhedronKratkacite%_note-7)^

* Also in 1993, R. Mäder ported this Kaleido solution to[Mathematica](https://en.wikipedia.org/wiki/Mathematica) with a slightly different % indexing system.^[**[**8**]**](https://en.wikipedia.org/wiki/Uniform%_polyhedronKratkacite%_note-8)^

* In 2002 Peter W. Messer discovered a minimal set of closed-form

expressions for determining the main combinatorial and metrical

quantities of any uniform polyhedron (and its dual) given only its[Wythoff symbol](https://en.wikipedia.org/wiki/Wythoff%_symbol "Wythoff symbol").^[**[**9**]**](https://en.wikipedia.org/wiki/Uniform%_polyhedronKratkacite%_note-9)^

KratkaKratkaKratka archimedesowe -> semiregular

Kepler coined kategorię wielościanów półforemnych w *Harmonice Mundi* (1619) jako 13 wielościanów archimedesowych, graniastosłupy, antygraniastosłupy, i dwa wielościany Catalana, krawędziowo tranzytywne **rhombic dodecahedron + rhombic triacontahedron** i niejawnie **trigonal trapezohedron**.

Thorold Gosset zdefiniuje (*Thorold Gosset On the Regular and Semi-Regular Figures in Space of n Dimensions, Messenger of Mathematics, Macmillan, 1900*) ogólniejsze pojęcie wielotopu półforemnego, z którego wynika, że

Coxeter et al. 1954 używali terminu "wielościany półforemne" do sklasyfikowania jednorodnych wielościanów, których symbol Wythoffa to "p q | r", co obejmuje tylko sześć wielościanów archimedesowych, graniastosłupy, ale nie antygraniastosłupy, ale za to liczne niewypukłe bryły.

W 1973 zmieni zdanie i przyjmie definicje Gosseta bez komentarza.

Cromwell 1997 napisze, że wielościany półforemne to archimedesowe, Catalana, chociaz w dalszej czesci (tej samej książki???) uzna wielościany Catalana, że nie są półforemne.

https://en.wikipedia.org/wiki/Semiregular%_polyhedron

https://en.wikipedia.org/wiki/Quasiregular%_polyhedron => raczej krtkie

frustrum (frusta) to część bryły wycięta przez dwie płaszczyzny równoległ

% https://en.wikipedia.org/wiki/Polyhedron
% https://en.wikipedia.org/wiki/Dual_polyhedron
% https://www.youtube.com/watch?v=yAEveAOH2KwI => Schwarz lantern

%