\subsection{Ostrosłupy} % pyramid

\begin{definition}[ostrosłup]
    A pyramid is a polyhedron that may be formed by connecting each vertex in a planar polygon to a point lying outside that plane. This point is called the pyramid's apex, and the planar polygon is the pyramid's base. Each other face of the pyramid is a triangle[1] consisting of one of the base's edges, and the two edges connecting that edge's endpoints to the apex. These faces are called the pyramid's lateral faces, and each edge connected to the apex is called a lateral edge.[2] 
\end{definition}

The terms "right pyramid" and "regular pyramid" are used to describe special cases of pyramids. Their common notions are as follows. A regular pyramid is one with a regular polygon as its base. A right pyramid is one where the axis (the line joining the centroid of the base and the apex) is perpendicular to the base.[6][7][8] An oblique pyramid is one where the axis is not perpendicular to the base.[9] However, there are no standard definitions for these terms, and different sources use them somewhat differently.

The truncated pyramid is a pyramid cut off by a plane; if the truncation plane is parallel to the base of a pyramid, it is called a frustum.

Jeżeli spodek wysokości ostrosłupa pokrywa się ze środkiem okręgu opisanego na jego podstawie, to taki ostrosłup nazywamy ostrosłupem prostym. J


Ostrosłupy są szczególnymi przypadkami pryzmatoidów:

\begin{definition}
A prismatoid is defined as a polyhedron where its vertices lie on two parallel planes, with its lateral faces as triangles, trapezoids, and parallelograms.[4] \end{definition}

Ostrosłup ścięty – bryła powstała w wyniku przecięcia ostrosłupa płaszczyzną równoległą do podstawy ostrosłupa i odrzucenia punktów leżących po stronie jego wierzchołka[1].
PO ANGIELSKU JEST INACZEJ
The truncated pyramid is a pyramid cut off by a plane; if the truncation plane is parallel to the base of a pyramid, it is called a frustum.



Eves \cite[s.4]{eves1_1972} napisze, że około 1850 lat przed Chrystusem znany będzie wzór na dokładną objętość ostrosłupa czworokątnego ściętego, o podstawie długości $a$, $b$ i wysokości $h$:
\begin{equation}
	V = \frac 1 3 h (a^2 + ab + b^2).
\end{equation}
(Po angielsku taką bryłę nazywa się \emph{frustum}, co w łacinie znaczy \emph{kęs, kawałek}).
Wzór pojawi się bez dowodu w papirusie moskiewskim, nazwanym tak, ponieważ jego pierwszym właścicielem spoza Egiptu będzie Władimir Goleniszczew, rosyjski egiptolog i kolekcjoner sztuki.
\index[persons]{Goleniszczew, Władimir}%






