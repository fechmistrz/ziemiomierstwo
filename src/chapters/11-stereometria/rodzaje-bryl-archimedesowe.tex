
\subsection{Wielościany archimedesowe}

Wszystkie naroża wielościanów archimedesowych są identyczne, ich ściany to wielokąty foremne (ale nie muszą mieć tyle samo boków), a one same są wypukłe.
Zostaną odnalezione przez Archimedesa i chociaż jego książka na ten temat zaginie, Pappus z Aleksandrii wspomni, że Archimedes znał ,,wszystkie'' trzynaście.

Tak jak Grünbaum \cite{grunbaum_2009} wierzymy, że przed XIX wiekiem nikt nie będzie myśleć o działaniu grup na zbiory, przynajmniej w geometrii i dlatego przyjmujemy, że wielościan Johnsona $J_{37}$ też zasługuje na miano archimedesowego.
W każdym wierzchołku spotykają się trzy kwadraty i trójkąt, ale nie jest wierzchołkowo przechodni;  zapewne przez to pojawi się w druku dopiero w artykule Sommerville'a z 1905 roku, a potem zostanie przez przypadek odkryty na nowo: przez Jeffreya Charlesa Percy'ego Millera w 1930 i niezależnie przez Aszkinuze około 1957.
\index[persons]{Sommerville, Duncan}% % Duncan McLaren Young Sommerville
\index[persons]{Miller, Percy}%
\index[persons]{Aszkinuze, Władimir Gieorgijewicz}%
% https://www.mathedu.ru/indexes/authors/ashkinuze_v_g/

% angielskie słowo fiat zostaje użyte u Gruenbauma w tym kontekście
Dziwna tradycja każe wykluczać dwie nieskończone serie:
\label{dwie_serie}%
\begin{itemize}
\item póforemne graniastosłupy (o ścianach bocznych będących kwadratami),
\item półforemne antygraniastosłupy (o ścianach bocznych będących trójkątami równobocznymi)
\end{itemize} 
z~definicji wielościanów archimedesowych i półforemnych.

%