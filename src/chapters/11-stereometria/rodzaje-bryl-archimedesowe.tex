
\subsection{Wielościany archimedesowe}

Wszystkie naroża wielościanów archimedesowych są identyczne, ich ściany to wielokąty foremne (ale nie muszą mieć tyle samo boków), a one same są wypukłe.
Zostaną odnalezione przez Archimedesa i chociaż jego książka na ten temat zaginie, Pappus z Aleksandrii wspomni, że Archimedes znał wszystkie trzynaście.

Niektórzy (zaskakująco wiele osób) mieszają wielościany archimedesowe z półforemnymi, które oprócz wspomnianych trzynastu brył obejmują dwie nieskończone serie:
\label{dwie_serie}%
\begin{itemize}
\item póforemne graniastosłupy (o ścianach bocznych będących kwadratami)
\item półforemne antygraniastosłupy (o ścianach bocznych będących trójkątami równobocznymi).
\end{itemize} 

Branko Grünbaum uzna także $J_{37}$, wielościan Johnsona, za wielościan półforemny.
W każdym wierzchołku spotykają się trzy kwadraty i trójkąt, ale nie jest wierzchołkowo przechodni;  zapewne przez to pojawi się w druku dopiero w 1905 roku, a potem zostanie przez przypadek odkryty na nowo: przez Jeffreya Charlesa Percy'ego Millera w 1930 i niezależnie przez Ashkinuze.
% TODO: Ashkinuse, V.G.: On the number of semiregular polyhedra. -> rosyjski mathscinet

%