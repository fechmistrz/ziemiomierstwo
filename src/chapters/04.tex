%

\chapter{Pięciokąty i wyższe $n$-kąty}

\begin{definition}[wielokąt]
\index{wielokąt!pięciokąt}%
    Łamaną zamkniętą (sumę odcinków łączących kolejne punkty, aż do początkowego) bez samoprzecięć razem z jej wnętrzem nazywamy wielokątem.
\end{definition}

Złoty podział i pięciokąt; u Coxetera: strony 178-190. % TODO

% TODO: https://en.wikipedia.org/wiki/Golden_ratio

\begin{proposition}[Heineken]
\index{twierdzenie!Heinekena (o współpękowych przekątnych wielokąta)}%
    Jeśli $n \ge 5$ jest liczbą nieparzystą, to żadne trzy przekątne $n$-kąta foremnego nie przechodzą przez jeden punkt.
    % TODO: https://arxiv.org/pdf/math/9508209v3 ... In the 1960s, Heineken [6] gave a delightful argument which generalized this to all odd n,
\end{proposition}

% TODO: https://www.deltami.edu.pl/2020/12/odkryj-wielokat/

%